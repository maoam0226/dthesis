%\documentclass[a4paper,12pt,oneside,openany]{jsbook}
%%\setlength{\topmargin}{10mm}
%\addtolength{\topmargin}{-1in}
%\setlength{\oddsidemargin}{27mm}
%\addtolength{\oddsidemargin}{-1in}
%\setlength{\evensidemargin}{20mm}
%\addtolength{\evensidemargin}{-1in}
%\setlength{\textwidth}{160mm}
%\setlength{\textheight}{250mm}
%\setlength{\evensidemargin}{\oddsidemargin}

%\usepackage{ascmac}

\usepackage{color}
\usepackage{textcomp}
%\usepackage[dviout]{graphicx}
%\usepackage[dvipdfm]{graphicx,color}
\usepackage{wrapfig}
\usepackage{ccaption}
\usepackage{color}
%\usepackage{jumoline} %%行にまたがって下線を引ける、ダウンロードの必要有
\usepackage{umoline}
\usepackage{fancybox}
\usepackage{pifont}
\usepackage{indentfirst} %%最初の段落も字下げしてくれる

\usepackage{amsmath,amssymb,amsfonts}
\usepackage{bm}
%\usepackage{graphicx}
\usepackage[dvipdfmx]{graphicx}
%\usepackage[dvipsnames]{xcolor}
\usepackage{subfigure}
\usepackage{verbatim}
\usepackage{makeidx}
\usepackage{accents}
%\usepackage{slashbox} %%ダウンロードの必要有

\usepackage[dvipdfmx]{hyperref} %%pdfにリンクを貼る
\usepackage{pxjahyper}

\usepackage[flushleft]{threeparttable}
\usepackage{array,booktabs,makecell}

\usepackage{geometry}
\geometry{left=30mm,right=30mm,top=50mm,bottom=5mm}

\usepackage[super]{nth} %1st, 2nd ...を出力
\usepackage{dirtytalk} %クォーテーションマーク
\usepackage{amsmath} %行列が書ける
\usepackage{tikz} %\UTF{2460}などが書ける
\usepackage{cite} %複数の引用ができる

\usepackage[toc,page]{appendix}

%\graphicspath{{./pictures/}}

%\setlength{\textwidth}{\fullwidth}
\setlength{\textheight}{40\baselineskip}
\addtolength{\textheight}{\topskip}
\setlength{\voffset}{-0.55in}

\renewcommand{\baselinestretch}{1} %% 行間

%\setcounter{tocdepth}{5}  %% 目次section depth
\setcounter{secnumdepth}{5}
%\renewcommand{\bibname}{参考文献}

%%%%%%%%% accent.sty 設定 %%%%%%%%%
\makeatletter
  \def\widebar{\accentset{{\cc@style\underline{\mskip10mu}}}}
\makeatother

%%%%%%%%%  chapter 設定 %%%%%%%%%%%
%\makeatletter
%\def\@makechapterhead#1{%
%  \vspace*{1\Cvs}% 欧文は50pt 章上部の空白
%  {\parindent \z@ \raggedright \normalfont
%    \ifnum \c@secnumdepth >\m@ne
%      \if@mainmatter
%        \huge\headfont \@chapapp\thechapter\@chappos
%       \par\nobreak
%       \vskip \Cvs % 欧文は20pt
%         \hskip1zw
%      \fi
%    \fi
%    \interlinepenalty\@M
%    \centering \huge \headfont #1\par\nobreak
%    \vskip 3\Cvs}} % 欧文は40pt 章下部の空白
%\makeatother

%%%%%%%%%  chapter* 設定 %%%%%%%%%%%



%%%%%%%%%  chapter* 設定 %%%%%%%%%%%

%\makeatletter
%\def\@makeschapterhead#1{%
%  \vspace*{1\Cvs}
%  {\parindent \z@ \raggedright
%    \normalfont
%    \interlinepenalty\@M
%    \centering \huge \headfont #1\par\nobreak
%    \vskip 3\Cvs}}
%\makeatother

%%%%%%%%%  section 設定 %%%%%%%%%%%
\makeatletter
\renewcommand{\section}{%
  \@startsection{section}%
   {1}%
   {\z@}%
   {-3.5ex \@plus -1ex \@minus -.2ex}%
   {2.3ex \@plus.2ex}%
   {\normalfont\Large\bfseries}%
}%
\makeatother

%%%%%%%%%  subsection 設定 %%%%%%%%%%%
\makeatletter
\renewcommand{\subsection}{%
  \@startsection{subsection}%
   {2}%
   {\z@}%
   {-3.5ex \@plus -1ex \@minus -.2ex}%
   {2.3ex \@plus.2ex}%
   {\normalfont\large\bfseries}%
}%
\makeatother

%%%%%%%%%  subsubsection 設定 %%%%%%%%%%%
\makeatletter
\renewcommand{\subsubsection}{%
  \@startsection{subsubsection}%
   {3}%
   {\z@}%
   {-3.5ex \@plus -1ex \@minus -.2ex}%
   {2.3ex \@plus.2ex}%
   %{\normalfont\normalsize\bfseries$\blacksquare$}%
   {\normalfont\normalsize\bfseries}%
}%
\makeatother

%%%%%%%%%  paragraph 設定 %%%%%%%%%%%
\makeatletter
\renewcommand{\paragraph}{%
  \@startsection{paragraph}%
   {4}%
   {\z@}%
   {0.5\Cvs \@plus.5\Cdp \@minus.2\Cdp}
   {-1zw}
   {\normalfont\normalsize\bfseries $\blacklozenge$\ }%
  % {\normalfont\normalsize\bfseries $\Diamond$\ }%
}%
\makeatother

%%%%%%%%%  subparagraph 設定 %%%%%%%%%%%
\makeatletter
\renewcommand{\subparagraph}{%
  \@startsection{subparagraph}%
   {4}%
   {\z@}%
   {0.5\Cvs \@plus.5\Cdp \@minus.2\Cdp}
   {-1zw}
   {\normalfont\normalsize\bfseries $\Diamond$\ }%
}%
\makeatother

%%%%%%%%% caption 設定 %%%%%%%%%%%%
\makeatletter

\newcommand*\circled[1]{\tikz[baseline=(char.base)]{
            \node[shape=circle,draw,inner sep=2pt] (char) {#1};}}

\newcommand*{\rom}[1]{\expandafter\@slowromancap\romannumeral #1@}

\newcommand{\msolar}{M_\odot}

\newcommand{\anapow}{A_{y}(\theta)}
\newcommand{\depo}{D^{y}_{y}(\theta)}

\newcommand{\figcaption}[1]{\def\@captype{figure}\caption{#1}}
\newcommand{\tblcaption}[1]{\def\@captype{table}\caption{#1}}
\newcommand{\klpionn}{K_L \to \pi^0 \nu \overline{\nu}}
\newcommand{\kppipnn}{K^+ \to \pi^+ \nu \overline{\nu}}
\newcommand{\hfl}{{}_\Lambda^4\rm{H}}
\newcommand{\htl}{{}_\Lambda^3\rm{H}}
\newcommand{\hefl}{{}_\Lambda^4\rm{He}}
\newcommand{\hefil}{{}_\Lambda^5\rm{He}}
\newcommand{\lisl}{{}_\Lambda^7\rm{Li}}
\newcommand{\benl}{{}_\Lambda^9\rm{Be}}
\newcommand{\btl}{{}_\Lambda^{10}\rm{B}}
\newcommand{\bel}{{}_\Lambda^{11}\rm{B}}

\newcommand{\nfl}{{}_\Lambda^{15}\rm{N}}
\newcommand{\osl}{{}_\Lambda^{16}\rm{O}}
\newcommand{\ctl}{{}_\Lambda^{13}\rm{C}}
\newcommand{\pbtl}{{}_\Lambda^{208}\rm{Pb}}

\def\vector#1{\mbox{\boldmath$#1$}}
\newcommand{\Kpi}{(K^-,\pi^-)}
\newcommand{\piKz}{(\pi^-,K^0)}
\newcommand{\pPK}{(\pi^+,K^+)}
\newcommand{\pMK}{(\pi^-,K^+)}
\newcommand{\pPMK}{(\pi^{\pm},K^+)}

\newcommand{\eeK}{(e,e' K^+)}
\newcommand{\gK}{(\gamma + p \to \Lambda + K^+)}
\newcommand{\PiKL}{\pi^-  p \to K^0 \Lambda}
\newcommand{\multipi}{\pi^-  p \to \pi^-\pi^-\pi^+p}
\newcommand{\PiKX}{\pi^-  p \to K^0 X}
\newcommand{\PiKSM}{\pi^-  p \to K^+ \Sigma^-}
\newcommand{\pipKS}{\pi^{\pm}p \to K^+ \Sigma^{\pm}}
\newcommand{\pipKX}{\pi^{\pm}p \to K^+ X}
\newcommand{\pipLn}{\pi^- p \to \Lambda n}
\newcommand{\PiKS}{\pi^{-}p \to K^{0}\Sigma^{0}}

\newcommand{\kzdecay}{K^0 \to \pi^+ \pi^-}
\newcommand{\kzsd}{K^0_s \to \pi^+ \pi^-\ \rm{or}\ \pi^0 \pi^0}
\newcommand{\Ldecay}{\Lambda\to p\pM}
\newcommand{\scatldecay}{\Lambda'\to p\pM}




\newcommand{\triton}{{}^3\rm{H}}

\newcommand{\BB}{B_{8}B_{8}}
\newcommand{\SM}{\Sigma^{-}}
\newcommand{\SP}{\Sigma^{+}}
\newcommand{\Sz}{\Sigma^{0}}
\newcommand{\SMp}{\Sigma^{-}p}
\newcommand{\SMn}{\Sigma^{-}n}
\newcommand{\SPp}{\Sigma^{+}p}
\newcommand{\SPn}{\Sigma^{+}n}
\newcommand{\Sp}{\Sigma p}
\newcommand{\SPMp}{\Sigma^{\pm}p}
\newcommand{\SPM}{\Sigma^{\pm}}
\newcommand{\SPdecay}{\Sigma^+ \to \pi^0 p}
\newcommand{\SMdecay}{\Sigma^- \to \pi^- n}
\newcommand{\SMpLn}{\Sigma^- p \to \Lambda n}

\newcommand{\XM}{\Xi^{-}}
\newcommand{\Xz}{\Xi^{0}}

\newcommand{\pM}{\pi^{-}}
\newcommand{\pP}{\pi^{+}}
\newcommand{\pZ}{\pi^{0}}
\newcommand{\pPM}{\pi^{\pm}}
\newcommand{\KP}{K^{+}}
\newcommand{\KM}{K^{-}}
\newcommand{\Kz}{K^{0}}
\newcommand{\Lp}{\Lambda p}
\newcommand{\LpLX}{\Lambda p \to \Lambda X}

\newcommand{\LN}{\Lambda N}
\newcommand{\SN}{\Sigma N}
\newcommand{\LNtoSN}{\Lambda N\to\Sigma N}
\newcommand{\LS}{\Lambda - \Sigma}

%\newcommand{\dp}{\Delta p}
%\newcommand{\dE}{\Delta E}

\newcommand{\dcs}{d\sigma/d\Omega}
\newcommand{\fdcs}{\frac{d\sigma}{d\Omega}}
\newcommand{\dz}{\Delta z}
\newcommand{\dzkz}{\Delta z_{K^{0}}}


\newcommand{\bgct}{\beta\gamma c\tau}

\newcommand{\costp}{\cos{\theta_p}}
\newcommand{\costkz}{\cos{\theta_{K0,CM}}}
\newcommand{\costcm}{\cos{\theta}_{CM}}
\newcommand{\PL}{P_{\Lambda}}
\newcommand{\PLall}{P_{\Lambda,\ all}}
\newcommand{\PLsele}{P_{\Lambda,\ selected}}
\newcommand{\errPL}{\sigma(P_{\Lambda})}

\newcommand{\rud}{r_{ud}}
\newcommand{\errrud}{\sigma(\rud)}

\newcommand{\accPL}{\epsilon_{\PL}}
\newcommand{\erraccPL}{\sigma(\epsilon_{\PL})}

\newcommand{\PLscat}{P_{\Lambda'}}
\newcommand{\effPLw}{\epsilon_{\PL,\ w/}}
\newcommand{\erreffPLw}{\sigma(\epsilon_{\PL,\ w/})}
\newcommand{\effPLwo}{\epsilon_{\PL,\ w/o}}
\newcommand{\erreffPLwo}{\sigma(\epsilon_{\PL,\ w/o})}

\newcommand{\chisq}{\chi^{2}}

\newcommand{\centered}[1]{\begin{tabular}{l} #1 \end{tabular}}

\makeatother

\begin{document}

\graphicspath{{./pictures/chapter_summary/}}

%%%%%
\chapter{Summary} 
\label{chap-summary}
%intro.
Systematic studies of the baryon-baryon interaction between octet baryons containing $s$ quarks ($\BB$ interaction) are important to understand the nuclear forces that shape matters in the world. Studying $\BB$ interactions is equivalent to investigating the properties of flavor SU(3) multiplets. Among them, ${\bf(8)_s}$ and ${\bf(10)}$ terms are predicted by QCM to be quite repulsive due to the Pauli effect at the quark level, which is related to the origin of the repulsive core of the nuclear force. In particular, for ${\bf(10)}$ term, the J-PARC E40 experiment newly measured the high statistic $\SPp$ ($\SN (I=3/2)$ channel) differential cross-sections in the momentum range of $0.44-0.80$ GeV/$c$., which greatly restricts theoretical models. As the same experiment measured the high statistic $\SMp$ ($\SN (I=1/2)$ channel) differential cross-sections in the momentum range of $0.47-0.85$ GeV/$c$ too, it is expected that the understanding of other terms will advance.

Based on the success of the above J-PARC E40 experiment, which pioneered the technique of $YN$ scattering experiment, we newly planned a next-generation $\Lp$ scattering experiment at J-PARC (J-PARC E86). The beam $\Lambda$ generated by the $\PiKL$ reaction causes $\Lp$ scattering with the target proton in a LH$_{2}$ target. J-PARC E86 is scheduled to measure not only $\Lp$ differential cross-sections but also spin observables (analyzing power $\anapow$ and depolarization $\depo$) in the momentum range of $0.4-0.8$ GeV/$c$. $\Lp$ spin observable measurement requires a highly polarized beam $\Lambda$, and its value must also be measured. Past experiments (Ref. \cite{Baker}) have reported that $\Lambda$ derived from the $\PiKL$ reaction is almost 100\% polarized in the forward $\Kz$ scattering angle range, but it is uncertain since the statistics are small. To verify the result, we began to precisely measure $\Lambda$ polarization assuming the J-PARC E40 setup and to establish the analysis method. In this paper, as a basic research for J-PARC E86, the $\Lp$ scattering identification method and the beam $\Lambda$ polarization measurement method necessary for $\Lp$ spin observable measurements were developed. The $\PiKL$ reaction data obtained by the J-PARC E40 experiment (in $\SM$ production) was used for the analysis.

%exp.
The $\Sp$ scattering experiment (J-PARC E40) \cite{Miwa-AIP2019} \cite{Miwa-JP2020} was conducted at the K1.8 beamline in the J-PARC Hadron Experimental Facility. In the $\SM$ production, the 1.33 GeV/$c$ $\pM$ beam with $2.0\times10^{7}$/spill was generated from the 30 GeV proton beam with a spill cycle of 5.2 seconds and a beam duration of 2 seconds. When the $\PiKL$ reaction occurs inside the LH$_{2}$ target, each $\Kz$ was reconstructed from the decay $\pM$ and decay $\pP$ analyzed by a cylindrical detector cluster (CATCH) and a forward magnetic spectrometer (KURAMA), respectively. The momentum of each $\Lambda$ was tagged as the missing momentum calculated by the $\pM$ beam analyzed at the K1.8 beamline spectrometer \cite{K1.8} and the reconstructed $\Kz$. In the detection case \rom{1}, where CATCH was requested to detect two protons, $2.72\times10^{3}$ (S/N$=1.47087$) $\Lambda$s were tagged. In the detection case \rom{2}, where CATCH was requested to detect one proton and one $\pM$, $6.98\times10^{4}$ (S/N$=1.52301$) $\Lambda$s were tagged. The recoil protons derived from $\Lp$ scattering were detected by CATCH. Here, $\Lp$ scattering events were identified from two types of kinematical consistency analyses: (1) Kinematical consistency ($\Delta E$) between scattering angle and kinetic energy of recoil protons measured by CFT and BGO and (2) Kinematical consistency  ($\Delta p$) between scattering angle and momentum of scattered $\Lambda$. 

%Lp id.
The $\Lp$ scattering identification was performed in case \rom{1} with/without $\pM$ detection. To estimate the background contaminations, Geant4-based Monte Carlo simulation, which generates $pp$ elastic scattering and $\Ldecay$ decay as well as $\Lp$ scattering, was performed. We found that in the case without decay $\pM$, at least 43 counts of $\Lp$ scattering events were identified with S/N$=11.678$. In the case with decay $\pM$, we finally succeeded in identifying at least 15 counts of $\Lp$ scattering events with a high accuracy of S/N$=17.9785$. Although it is thought that there are still unknown background events that cannot be predicted by the simulation, it is considered that it is possible to identify $\Lp$ scattering events with a high S/N using the analysis method developed in this study. When we obtain high statistics in future J-PARC E86 experiments, we precisely identify $\Lp$ scattering events and derive the differential cross-section.

%PL
The beam $\Lambda$ polarization ($\PL$) measurement was performed in case \rom{2} since it requires analyzing the $\Ldecay$ decay-only events. The $\PL$ is synonymous with the slope of the emission angle distribution of decay protons in the rest frame of $\Lambda$ ($\costp$), so its value was determined by experimentally measuring the $\costp$ distribution. After correcting the measured $\costp$ distribution by the CATCH detection efficiency, we fitted it with a certain function and measured the $\PL$ values. Our measurement was performed in the $\Kz$ scattering angle range of $0.6<\costkz<1.0$ with an angular step of $d\costkz=0.05$ (finer angular interval than Ref. \cite{Baker}). As a result, we confirmed that the average polarization in the angle region of $0.6<\costkz<0.8$ is extremely high as $0.94\pm0.065$. Therefore, we concluded that by selecting this specific $\Kz$ scattering angle range in the J-PARC E86 experiment, it is possible to measure spin observables using a nearly 100\% polarized beam $\Lambda$.

%last
In the $YN$ channel, the analyzing power $\anapow$ is expected to be sensitive to the ALS force, and the depolarization $\depo$ would preserve information on the tensor force \cite{Ishikawa-2004}. The J-PARC E86 will provide essential experimental inputs on the spin-dependent terms of the $\BB$ interaction potential.



%%%%%%%%%%%%
%%%%%%%%%%%%
%\end{document}

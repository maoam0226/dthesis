%\documentclass[a4paper,12pt,oneside,openany]{jsbook}
%%\setlength{\topmargin}{10mm}
%\addtolength{\topmargin}{-1in}
%\setlength{\oddsidemargin}{27mm}
%\addtolength{\oddsidemargin}{-1in}
%\setlength{\evensidemargin}{20mm}
%\addtolength{\evensidemargin}{-1in}
%\setlength{\textwidth}{160mm}
%\setlength{\textheight}{250mm}
%\setlength{\evensidemargin}{\oddsidemargin}

%\usepackage{ascmac}

\usepackage{color}
\usepackage{textcomp}
%\usepackage[dviout]{graphicx}
%\usepackage[dvipdfm]{graphicx,color}
\usepackage{wrapfig}
\usepackage{ccaption}
\usepackage{color}
%\usepackage{jumoline} %%行にまたがって下線を引ける、ダウンロードの必要有
\usepackage{umoline}
\usepackage{fancybox}
\usepackage{pifont}
\usepackage{indentfirst} %%最初の段落も字下げしてくれる

\usepackage{amsmath,amssymb,amsfonts}
\usepackage{bm}
%\usepackage{graphicx}
\usepackage[dvipdfmx]{graphicx}
%\usepackage[dvipsnames]{xcolor}
\usepackage{subfigure}
\usepackage{verbatim}
\usepackage{makeidx}
\usepackage{accents}
%\usepackage{slashbox} %%ダウンロードの必要有

\usepackage[dvipdfmx]{hyperref} %%pdfにリンクを貼る
\usepackage{pxjahyper}

\usepackage[flushleft]{threeparttable}
\usepackage{array,booktabs,makecell}

\usepackage{geometry}
\geometry{left=30mm,right=30mm,top=50mm,bottom=5mm}

\usepackage[super]{nth} %1st, 2nd ...を出力
\usepackage{dirtytalk} %クォーテーションマーク
\usepackage{amsmath} %行列が書ける
\usepackage{tikz} %\UTF{2460}などが書ける
\usepackage{cite} %複数の引用ができる

\usepackage[toc,page]{appendix}

%\graphicspath{{./pictures/}}

%\setlength{\textwidth}{\fullwidth}
\setlength{\textheight}{40\baselineskip}
\addtolength{\textheight}{\topskip}
\setlength{\voffset}{-0.55in}

\renewcommand{\baselinestretch}{1} %% 行間

%\setcounter{tocdepth}{5}  %% 目次section depth
\setcounter{secnumdepth}{5}
%\renewcommand{\bibname}{参考文献}

%%%%%%%%% accent.sty 設定 %%%%%%%%%
\makeatletter
  \def\widebar{\accentset{{\cc@style\underline{\mskip10mu}}}}
\makeatother

%%%%%%%%%  chapter 設定 %%%%%%%%%%%
%\makeatletter
%\def\@makechapterhead#1{%
%  \vspace*{1\Cvs}% 欧文は50pt 章上部の空白
%  {\parindent \z@ \raggedright \normalfont
%    \ifnum \c@secnumdepth >\m@ne
%      \if@mainmatter
%        \huge\headfont \@chapapp\thechapter\@chappos
%       \par\nobreak
%       \vskip \Cvs % 欧文は20pt
%         \hskip1zw
%      \fi
%    \fi
%    \interlinepenalty\@M
%    \centering \huge \headfont #1\par\nobreak
%    \vskip 3\Cvs}} % 欧文は40pt 章下部の空白
%\makeatother

%%%%%%%%%  chapter* 設定 %%%%%%%%%%%



%%%%%%%%%  chapter* 設定 %%%%%%%%%%%

%\makeatletter
%\def\@makeschapterhead#1{%
%  \vspace*{1\Cvs}
%  {\parindent \z@ \raggedright
%    \normalfont
%    \interlinepenalty\@M
%    \centering \huge \headfont #1\par\nobreak
%    \vskip 3\Cvs}}
%\makeatother

%%%%%%%%%  section 設定 %%%%%%%%%%%
\makeatletter
\renewcommand{\section}{%
  \@startsection{section}%
   {1}%
   {\z@}%
   {-3.5ex \@plus -1ex \@minus -.2ex}%
   {2.3ex \@plus.2ex}%
   {\normalfont\Large\bfseries}%
}%
\makeatother

%%%%%%%%%  subsection 設定 %%%%%%%%%%%
\makeatletter
\renewcommand{\subsection}{%
  \@startsection{subsection}%
   {2}%
   {\z@}%
   {-3.5ex \@plus -1ex \@minus -.2ex}%
   {2.3ex \@plus.2ex}%
   {\normalfont\large\bfseries}%
}%
\makeatother

%%%%%%%%%  subsubsection 設定 %%%%%%%%%%%
\makeatletter
\renewcommand{\subsubsection}{%
  \@startsection{subsubsection}%
   {3}%
   {\z@}%
   {-3.5ex \@plus -1ex \@minus -.2ex}%
   {2.3ex \@plus.2ex}%
   %{\normalfont\normalsize\bfseries$\blacksquare$}%
   {\normalfont\normalsize\bfseries}%
}%
\makeatother

%%%%%%%%%  paragraph 設定 %%%%%%%%%%%
\makeatletter
\renewcommand{\paragraph}{%
  \@startsection{paragraph}%
   {4}%
   {\z@}%
   {0.5\Cvs \@plus.5\Cdp \@minus.2\Cdp}
   {-1zw}
   {\normalfont\normalsize\bfseries $\blacklozenge$\ }%
  % {\normalfont\normalsize\bfseries $\Diamond$\ }%
}%
\makeatother

%%%%%%%%%  subparagraph 設定 %%%%%%%%%%%
\makeatletter
\renewcommand{\subparagraph}{%
  \@startsection{subparagraph}%
   {4}%
   {\z@}%
   {0.5\Cvs \@plus.5\Cdp \@minus.2\Cdp}
   {-1zw}
   {\normalfont\normalsize\bfseries $\Diamond$\ }%
}%
\makeatother

%%%%%%%%% caption 設定 %%%%%%%%%%%%
\makeatletter

\newcommand*\circled[1]{\tikz[baseline=(char.base)]{
            \node[shape=circle,draw,inner sep=2pt] (char) {#1};}}

\newcommand*{\rom}[1]{\expandafter\@slowromancap\romannumeral #1@}

\newcommand{\msolar}{M_\odot}

\newcommand{\anapow}{A_{y}(\theta)}
\newcommand{\depo}{D^{y}_{y}(\theta)}

\newcommand{\figcaption}[1]{\def\@captype{figure}\caption{#1}}
\newcommand{\tblcaption}[1]{\def\@captype{table}\caption{#1}}
\newcommand{\klpionn}{K_L \to \pi^0 \nu \overline{\nu}}
\newcommand{\kppipnn}{K^+ \to \pi^+ \nu \overline{\nu}}
\newcommand{\hfl}{{}_\Lambda^4\rm{H}}
\newcommand{\htl}{{}_\Lambda^3\rm{H}}
\newcommand{\hefl}{{}_\Lambda^4\rm{He}}
\newcommand{\hefil}{{}_\Lambda^5\rm{He}}
\newcommand{\lisl}{{}_\Lambda^7\rm{Li}}
\newcommand{\benl}{{}_\Lambda^9\rm{Be}}
\newcommand{\btl}{{}_\Lambda^{10}\rm{B}}
\newcommand{\bel}{{}_\Lambda^{11}\rm{B}}

\newcommand{\nfl}{{}_\Lambda^{15}\rm{N}}
\newcommand{\osl}{{}_\Lambda^{16}\rm{O}}
\newcommand{\ctl}{{}_\Lambda^{13}\rm{C}}
\newcommand{\pbtl}{{}_\Lambda^{208}\rm{Pb}}

\def\vector#1{\mbox{\boldmath$#1$}}
\newcommand{\Kpi}{(K^-,\pi^-)}
\newcommand{\piKz}{(\pi^-,K^0)}
\newcommand{\pPK}{(\pi^+,K^+)}
\newcommand{\pMK}{(\pi^-,K^+)}
\newcommand{\pPMK}{(\pi^{\pm},K^+)}

\newcommand{\eeK}{(e,e' K^+)}
\newcommand{\gK}{(\gamma + p \to \Lambda + K^+)}
\newcommand{\PiKL}{\pi^-  p \to K^0 \Lambda}
\newcommand{\multipi}{\pi^-  p \to \pi^-\pi^-\pi^+p}
\newcommand{\PiKX}{\pi^-  p \to K^0 X}
\newcommand{\PiKSM}{\pi^-  p \to K^+ \Sigma^-}
\newcommand{\pipKS}{\pi^{\pm}p \to K^+ \Sigma^{\pm}}
\newcommand{\pipKX}{\pi^{\pm}p \to K^+ X}
\newcommand{\pipLn}{\pi^- p \to \Lambda n}
\newcommand{\PiKS}{\pi^{-}p \to K^{0}\Sigma^{0}}

\newcommand{\kzdecay}{K^0 \to \pi^+ \pi^-}
\newcommand{\kzsd}{K^0_s \to \pi^+ \pi^-\ \rm{or}\ \pi^0 \pi^0}
\newcommand{\Ldecay}{\Lambda\to p\pM}
\newcommand{\scatldecay}{\Lambda'\to p\pM}




\newcommand{\triton}{{}^3\rm{H}}

\newcommand{\BB}{B_{8}B_{8}}
\newcommand{\SM}{\Sigma^{-}}
\newcommand{\SP}{\Sigma^{+}}
\newcommand{\Sz}{\Sigma^{0}}
\newcommand{\SMp}{\Sigma^{-}p}
\newcommand{\SMn}{\Sigma^{-}n}
\newcommand{\SPp}{\Sigma^{+}p}
\newcommand{\SPn}{\Sigma^{+}n}
\newcommand{\Sp}{\Sigma p}
\newcommand{\SPMp}{\Sigma^{\pm}p}
\newcommand{\SPM}{\Sigma^{\pm}}
\newcommand{\SPdecay}{\Sigma^+ \to \pi^0 p}
\newcommand{\SMdecay}{\Sigma^- \to \pi^- n}
\newcommand{\SMpLn}{\Sigma^- p \to \Lambda n}

\newcommand{\XM}{\Xi^{-}}
\newcommand{\Xz}{\Xi^{0}}

\newcommand{\pM}{\pi^{-}}
\newcommand{\pP}{\pi^{+}}
\newcommand{\pZ}{\pi^{0}}
\newcommand{\pPM}{\pi^{\pm}}
\newcommand{\KP}{K^{+}}
\newcommand{\KM}{K^{-}}
\newcommand{\Kz}{K^{0}}
\newcommand{\Lp}{\Lambda p}
\newcommand{\LpLX}{\Lambda p \to \Lambda X}

\newcommand{\LN}{\Lambda N}
\newcommand{\SN}{\Sigma N}
\newcommand{\LNtoSN}{\Lambda N\to\Sigma N}
\newcommand{\LS}{\Lambda - \Sigma}

%\newcommand{\dp}{\Delta p}
%\newcommand{\dE}{\Delta E}

\newcommand{\dcs}{d\sigma/d\Omega}
\newcommand{\fdcs}{\frac{d\sigma}{d\Omega}}
\newcommand{\dz}{\Delta z}
\newcommand{\dzkz}{\Delta z_{K^{0}}}


\newcommand{\bgct}{\beta\gamma c\tau}

\newcommand{\costp}{\cos{\theta_p}}
\newcommand{\costkz}{\cos{\theta_{K0,CM}}}
\newcommand{\costcm}{\cos{\theta}_{CM}}
\newcommand{\PL}{P_{\Lambda}}
\newcommand{\PLall}{P_{\Lambda,\ all}}
\newcommand{\PLsele}{P_{\Lambda,\ selected}}
\newcommand{\errPL}{\sigma(P_{\Lambda})}

\newcommand{\rud}{r_{ud}}
\newcommand{\errrud}{\sigma(\rud)}

\newcommand{\accPL}{\epsilon_{\PL}}
\newcommand{\erraccPL}{\sigma(\epsilon_{\PL})}

\newcommand{\PLscat}{P_{\Lambda'}}
\newcommand{\effPLw}{\epsilon_{\PL,\ w/}}
\newcommand{\erreffPLw}{\sigma(\epsilon_{\PL,\ w/})}
\newcommand{\effPLwo}{\epsilon_{\PL,\ w/o}}
\newcommand{\erreffPLwo}{\sigma(\epsilon_{\PL,\ w/o})}

\newcommand{\chisq}{\chi^{2}}

\newcommand{\centered}[1]{\begin{tabular}{l} #1 \end{tabular}}

\makeatother

\begin{document}


%\renewcommand{\labelitemi}{・}
%\renewcommand{\labelitemii}{・}
\graphicspath{{./pictures/chapter_Lbeam/}}

\chapter{Analysis \rom{1}: \\Beam $\Lambda$ identification} 
\label{chap-Lbeam}

%%%%%
\section{Overview}
\label{sec-overview}
Before tagging the beam $\Lambda$ generated by the $\PiKL$ reaction, we first identified the beam $\pM$ and reconstructed $\Kz$. The trajectory of the beam $\pM$ was reconstructed from the \nth{3}-order transfer matrix using position information at BFT, BC3, and BC4. Since $\Kz$ itself is a neutral particle and cannot be directly measured, we detected $\pM$ and $\pP$ derived from $\kzdecay$ decay with KURAMA and CATCH, respectively, to detect $\Kz$ decay at a large solid angle and reconstruct $\Kz$.

After identifying the $\pM$ beam and reconstructing $\Kz$, $\Lambda$ derived from the $\PiKL$ reaction was tagged as the missing momentum of the $\PiKX$ reaction. The yields of tagged $\Lambda$s were calculated by fitting each missing mass spectrum in case \rom{1} and \rom{2} with two Gaussians and a \nth{3}-order polynomial. When not requesting the detected particle combinations of case \rom{1} and \rom{2}, the $\Lambda$ mass peak has the $\pm3\sigma$ interval of $1.0707 - 1.1626$ GeV/$c^{2}$, which was defined as the $\Lambda$ region. The $\Lambda$ yield within this range was $3.99\times10^{5}$. The $\Lambda$ yield was $6.976\times10^{4}$ (S/N$=1.523$) in case \rom{1}, and $2.721\times10^{3}$ (S/N$=1.471$) in case \rom{2}, respectively. %These numbers of tagged beam $\Lambda$ were sufficient for both the $\Lp$ scattering event search (case \rom{1}) and the beam $\Lambda$ polarization measurement (case \rom{2}).

%%%%%
\section{$\pM$ beam identification}
%%%%
\subsection{Identification of the time-zero segment of BH1 and BH2}
The true hit in a multi-hit event is identified by referring to the timing information of BH1 and BH2. After defining the time-zero ($t0$) segment, a segment of BH1 and BH2 with the hit timing nearest to 0, the time-of-flight distribution between each $t0$ segment of BH1 and BH2 was measured.The time-of-flight distribution between $t0$ segments of BH1 and BH2 is shown in Figure \ref{fig-btof}. Since the hit timing of BH2 is used as the start/stop timing of the DAQ system, the timing distribution of BH2 hardly spreads.1
Therefore, the t0 segment of BH2 was identified with more than 90\% efficiency by applying a timing gate of $\pm1$ ns to this distribution. On the other hand, a timing gate of $\pm5$ ns was applied to the beam TOF distribution between BH1 and BH2 to cover the time walk. In principle, the time walk needs to be corrected by ADC information. However, correction is not possible under high-intensity conditions because multiple particles collide with the same segment within the ADC gate.

Typical timing resolution for BH1 and BH2 ($\sim10.5$ m apart) is 300 ps ($\sigma$). Since the time difference between BH1 and BH2 of $\pM$ and $\KM$ of 1.33 GeV/$c$ is $\sim2.137$ ns, this resolution value can be considered sufficient to distinguish between them. The $t0$ segments of BH1 and BH2 are also used to analyze the hit position of BFT, BC3, and BC4. Here, the contamination of other particles is negligible thanks to two ESSs, four CM magnets, and two MS slits. 

\begin{figure}[!h]
  \begin{center}
    \includegraphics[width=12cm]{btof.eps}
    \caption{The time-of-flight distribution between t0 segments of BH1 and BH2. A timing gate of $\pm5$ ns was applied to the beam TOF distribution between BH1 and BH2 to cover the time walk, as represented by the red solid lines.}
    \label{fig-btof}
  \end{center}
\end{figure}

%%%%
\subsection{BFT analysis}
BFT measures the horizontal hit position $x$ at the entrance of the QQDQQ system. BFT has two overlapping layers ($xx'$), which allows adjacent hits to be clustered. The typical time resolution of BFT was 900 ps ($\sigma$) under the 20M/spill beam intensity \cite{Nana-D}. The timing gate of BFT was set to be $\pm5$ ns. Only the timing gate was applied for a single BFT hit, which was unconditionally used to reconstruct a beam momentum. The correlation between the $t0$ segment of BH1 and a hit pattern of BFT was used to reject the accidental background when several BFT hits remained after applying the timing gate.

% ?? vs. bft_clpos[i] 載せないかも。
\begin{comment}
\begin{figure}[!h]
  \begin{center}
    \includegraphics[clip,width=12cm]{bh1fbft.png}
    \caption{Correlation between the hit segment of BH1 and the hit position of BFT.}
    \label{fig: bh1bft}
  \end{center}
\end{figure}
\end{comment}

%%%%
\subsection{BC3 and BC4 analysis}
Three-dimensional trajectories at the exit of the QQDQQ system were reconstructed using hit positions on BC3 and BC4 with the least squares approximation, so-called \say{BcOut tracking} \cite{Honda-D}. The track candidates, combinations of at least 9 planes, are sorted according to chi-squared ($\chi^2$) defined by
\begin{align}
  \chi^2 &= \frac{1}{n-4} \sum^{12}_{i=1}H_{i} \left( \frac{X_{i} - f(z_{i})}{\sigma_{i}} \right)^{2} , \\
  n &= \sum^{12}_{i=1}H_{i} , \nonumber \\
   \nonumber \\
  H_i &= \left\{ \begin{gathered} 1\ ({\rm if\ i-th\ plane\ has\ a\ hit}) \\ 0\ ({\rm if\ i-th\ plane\ has\ no\ hit}) \end{gathered}\right. \nonumber \\
  X_{i} &= wp \pm dl_{i}(t), \nonumber \\
  f(z_{i}) &= x(z_i)\cos(\theta) + y(z_i)\sin(\theta), \nonumber \\
  x(z_i) &= x_0 + u_0z_i, \nonumber \\
  x(z_i) &= y_0 + v_0z_i, \nonumber
\end{align}
where $X_i$ is a local hit position in each plane represented by $wp$ (a wire position) and $dl$ (a drift length) and $f(z_i)$ is a position calculated from a tracking result at the $z$ position of each plane. Each coordinate definition is shown in Figure \ref{fig-BcOuttracking}. The local straight track is expressed by a wire tilt angle, theta, and four parameters (i.e., $x_0$, $y_0$, $u_0$, and $v_0$) which respectively denote the $x$ and $y$ position at the origin of a local tracking coordinate and its slope, $dx/dz$ and $dy/dz$. 

Since the beam particles are incident almost perpendicular to the chamber planes, in the BcOut analysis, the pair plane analysis method is adopted to solve a left/right ambiguity. This method treats a set of pair planes such as $xx'$, $uu'$, and $vv'$ as one apparent plane. To reduce the high multiplicity of each pair plane, the correlation between hits on each plane and the $t0$ segment of BH2 (the BH2 filter) was applied to limit the number of hits participating in the track search. The schematic of the BH2 filter is shown in Figure \ref{fig-bh2filter}. The reduced chi-squared distribution of BcOut tracks is shown in Figure \ref{fig-chisqrBcOut}. The threshold was set to 5.

\begin{figure}[!h]
  \begin{center}
    \includegraphics[width=12cm]{BcOuttracking.eps}
    \caption{The coordinate definition for BcOut tracking.}
    \label{fig-BcOuttracking}
  \end{center}
\end{figure}

\begin{figure}[!h]
  \begin{center}
    \includegraphics[width=12cm]{bh2filter.eps}
    \caption{Schematic of the BH2 filter. Only hits correlated with the $t0$ segment of BH2 are used to make combinations.}
    \label{fig-bh2filter}
  \end{center}
\end{figure}

%chisqrBcOut[it]
\begin{figure}[!h]
  \begin{center}
    \includegraphics[width=12cm]{chisqrBcOut_single.eps}
    \caption{Reduced chi-squared of BcOut tracking in single-track events. The upper threshold was $\chisq_{BcOut}=5$, as represented by the red solid line.}
    \label{fig-chisqrBcOut}
  \end{center}
\end{figure}

%%%%
\subsection{K1.8 beamline analysis\\(Beam momentum reconstruction)}
The beam momentum is reconstructed by a \nth{3}-order transfer matrix calculated by ORBIT using BFT and BcOut trackings. A vector represents a charged particle passing K1.8 beamline magnets as 
\begin{equation}
  \bm{X} = 
\begin{pmatrix}
 x  \\
 u  \\
 y  \\
 v  \\
 \delta   
\end{pmatrix},
\end{equation}
where $x$ is the horizontal position of the central track, $u$ is the angle in the horizontal plane of the central track, $y$ is the vertical position of the central track, $v$ is the angle in the vertical plane of the central track, and $\delta$ is the momentum deviation from the central momentum. To confirm, the final vector of a path-through particle ($X(1)$) can be written by the initial vector ($X(0)$) and a square matrix of magnetic components ($M^{(1)}$) as
\begin{equation}
  X(1) = M^{(1)} X(0)
\end{equation}
As an extension of this equation, the final vector, including \nth{2}-order and \nth{3}-order matrices ($M^{(2)}, M^{(3)}$), can be written by
\begin{equation}
  X_{i}(1) = \sum_{j} M_{ij}^{(1)} X_{i}(0) + \sum_{jk} M_{ijk}^{(2)} X_{i}(0)X_{j}(0) + \sum_{jkl} M_{ijkl}^{(3)} X_{i}(0)X_{j}(0)X_{k}(0).
\end{equation}
The components $x$, $u$, $y$, and $v$ of the final vector at the exit of the QQDQQ beamline are known from the BcOut tracking, but $\delta$ is unknown. Taking advantage of that the $x$ position of the initial vector is measured by BFT, the component $x$ of the initial vector can be defined by inverse transfer matrices ($M^{'(1)}$, $M^{'(2)}$, $M^{'(3)}$) as
\begin{equation}
  x(0) = \sum_{j} M_{1j}^{'(1)} X_{i}(1) + \sum_{jk} M_{1jk}^{'(2)} X_{i}(1)X_{j}(1) + \sum_{jkl} M_{1jkl}^{'(3)} X_{i}(1)X_{j}(1)X_{k}(1).
  \label{eq: transmtrx}
\end{equation}
Equation (\ref{eq: transmtrx}) is a cubic equation for $\delta$ and gives the beam momentum as a general solution. The $\pM$ beam momentum distribution reconstructed by the \nth{3}-order transfer matrix is shown in Figure \ref{fig-p_2nd}. The setting of the momentum slit in the beamline mainly determines the spread of momentum distribution.

%p_2nd[ntK18_Bh1]
\begin{figure}[!h]
  \begin{center}
    \includegraphics[width=12cm]{p_2nd.eps}
    \caption{The $\pM$ beam momentum distribution reconstructed by the \nth{3}-order transfer matrix in K1.8 beamline.}
    \label{fig-p_2nd}
  \end{center}
\end{figure}


%%%%%
\clearpage
\section{$\Kz$ identification}
\label{sec-kzreco}
To reconstruct $\Kz$, $\pM$ and $\pP$ derived from the $\kzdecay$ decay were detected by the KURAMA spectrometer and CATCH system, respectively. Typically, the energy of decay $\pM$ is large and often penetrates the BGO calorimeter. At this time, the CATCH system can only measure the direction. Therefore, we recalculated the momentum of decay $\pM$ ($p_{\pM}$) so that the invariant masses of decay $\pP$ and decay $\pM$ are equal to the mass of $\Kz$. Here, an opening angle between the two $\pi$s ($\theta_{\pi\pi}$) was used to solve the kinematics of $\kzdecay$ decay.

%%%%
\subsection{Outgoing $\pP$ analysis}
The KURAMA spectrometer analyzes the outgoing particles produced by the $\pM p$ reaction that occurred in the LH$_2$ target. To determine the momentum of the outgoing particle, the trajectory in the KURAMA magnet connecting the local straight-line tracks upstream and downstream of the KURAMA spectrometer, called \say{SDCin track} and \say{SDCout track}, was obtained by numerically calculating the equation of motion in the magnetic field using the \nth{4}-order Runge-Kutta method by ANSYS \cite{Runge}. This step is called \say{KURAMA tracking.} KURAMA tracking starts from the TOF position to the target position using five parameters defined as follows:
\begin{itemize}
  \item $x$: horizontal position on the TOF plane,
  \item $y$: vertical position on the TOF plane,
  \item $u$: slope in the horizontal plane ($dx/dz$) at TOF,
  \item $v$: slope in the vertical plane ($dy/dz$) at TOF,
  \item $q$: charge divided by momentum.
\end{itemize}

The optimal momentum and trajectory of the ejected particle were determined from five parameters ($x$, $y$, $u$, $v$, $p$) representing the momentum vector and position in the TOF that minimized the reduced chi-square:
\begin{align}
  \chi^{2}_{KURAMA}/ndf &= \frac{1}{N_{hit}-5}\sum_{i}H_{i}(\frac{X^{hit}_{i} - X^{track}_{i}}{\sigma{i}}), \\
    \nonumber \\
  H_i &= \left\{ 
    \begin{gathered}
  	1\ ({\rm if\ i-th\ layer\ has\ a\ hit\ including\ the\ SdcIn/SdcOut\ track}) \\ 
	0\ ({\rm otherwise}) 
    \end{gathered}
    \right. \nonumber \\
  N_{hit} &= \sum_{i}H_{i}, \nonumber 
\end{align}
where $X^{hit}_{i}$ and $X^{track}_{i}$ respectively represent a local position obtained from each detector hit and KURAMA tracking, and $\sigma_{i}$ is a position resolution of each plane. For the multi-hit events, all combinations between the SdcIn tracks and the SdcOut tracks are examined and sorted by $\chi^{2}$ values. The convergence of the iteration was judged by the following criterion: $(\chi^{2}_{l+1} - \chi^{2}_{l})/\chi^{2}_{l} < 10^{-3}$, where $\chi^{2}_{l}$ represents the result of the $l$-th iteration. The reduced chi-squared distribution of KURAMA tracking is shown in Figure \ref{fig-chisqrKurama}. Good tracks with a reduced $\chisq$ value of less than 50 are accepted.

%chisqrKurama[i] ???
\begin{figure}[!h]
  \begin{center}
    \includegraphics[width=12cm]{chisqrKurama_single.eps}
    \caption{Reduced chi-squared distribution of KURAMA tracking in single-track events. Good tracks with a reduced $\chisq$ value of less than 50 are accepted, as represented by the red solid line.}
    \label{fig-chisqrKurama}
  \end{center}
\end{figure}

The mass of the outgoing particle was calculated using the KURAMA tracking result and the measured time-of-flight from the target center to the TOF counter. So, the mass squared ($m^{2}$) of outgoing particles can be calculated as
\begin{align}
  m^2 &= \left( \frac{p}{c} \right)^{2} \left( \frac{1}{\beta^{2}} - 1\right), \\
  \beta &= \frac{L}{ct},
\end{align}
where 
$p$ is the measured momentum of the scattered particle, $c$ is the speed of light, $L$ is the track length from the target center to the TOF counter, and $t$ is the time-of-flight. Figure \ref{fig-pKurama_m2} shows the correlation between mass squared $m^{2}$ and the momentum of the scattered particles measured by the KURAMA spectrometer. $\pP$ from the $\kzdecay$ decay was identified by the $m^{2}$ gate of $m^{2}<0.15$ (GeV/$c^{2}$)$^{2}$ with a positive charge.

KURAMA tracking allows charge discrimination of detected particles, and when the KURAMA spectrometer requires positively charged scattering particles, the high-intensity $\pM$ beam is no longer the dominant background. However, by-products from other reactions must be removed. One of the main backgrounds is the multiple-$\pi$ production (i.e. the $\multipi$ reaction) since the combination of final state particles is the same as the $\Lp$ scattering. To reject such background events and select only the $\PiKL$ reaction, some cuts were applied. The details were explained in Sec. \ref{sec-Lbeam-EVselect}. 

%m2[i] vs. pKurama[i]
\begin{figure}[!h]
  \begin{center}
    \includegraphics[width=12cm]{pKurama_m2.eps}
    \caption{Correlation between mass squared $m^{2}$ and the momentum of the scattered particles measured by the KURAMA spectrometer. $\pP$ from the $\kzdecay$ decay was identified by the $m^{2}$ gate of $m^{2}<0.15$ (GeV/$c^{2}$)$^{2}$ (red solid line) with a positive charge.}
    \label{fig-pKurama_m2}
  \end{center}
\end{figure}



%2023/10/16 15:00
%%%%
\subsection{Outgoing $\pM$ analysis}
After energy calibrations \cite{Miwa-SMp} were applied, particle identification in CATCH was performed using the correlation between the energy loss in CFT $\Delta E_{CFT}$ and the total energy deposit in BGO calorimeter $E_{tot\ BGO}$ of each track ($\Delta E-E$ method). $\Delta E_{CFT}$ is the energy loss per CFT layer, calculated by dividing the sum of the energy losses in each layer by the number of layers used for tracking. $\Delta E$ is proportional to the length of the fiber through which the scattering particles pass, which causes the energy loss in the CFT $\phi$ layer stretched parallel to the $\pM$ beam axis to depend greatly on the incident angle $\theta$ of the particles. So, the energy loss in the CFT $\phi$ layer was corrected by multiplying by $\sin\theta$. 

%!!
\subsubsection{Energy calibration of CATCH}
In Ref. \cite{Miwa-SMp}, in which the differential cross-section of $\SMp$ scattering was measured, the energy calibration of CATCH was performed by comparing the simulated and real $pp$ elastic scattering data. Here, the correlations between proton scattering angle and energy deposit in CFT and BGO in the momentum range of $450-850$ MeV/$c$ were used. In this paper, this calibration data was applied to the CATCH analysis part since the $\PiKL$ reaction data was simultaneously accumulated during the $\SM$ production.

\vspace{10pt}
%%%
\paragraph{Energy calibration of BGO calorimeter}
\label{sec-EcalibBGO}
The energy calibration of the BGO calorimeter was performed using the $pp$ elastic scattering data with a 600 MeV/$c$ proton beam by referring to the correlation between the scattering angle and kinetic energy of the proton. The kinetic energy can be kinematically calculated by the scattering angle. The correlation between the pulse height and energy deposit in BGO was obtained by comparing the measured pulse height with the simulated energy deposit for the same scattering angle. This obtained correlation was fitted with a phenomenological relation between the photon yield and the energy deposit. %as
%\begin{equation}
  %h_{BGO} = aE - b\log{\left( \frac{E+b}{b} \right)},
%\end{equation}
%where $h_{BGO}$ and  $E$ are pulse height and energy deposit at the BGO calorimeter, and $a$ and $b$ are parameters. 
Finally, the correlation between the scattering angle and kinetic energy measured as a sum of the energy deposits in the CFT and BGO was obtained, as shown in Figure \ref{fig-EcalibBGO} \cite{Miwa-SMp}. The energy dependence of the energy resolution of the BGO calorimeter was estimated from the width of the $pp$ elastic scattering locus for different beam momenta. The energy resolution is then expressed as
\begin{equation}
  \sigma (\rm{MeV}) = 0.5 \sqrt{\frac{E(\rm{MeV})}{80}} + 4.0,
\end{equation}
where $E$ represents the kinetic energy of a proton. This energy calibration was performed for all BGO segments.

\begin{figure}[!h]
  \begin{center}
    \includegraphics[width=12cm]{EcalibBGO.eps}
    \caption{Correlation between the scattering angle and kinetic energy measured as a sum of the energy deposits in the CFT and BGO for the $pp$ elastic scattering data with a 600 MeV/$c$ proton beam \cite{Miwa-SMp}. The locus corresponds to the elastic scattering events.}
    \label{fig-EcalibBGO}
  \end{center}
\end{figure}


\vspace{10pt}
%%%
\paragraph{Energy calibration of CFT}
In the J-PARC E40 experiment, the pulse height information acquired by each MPPC attached to the scintillation fiber of the CFT was recorded by the peak-hold ADC of the VME-EASIROC module. These ADC values were normalized for each layer using $pp$ elastic scattering data. The energy calibration of CFT was performed the same way as for the BGO calorimeter explained in Sec. \ref{sec-EcalibBGO}. The correlation between the normalized CFT ADC value and the energy deposit of the CFT fiber was obtained by comparing the ADC value with the simulated energy deposit of the same BGO energy loss. The obtained correlation was fitted with a phenomenological relation between the photon yield and the energy loss.
%\begin{equation}
  %h_{CFT} = a\left\{ 1-\exp{\left( \frac{-b\times dE}{a} \right)} \right\},
%\end{equation}
%where $a$ and $b$ are parameters corresponding to the effective number of pixels and the number of photons per energy deposit in the MPPC. 
This parameterization takes into account the saturation effect of MPPC. 

The angular resolution of the CFT was estimated as 1.5\% ($\sigma$) from the width of the opening angle of two protons in the $pp$ elastic scattering, which was constant at approximately $90^{\circ}$ kinematically. The vertex resolution of the CFT tracking was studied by reconstructing the image of the target container from the vertex of the two-proton tracks in CATCH for the $\pM$ beam run. The $z$ vertex resolution was estimated as 1.8 mm ($\sigma$), and the $x$ and $y$ vertex resolutions were estimated as 1.9 mm ($\sigma$) for both directions \cite{Miwa-SMp}. 


%%%%
\subsubsection{Evaluation of CATCH detection efficiency}
\label{subsubsec-evaCATCHeff}

CATCH detection efficiency includes the detector acceptance, the tracking efficiency of CFT, and the energy measurement efficiency of BGO. They depend on the angle $\theta_{lab}$, the kinetic energy $E$, and the $z-$vertex position $z$. The efficiency was evaluated based on $pp$ elastic scattering data at seven beam momentums in the $0.45-0.85$ GeV/$c$ range compared with the simulation. 

To obtain a realistic detection efficiency, we used $pp$ elastic scattering data in which two protons are emitted. We detect a proton and identify $pp$ elastic scattering events by solving the kinematics between scattering angle and kinetic energy. By calculating the missing momentum of the $pp\rightarrow pX$ reaction, the angle and momentum of the other proton can be predicted.

\vspace{10pt}
%CFT tracking efficiency
\paragraph{CFT tracking efficiency}
The CFT tracking efficiency was obtained by checking whether the predicted track could be detected, referring to the momentum dependence of the recoil proton, as shown in Figure \ref{fig-CFTtrackeff} \cite{Miwa-SMp}. The black square represents the simulation results, and the blue circle represents the results in $pp$ elastic scattering data. The CFT tracking efficiency was modeled as the Fermi function with three parameters: the maximum efficiency $P_{max}$, the momentum with half efficiency $P(1/2)$s, and blurriness $\mu$. These parameters were determined by fitting the $pp$ elastic scattering data in Figure \ref{fig-CFTtrackeff}. This parametrization of the CFT tracking efficiency was performed for each scattering angle $\theta_{lab}$. The estimated CFT tracking efficiency map as a function of the scattering angle and momentum of the proton is shown in Figure \ref{fig-CFTtrackmap} \cite{Miwa-SMp}. For more details, refer to Ref. \cite{Miwa-SMp}.

\begin{figure}[!h]
  \begin{center}
    \includegraphics[width=12cm]{CFTtrackeff.eps}
    \caption{Momentum dependence of the CFT tracking for protons from the $pp$ elastic scattering data and the simulation data for a scattering angle of $\theta_{lab}=51^{\circ}$ \cite{Miwa-SMp}.}
    \label{fig-CFTtrackeff}
  \end{center}
\end{figure}

\begin{figure}[!h]
  \begin{center}
    \includegraphics[width=12cm]{CFTtrackmap.eps}
    \caption{Estimated CFT tracking efficiency map as a function of the scattering angle and momentum of the proton \cite{Miwa-SMp}.}
    \label{fig-CFTtrackmap}
  \end{center}
\end{figure}


\vspace{10pt}
%BGO energy measurement efficiency 
\paragraph{BGO energy measurement efficiency }
The efficiency was estimated by checking whether or not the measured energy by BGO was consistent with the predicted energy from the pp scattering kinematics, using the $pp$ elastic scattering data the same way as in the study of CFT tracking efficiency. Figure \ref{fig-BGOeff}  shows the momentum dependence of the
BGO efficiency for protons emitted at $\theta_{lab}=41^{\circ}$ where the crossed line represents the simulation result and the red circle represents the result of $pp$ elastic scattering data. Regarding BGO energy measurement efficiency, consistency between data and simulations was confirmed in all angular regions. Therefore, the simulated efficiency was used as the efficiency map, as shown in Figure \ref{fig-BGOeffmap}. For more details, refer to Ref. \cite{Miwa-SMp}.

\begin{figure}[!h]
  \begin{center}
    \includegraphics[width=12cm]{BGOeff.eps}
    \caption{Momentum dependence of the BGO energy measurement for protons from the $pp$ elastic scattering data and the simulation data for a scattering angle of $\theta_{lab}=41^{\circ}$ \cite{Miwa-SMp}.}
    \label{fig-BGOeff}
  \end{center}
\end{figure}

\begin{figure}[!h]
  \begin{center}
    \includegraphics[width=12cm]{BGOeffmap.eps}
    \caption{Estimated BGO efficiency map as a function of the scattering angle and momentum of the proton \cite{Miwa-SMp}.}
    \label{fig-BGOeffmap}
  \end{center}
\end{figure}


%%%%%%%%%%%%%%%%%%%%%
\subsubsection{Particle identification}
Figure \ref{fig-dE_E} shows the $\Delta E-E$ plot of $\Lambda$ production events obtained after applying the CATCH calibration. The track defined by the area surrounded by the solid red line corresponds to $\pi$. The other track distributed above it mainly corresponds to protons. Because it has a mass heavier than $\pi$, it tends to stop in the BGO calorimeter, causing the $dE/dx$ distribution to rise to the left. Most of the $\pi$ penetrated the BGO calorimeter with only a partial loss of kinetic energy. Therefore, the only information available in $\pi$s is track information.

\begin{figure}[!h]
  \begin{center}
    \includegraphics[clip,width=12cm]{dE_E.eps}
    \caption{Correlation of energy loss in CFT of scattering particles and total energy measured with BGO calorimeter. We required CATCH to detect $\pM$ when reconstructing $\Kz$, as represented by the red solid lines.}
    \label{fig-dE_E}
  \end{center}
\end{figure}

%%%
\subsubsection{Determination of the outgoing $\pM$ momentum}
As mentioned, the CATCH system can not accurately measure the energy of scattered $\pi$s since most penetrate the BGO calorimeter. As a result, measuring their momenta is difficult. To compensate for the momentum information of the scattered $\pi$, we recalculated that value by solving the kinematics of the $\kzdecay$ decay. Here, we assumed that the two $\pi$s detected by KURAMA and CATCH were derived from the $\kzdecay$ decay. Then, using the opening angle between two $\pi$s ($\theta_{\pi\pi}$) and the $\pP$ momentum ($p_{\pP}$), the $\pM$ momentum ($p_{\pM}$) was recalculated so that the invariant mass of $\pP$ (detected by KURAMA) and $\pM$ (detected by CATCH) became equal to the $\Kz$ mass. The details of this calculation are as follows:

\begin{enumerate}
  \item The $\pP$ total energy $E_{\pP}$ was calculated by 
\begin{equation}
  E_{\pP} = \sqrt{m_{\pP}^{2} + p_{\pP}^{2}}.
\end{equation}

  \item According to the law of conservation of energy and the law of conservation of momentum assuming the $\kzdecay$ decay kinematics, two positive and negative solutions are obtained for the $\pM$ momentum using the solution method. This time, we adopted the positive solution as the absolute value of the momentum as 
\begin{align}
  p_{\pM} &= \frac{Ap_{\pP} + \cos{\theta_{\pi\pi}} + B}{E_{\pP}^{2} - p_{\pP}^{2} \cos^{2}{\theta_{\pi\pi}}} \\
  A &= \frac{m_{\Kz}^{2} - (m_{\pP}^{2} + m_{\pM}^{2})}{2} \\
  B &= \sqrt{(Ap_{\pP}\cos{\theta_{\pi\pi}})^{2} - (E_{\pP}^{2} - p_{\pP}^{2} \cos^{2}{\theta_{\pi\pi}}) (E_{\pP}^{2}m_{\pM}^{2} - A^{2})}
\end{align}

\end{enumerate}

%%%%
\clearpage
\subsection{$\Kz$ reconstruction assuming the $\kzdecay$ decay}
Using the $\pM$ momentum determined in the previous subsection and the direction of the $\pM$ track measured by CFT, the $\pM$ momentum vector $\bm{p_{\pM}}$ can be determined. Plus, the KURAMA spectrometer already measured the $\pP$ momentum vector $\bm{p_{\pP}}$. Therefore, the $\Kz$ momentum vector $\bm{p_{\Kz}}$ can be obtained as
\begin{equation}
  \bm{p_{\Kz}} = \bm{p_{\pP}} + \bm{p_{\pM}}.
\end{equation}
The beam $\Lambda$ produced by the $\PiKL$ reaction was identified by calculating the missing mass of the $\PiKX$ reaction using this reconstructed $\Kz$ momentum vector. The beam $\Lambda$ identification details are explained in Sec. \ref{sec-beamLID}.

%%%%
\subsection{Event selection for $\Kz$ production}
\label{sec-Lbeam-EVselect}
In the previous section, we reconstructed the \say{temporary $\Kz$} using the two $\pi$s detected by KURAMA and CATCH, respectively. To ensure that this particle was indeed a $\Kz$ produced by the $\PiKL$ reaction, two cuts were used for event screening: (1) Opening angle between two $\pi$s (2) Distance of closest approach between $\pM$ beam and $\Kz$ tracks. 

In this paper, we analyzed two detection cases of CATCH: case \rom{1} for beam $\Lambda$ polarization measurement and case \rom{2} for $\Lp$ scattering identification independently. Detection case \rom{1} requires that CATCH detects $pM$ of $\kzdecay$ decay and decay products of $\Ldecay$ decay. Detection case \rom{2} requires that CATCH detects $pM$ of $\kzdecay$ decay, recoil proton, and decay products of $\scatldecay$ decay. Therefore, the event screening to identify $\Kz$ production above was applied in each detection case. 




%%%
\subsubsection{Opening angle between two $\pi$s}
%K0選択カットについては、ケース毎にも記載するからここに説明書くべきか?要検討。
In case \rom{1} and case \rom{2}, the opening angle of two $\pi$s from the $\Kz$ decay should be under $90^{\circ}$ according to the Monte Carlo simulation as shown in Figure \ref{fig-thetapipi_1p1pi_sim} (case \rom{1}) and Figure \ref{fig-thetapipi_2p_sim} (case \rom{2}). If the main background reaction, multiple $\pi$ production, occurs, it would distribute wider. To remove such contamination, the cut was chosen at $90^{\circ}$ as represented by the red solid lines in Figure \ref{fig-thetapipi_1p1pi} (case \rom{1}) and Figure \ref{fig-thetapipi_2p} (case \rom{2}). 

%theta(pipi) sim
\begin{figure}[!h]
  \begin{center}
    \includegraphics[clip,width=12cm]{thetapipi_1p1pi_sim.eps}
    \caption{Opening angle between $\pM$ and $\pP$ of the $\kzdecay$ decay in case \rom{1}, taken in simulation data.}
    \label{fig-thetapipi_1p1pi_sim}
  \end{center}
\end{figure}

\begin{figure}[!h]
  \begin{center}
    \includegraphics[clip,width=12cm]{thetapipi_2p_sim.eps}
    \caption{Opening angle between $\pM$ and $\pP$ of the $\kzdecay$ decay in case \rom{2}, taken in simulation data.}
    \label{fig-thetapipi_2p_sim}
  \end{center}
\end{figure}

%theta(pipi) e40
\begin{figure}[!h]
  \begin{center}
    \includegraphics[clip,width=12cm]{thetapipi_1p1pi.eps}
    \caption{Opening angle between $\pM$ and $\pP$ of the $\kzdecay$ decay in case \rom{1}, taken in E40 data.}
    \label{fig-thetapipi_1p1pi}
  \end{center}
\end{figure}

\begin{figure}[!h]
  \begin{center}
    \includegraphics[clip,width=12cm]{thetapipi_2p.eps}
    \caption{Opening angle between $\pM$ and $\pP$ of the $\kzdecay$ decay in case \rom{2}, taken in E40 data.}
    \label{fig-thetapipi_2p}
  \end{center}
\end{figure}


%minipageはなしかも
\begin{comment}

\begin{figure}[!h]
  \begin{minipage}[t]{0.48\columnwidth}
    \centering
    \includegraphics[clip,width=\columnwidth]{thetapipi_sim1.png}
    \caption{Simulation of the opening angle between $\pM$ and $\pP$ derived from the $\kzdecay$ decay in case \rom{1}.}
    \label{fig-thetapipi_sim1}
  \end{minipage}
  \hspace{0.04\columnwidth} % ここで隙間作成
  \begin{minipage}[t]{0.48\columnwidth}
    \centering
    \includegraphics[clip, width=\columnwidth]{thetapipi_sim2.png}
    \caption{Simulation of the opening angle between $\pM$ and $\pP$ derived from the $\kzdecay$ decay in case \rom{2}.}
    \label{fig-thetapipi_sim2}
  \end{minipage}
\end{figure}


%theta(pipi) e40
\begin{figure}[!h]
  \begin{minipage}[t]{0.48\columnwidth}
    \centering
    \includegraphics[clip,width=\columnwidth]{thetapipi_1.png}
    \caption{Opening angle between $\pM$ and $\pP$ derived from the $\kzdecay$ decay in case \rom{1}, taken in the E40 data.}
    \label{fig-thetapipi_1}
  \end{minipage}
  \hspace{0.04\columnwidth} % ここで隙間作成
  \begin{minipage}[t]{0.48\columnwidth}
    \centering
    \includegraphics[clip, width=\columnwidth]{thetapipi_2.png}
    \caption{Opening angle between $\pM$ and $\pP$ derived from the $\kzdecay$ decay in case \rom{2}, taken in the E40 data.}
    \label{fig-thetapipi_2}
  \end{minipage}
\end{figure}

\end{comment}


%%%
\subsubsection{Distance of closest approach of $\pM$ beam and $\Kz$}
The $\Kz$ reconstruction method sometimes regards the wrong combination of $\pM$ and $\pP$ as the decay products of $\kzdecay$ decay. For case \rom{1}, CATCH could mistakenly select the $\pM$ of $\Ldecay$ decay as the one of $\kzdecay$ decay. That results in the wrong $\Kz$ being reconstructed and its track would be placed apart from the $\pM$ beam track. For case \rom{1}, CATCH could mistakenly select the $\pM$ of $\scatldecay$ decay as the one of $\kzdecay$ decay. 
To ensure the $\Kz$ reconstruction was correctly performed, we checked the closest distance between the $\pM$ beam and $\Kz$ tracks. The simulation indicates that it would distribute under 20 mm in case \rom{1} and case \rom{2}, as shown in Figure \ref{fig-cdistk0_1p1pi_sim} and Figure \ref{fig-cdistk0_2p_sim}. Therefore, the cut was chosen at the same value for E40 data analysis, as shown in Figure \ref{fig-cdistk0_1p1pi} and Figure \ref{fig-cdistk0_2p}. 

%%%%%%%%
% 2023/11/17 18:12
%%%%%%%%


%cdistk0 sim
\begin{figure}[!h]
  \begin{center}
    \includegraphics[clip,width=12cm]{cdistk0_1p1pi_sim.eps}
    \caption{Closest distance between $\pM$ beam and $K^{0}$ in case \rom{1}, taken in simulation data.}
    \label{fig-cdistk0_1p1pi_sim}
  \end{center}
\end{figure}

\begin{figure}[!h]
  \begin{center}
    \includegraphics[clip,width=12cm]{cdistk0_2p_sim.eps}
    \caption{Closest distance between $\pM$ beam and $K^{0}$ in case \rom{2}, taken in simulation data.}
    \label{fig-cdistk0_2p_sim}
  \end{center}
\end{figure}

%cdistk0 e40
\begin{figure}[!h]
  \begin{center}
    \includegraphics[clip,width=12cm]{cdistk0_1p1pi.eps}
    \caption{Closest distance between $\pM$ beam and $K^{0}$ in case \rom{1}, taken in E40 data.}
    \label{fig-cdistk0_1p1pi}
  \end{center}
\end{figure}

\begin{figure}[!h]
  \begin{center}
    \includegraphics[clip,width=12cm]{cdistk0_2p.eps}
    \caption{Closest distance between $\pM$ beam and $K^{0}$ in case \rom{2}, taken in E40 data.}
    \label{fig-cdistk0_2p}
  \end{center}
\end{figure}

\begin{comment}

%DOCA (pi & K0) sim
\begin{figure}[!h]
  \begin{minipage}[t]{0.48\columnwidth}
    \centering
    \includegraphics[clip,width=\columnwidth]{cdistk0_sim1.png}
    \caption{Simulation of the distance of closest approach between the $\pM$ beam and reconstructed $\Kz$ tracks in case \rom{1}.}
    \label{fig-cdistk0_sim1}
  \end{minipage}
  \hspace{0.04\columnwidth} % ここで隙間作成
  \begin{minipage}[t]{0.48\columnwidth}
    \centering
    \includegraphics[clip, width=\columnwidth]{cdistk0_sim2.png}
    \caption{Simulation of the distance of closest approach between the $\pM$ beam and reconstructed $\Kz$ tracks in case \rom{2}.}
    \label{fig-cdistk0_sim2}
  \end{minipage}
\end{figure}

%DOCA (pi & K0) e40
\begin{figure}[!h]
  \begin{minipage}[t]{0.48\columnwidth}
    \centering
    \includegraphics[clip,width=\columnwidth]{cdistk0_1.png}
    \caption{Distance of closest approach between the $\pM$ beam and reconstructed $\Kz$ tracks in case \rom{1}, taken in the E40 data.}
    \label{fig-cdistk0_1}
  \end{minipage}
  \hspace{0.04\columnwidth} % ここで隙間作成
  \begin{minipage}[t]{0.48\columnwidth}
    \centering
    \includegraphics[clip, width=\columnwidth]{cdistk0_2.png}
    \caption{Distance of closest approach between the $\pM$ beam and reconstructed $\Kz$ tracks in case \rom{2}, taken in the E40 data.}
    \label{fig-cdistk0_2}
  \end{minipage}
\end{figure}

\end{comment}

%%%
%K0 path cutは入れないことにしたのでコメントアウト
\begin{comment}
\subsubsection{Path length of $\Kz$ projected onto the $z-$axis}
\label{sec-dzk0}
In the $\Kz$ event, ideally, the difference between the vertex position of the $\pM$ beam and $\Kz$ tracks, and the vertex position of the two outgoing $\pi$s' tracks should become larger depending on the path length of $\Kz$. In contrast, it must be zero in the non-strangeness multiple $\pi$ events. In other words, this vertex position difference corresponds to the $\Kz$ path length inside the LH$_2$ target. The $\Kz$ path length projected onto the $z-$axis, $\Delta z_{\Kz}$, can be calculated as 
\begin{align}
  \Delta z_{\Kz} &= z_{decay} - z_{prod}, \\
  		        &= z_{\pP,\ \pM} - z_{\pM\ beam,\ \Kz},
  \label{eq-dzk0}
\end{align}
where $z_{\pP,\ \pM}$ ($z_{decay}$) is the vertex position of the outgoing $\pM$ and $\pP$ tracks projected onto the $z-$axis, and $z_{\pM\ beam,\ \Kz}$ ($z_{prod}$) is the vertex position of the $\pM$ beam and reconstructed $\Kz$ tracks projected onto the $z-$axis. The Monte Carlo simulation indicates that it would distribute in the $-50 - 200$ mm range as shown in Figure \ref{fig-K0path_sim}, so the cut was chosen at the same region for E40 data as shown in Figure \ref{fig-K0path}. 
\end{comment}


%%%%%
\clearpage
\section{Beam $\Lambda$ identification}
\label{sec-beamLID}
%%%%
\subsection{Missing mass of the $\PiKX$ reaction}
The beam $\Lambda$ was identified by the missing mass method using the four-momentum vectors of the incident and outgoing particles analyzed by spectrometers based on the kinematics of the $\PiKL$ reaction, as shown in Equation (\ref{eq-mmom_lam}). Consequently, the missing mass of the $\PiKX$ reaction can be calculated by Equation \ref{eq-mm_lam} using the information of the reconstructed $\pM$ beam and outgoing $\Kz$. In the Monte Carlo simulations of case \rom{1} and case \rom{2}, the missing mass spectrums show clear $\Lambda$ mass peaks, as shown in Figure \ref{fig-mm0_1p1pi_sim} and Figure \ref{fig-mm0_2p_sim}.

In the E40 data analysis, the missing mass without requesting the number of detected particle combinations of case \rom{1} or case \rom{2} (case \say{all}) can be obtained as shown in Figure \ref{fig-mm_lam}. The missing masses in case \rom{1} and case \rom{2} are also shown in Figure \ref{fig-mm0_1p1pi} and Figure \ref{fig-mm0_2p}. The peak corresponding to the $\Lambda$ mass ($m_{\Lambda} = 1.116$ GeV/$c^{2}$) can be seen on the left, and the small peak corresponding to the $\Sigma^{0}$ mass ($m_{\Sigma^{0}} = 1.192$ GeV/$c^{2}$), which was produced by the $\PiKS$ reaction, can be seen on the right. 

To know the yields of tagged $\Lambda$s, each histogram was fitted with a function that combines two Gaussians and a \nth{3}-order polynomial, as represented by the red solid lines. The $\pm3\sigma$ interval of the first Gaussian taken in the spectrum of $m_{X,\ tot}$ was $1.0707 - 1.1626$ GeV/$c^{2}$ with the $\Lambda$ yield of $3.989\times10^{5}$ (S/N$=0.232$). This tagged $\Lambda$'s $\pm3\sigma$ interval of $m_{X,\ tot}$ was used as the integration range for the $\Lambda$ tagging in the missing masses of case \rom{1} and case \rom{2}. As a result, we found that the yield of tagged $\Lambda$ was $2.721\times10^{3}$ (S/N$=1.471$) in case \rom{1} and $6.976\times10^{4}$ (S/N$=1.523$) in case \rom{2}. The numerical values of parameters involved in the fitting are summarized in Table \ref{tab-MMfit}. The $\Lp$ scattering event search in case \rom{1} and the beam $\Lambda$ polarization measurement in case \rom{2} were performed only for the events in the missing mass range of $1.0 - 1.17$ GeV/$c^{2}$.


%mm0 sim
\begin{figure}[!h]
  \begin{center}
    \includegraphics[clip,width=12cm]{mm0_1p1pi_sim.eps}
    \caption{Missing mass of the $\PiKX$ reaction in case \rom{1}, taken in simulation data.}
    \label{fig-mm0_1p1pi_sim}
  \end{center}
\end{figure}

\begin{figure}[!h]
  \begin{center}
    \includegraphics[clip,width=12cm]{mm0_2p_sim.eps}
    \caption{Missing mass of the $\PiKX$ reaction in case \rom{2}, taken in simulation data.}
    \label{fig-mm0_2p_sim}
  \end{center}
\end{figure}

%mm0 e40
%ALL
\begin{figure}[!h]
  \begin{center}
    \includegraphics[clip,width=12cm]{mm0_fit_all.eps}
    \caption{Missing mass of the $\PiKX$ reaction without requesting the number of detected particle combinations of case \rom{1} or case \rom{2}. The red solid line represents the fitting function of two Gaussians and a \nth{3}-order polynomial, and the $\pm3\sigma$ interval of the first Gaussian was defined as the $\Lambda$ peak region.}
    \label{fig-mm_lam}
  \end{center}
\end{figure}

\begin{figure}[!h]
  \begin{center}
    \includegraphics[clip,width=12cm]{mm0_fit_1p1pi.eps}
    \caption{Missing mass of the $\PiKX$ reaction in case \rom{1}, taken in E40 data. The red solid lines represent the $\Lambda$ region defined by the fitting of Figure \ref{fig-mm_lam}.}
    \label{fig-mm0_1p1pi}
  \end{center}
\end{figure}

\begin{figure}[!h]
  \begin{center}
    \includegraphics[clip,width=12cm]{mm0_fit_2p.eps}
    \caption{Missing mass of the $\PiKX$ reaction in case \rom{2}, taken in E40 data. The red solid lines represent the $\Lambda$ region defined by the fitting of Figure \ref{fig-mm_lam}.}
    \label{fig-mm0_2p}
  \end{center}
\end{figure}


\begin{comment}

%sim
\begin{figure}[!h]
  \begin{minipage}[t]{0.48\columnwidth}
    \centering
    \includegraphics[clip,width=\columnwidth]{mm_lam_sim1.png}
    \caption{Simulation of missing mass of the $\PiKL$ reaction in case \rom{1}, taken in the E40 data.}
    \label{fig-mm_lam_sim1}
  \end{minipage}
  \hspace{0.04\columnwidth} % ここで隙間作成
  \begin{minipage}[t]{0.48\columnwidth}
    \centering
    \includegraphics[clip, width=\columnwidth]{mm_lam_sim2.png}
    \caption{Simulation of missing mass of the $\PiKL$ reaction in case \rom{2}, taken in the E40 data.}
    \label{fig-mm_lam_sim2}
  \end{minipage}
\end{figure}

%e40
\begin{figure}[!h]
  \begin{center}
    \includegraphics[clip,width=12cm]{mm_lam.png}
    \caption{Missing mass of the $\PiKL$ reaction without requesting the number of detected particle combinations of case \rom{1} or case \rom{2}. The red solid line represents the fitting function of two Gaussians and a \nth{3}-order polynomial, and the magenta dotted line represents the $\pm3\sigma$ interval of the first Gaussian as the $\Lambda$ peak region.}
    \label{fig-mm_lam}
  \end{center}
\end{figure}

\begin{figure}[!h]
  \begin{minipage}[t]{0.48\columnwidth}
    \centering
    \includegraphics[clip,width=\columnwidth]{mm_lam1.png}
    \caption{Missing mass of the $\PiKL$ reaction in case \rom{1}, taken in the E40 data. The red solid line represents the fitting function of two Gaussians and a \nth{3}-order polynomial, and the magenta dotted line represents the $\Lambda$ peak region defined by Figure \ref{fig-mm_lam}.}
    \label{fig-mm_lam1}
  \end{minipage}
  \hspace{0.04\columnwidth} % ここで隙間作成
  \begin{minipage}[t]{0.48\columnwidth}
    \centering
    \includegraphics[clip, width=\columnwidth]{mm_lam2.png}
    \caption{Missing mass of the $\PiKL$ reaction in case \rom{2}, taken in the E40 data. The red solid line represents the fitting function of two Gaussians and a \nth{3}-order polynomial, and the magenta dotted line represents the $\Lambda$ peak region defined by Figure \ref{fig-mm_lam}.}
    \label{fig-mm_lam2}
  \end{minipage}
\end{figure}

\end{comment}

%=====table=====%
\begin{table}[!tbph]
  \begin{center}
    \caption{The numerical values involved in the fitting of the $\PiKX$ reaction without requesting the number of detected particle combinations (All), case \rom{1}, and case \rom{2}.}
    \begin{tabular}{ccc}\hline\hline
      Detection case & $\Lambda$ yield & S/N \\ \hline
      All & $3.989\times10^{5}$ & 0.232 \\ \hline
      case \rom{1} & $6.976\times10^{4}$ & 1.523 \\ \hline
      case \rom{2} & $2.721\times10^{3}$ & 1.471 \\ \hline
    \end{tabular}
    \label{tab-MMfit}
  \end{center}
\end{table}

%%%%
\subsection{Missing momentum of the $\PiKX$ reaction}
Using the four-momentum vector of the missing particle in the $\PiKL$ reaction obtained by Equation (\ref{eq-mmom_lam}), its momentum, namely the missing momentum, can be measured. Figure \ref{fig-mmom0_1p1pi_sim} and Figure \ref{fig-mmom0_2p_sim} are the Monte Carlo simulation results in detection case \rom{1} and case \rom{2}, respectively. It indicates that $\Lp$ scattering events usually distribute in the momentum range of $0.25 - 1.25$ GeV/$c$. Therefore, both the beam $\Lambda$ polarization measurement (case \rom{1}) and the $\Lp$ scattering identification (case \rom{2}) were performed in such a momentum range. Figure \ref{fig-mmom0_1p1pi} and Figure \ref{fig-mmom0_2p} show the missing momentum taken in E40 data, where the red solid lines represent the event selection cut.

%mmom0 sim
\begin{figure}[!h]
  \begin{center}
    \includegraphics[clip,width=12cm]{mmom0_1p1pi_sim.eps}
    \caption{Opening angle between $\pM$ and $\pP$ of the $\kzdecay$ decay in case \rom{1}, taken in simulation data.}
    \label{fig-mmom0_1p1pi_sim}
  \end{center}
\end{figure}

\begin{figure}[!h]
  \begin{center}
    \includegraphics[clip,width=12cm]{mmom0_2p_sim.eps}
    \caption{Opening angle between $\pM$ and $\pP$ of the $\kzdecay$ decay in case \rom{2}, taken in simulation data.}
    \label{fig-mmom0_2p_sim}
  \end{center}
\end{figure}

%mmom0 e40
\begin{figure}[!h]
  \begin{center}
    \includegraphics[clip,width=12cm]{mmom0_1p1pi.eps}
    \caption{Opening angle between $\pM$ and $\pP$ of the $\kzdecay$ decay in case \rom{1}, taken in E40 data.}
    \label{fig-mmom0_1p1pi}
  \end{center}
\end{figure}

\begin{figure}[!h]
  \begin{center}
    \includegraphics[clip,width=12cm]{mmom0_2p.eps}
    \caption{Opening angle between $\pM$ and $\pP$ of the $\kzdecay$ decay in case \rom{2}, taken in E40 data.}
    \label{fig-mmom0_2p}
  \end{center}
\end{figure}

\begin{comment}

% sim
\begin{figure}[!h]
  \begin{minipage}[t]{0.48\columnwidth}
    \centering
    \includegraphics[clip,width=\columnwidth]{mmom_sim1.png}
    \caption{Simulation of the distance of closest approach between the $\pM$ beam and reconstructed $\Kz$ tracks in case \rom{1}.}
    \label{fig-mmom_sim1}
  \end{minipage}
  \hspace{0.04\columnwidth} % ここで隙間作成
  \begin{minipage}[t]{0.48\columnwidth}
    \centering
    \includegraphics[clip, width=\columnwidth]{mmom_sim2.png}
    \caption{Simulation of the distance of closest approach between the $\pM$ beam and reconstructed $\Kz$ tracks in case \rom{2}.}
    \label{fig-mmom_sim2}
  \end{minipage}
\end{figure}

% e40
\begin{figure}[!h]
  \begin{minipage}[t]{0.48\columnwidth}
    \centering
    \includegraphics[clip,width=\columnwidth]{mmom_1.png}
    \caption{Missing momentum of the $\PiKL$ reaction in case \rom{1}, taken in the E40 data.}
    \label{fig-mmom_1}
  \end{minipage}
  \hspace{0.04\columnwidth} % ここで隙間作成
  \begin{minipage}[t]{0.48\columnwidth}
    \centering
    \includegraphics[clip, width=\columnwidth]{mmom_2.png}
    \caption{Missing momentum of the $\PiKL$ reaction in case \rom{2}, taken in the E40 data.}
    \label{fig-mmom_2}
  \end{minipage}
\end{figure}

\end{comment}
%%%%%%%%%%%%
%%%%%%%%%%%%
%\end{document}

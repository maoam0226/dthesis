%\documentclass[a4paper,12pt,oneside,openany]{jsbook}
%%\setlength{\topmargin}{10mm}
%\addtolength{\topmargin}{-1in}
%\setlength{\oddsidemargin}{27mm}
%\addtolength{\oddsidemargin}{-1in}
%\setlength{\evensidemargin}{20mm}
%\addtolength{\evensidemargin}{-1in}
%\setlength{\textwidth}{160mm}
%\setlength{\textheight}{250mm}
%\setlength{\evensidemargin}{\oddsidemargin}

%\usepackage{ascmac}

\usepackage{color}
\usepackage{textcomp}
%\usepackage[dviout]{graphicx}
%\usepackage[dvipdfm]{graphicx,color}
\usepackage{wrapfig}
\usepackage{ccaption}
\usepackage{color}
%\usepackage{jumoline} %%行にまたがって下線を引ける、ダウンロードの必要有
\usepackage{umoline}
\usepackage{fancybox}
\usepackage{pifont}
\usepackage{indentfirst} %%最初の段落も字下げしてくれる

\usepackage{amsmath,amssymb,amsfonts}
\usepackage{bm}
%\usepackage{graphicx}
\usepackage[dvipdfmx]{graphicx}
%\usepackage[dvipsnames]{xcolor}
\usepackage{subfigure}
\usepackage{verbatim}
\usepackage{makeidx}
\usepackage{accents}
%\usepackage{slashbox} %%ダウンロードの必要有

\usepackage[dvipdfmx]{hyperref} %%pdfにリンクを貼る
\usepackage{pxjahyper}

\usepackage[flushleft]{threeparttable}
\usepackage{array,booktabs,makecell}

\usepackage{geometry}
\geometry{left=30mm,right=30mm,top=50mm,bottom=5mm}

\usepackage[super]{nth} %1st, 2nd ...を出力
\usepackage{dirtytalk} %クォーテーションマーク
\usepackage{amsmath} %行列が書ける
\usepackage{tikz} %\UTF{2460}などが書ける
\usepackage{cite} %複数の引用ができる

\usepackage[toc,page]{appendix}

%\graphicspath{{./pictures/}}

%\setlength{\textwidth}{\fullwidth}
\setlength{\textheight}{40\baselineskip}
\addtolength{\textheight}{\topskip}
\setlength{\voffset}{-0.55in}

\renewcommand{\baselinestretch}{1} %% 行間

%\setcounter{tocdepth}{5}  %% 目次section depth
\setcounter{secnumdepth}{5}
%\renewcommand{\bibname}{参考文献}

%%%%%%%%% accent.sty 設定 %%%%%%%%%
\makeatletter
  \def\widebar{\accentset{{\cc@style\underline{\mskip10mu}}}}
\makeatother

%%%%%%%%%  chapter 設定 %%%%%%%%%%%
%\makeatletter
%\def\@makechapterhead#1{%
%  \vspace*{1\Cvs}% 欧文は50pt 章上部の空白
%  {\parindent \z@ \raggedright \normalfont
%    \ifnum \c@secnumdepth >\m@ne
%      \if@mainmatter
%        \huge\headfont \@chapapp\thechapter\@chappos
%       \par\nobreak
%       \vskip \Cvs % 欧文は20pt
%         \hskip1zw
%      \fi
%    \fi
%    \interlinepenalty\@M
%    \centering \huge \headfont #1\par\nobreak
%    \vskip 3\Cvs}} % 欧文は40pt 章下部の空白
%\makeatother

%%%%%%%%%  chapter* 設定 %%%%%%%%%%%



%%%%%%%%%  chapter* 設定 %%%%%%%%%%%

%\makeatletter
%\def\@makeschapterhead#1{%
%  \vspace*{1\Cvs}
%  {\parindent \z@ \raggedright
%    \normalfont
%    \interlinepenalty\@M
%    \centering \huge \headfont #1\par\nobreak
%    \vskip 3\Cvs}}
%\makeatother

%%%%%%%%%  section 設定 %%%%%%%%%%%
\makeatletter
\renewcommand{\section}{%
  \@startsection{section}%
   {1}%
   {\z@}%
   {-3.5ex \@plus -1ex \@minus -.2ex}%
   {2.3ex \@plus.2ex}%
   {\normalfont\Large\bfseries}%
}%
\makeatother

%%%%%%%%%  subsection 設定 %%%%%%%%%%%
\makeatletter
\renewcommand{\subsection}{%
  \@startsection{subsection}%
   {2}%
   {\z@}%
   {-3.5ex \@plus -1ex \@minus -.2ex}%
   {2.3ex \@plus.2ex}%
   {\normalfont\large\bfseries}%
}%
\makeatother

%%%%%%%%%  subsubsection 設定 %%%%%%%%%%%
\makeatletter
\renewcommand{\subsubsection}{%
  \@startsection{subsubsection}%
   {3}%
   {\z@}%
   {-3.5ex \@plus -1ex \@minus -.2ex}%
   {2.3ex \@plus.2ex}%
   %{\normalfont\normalsize\bfseries$\blacksquare$}%
   {\normalfont\normalsize\bfseries}%
}%
\makeatother

%%%%%%%%%  paragraph 設定 %%%%%%%%%%%
\makeatletter
\renewcommand{\paragraph}{%
  \@startsection{paragraph}%
   {4}%
   {\z@}%
   {0.5\Cvs \@plus.5\Cdp \@minus.2\Cdp}
   {-1zw}
   {\normalfont\normalsize\bfseries $\blacklozenge$\ }%
  % {\normalfont\normalsize\bfseries $\Diamond$\ }%
}%
\makeatother

%%%%%%%%%  subparagraph 設定 %%%%%%%%%%%
\makeatletter
\renewcommand{\subparagraph}{%
  \@startsection{subparagraph}%
   {4}%
   {\z@}%
   {0.5\Cvs \@plus.5\Cdp \@minus.2\Cdp}
   {-1zw}
   {\normalfont\normalsize\bfseries $\Diamond$\ }%
}%
\makeatother

%%%%%%%%% caption 設定 %%%%%%%%%%%%
\makeatletter

\newcommand*\circled[1]{\tikz[baseline=(char.base)]{
            \node[shape=circle,draw,inner sep=2pt] (char) {#1};}}

\newcommand*{\rom}[1]{\expandafter\@slowromancap\romannumeral #1@}

\newcommand{\msolar}{M_\odot}

\newcommand{\anapow}{A_{y}(\theta)}
\newcommand{\depo}{D^{y}_{y}(\theta)}

\newcommand{\figcaption}[1]{\def\@captype{figure}\caption{#1}}
\newcommand{\tblcaption}[1]{\def\@captype{table}\caption{#1}}
\newcommand{\klpionn}{K_L \to \pi^0 \nu \overline{\nu}}
\newcommand{\kppipnn}{K^+ \to \pi^+ \nu \overline{\nu}}
\newcommand{\hfl}{{}_\Lambda^4\rm{H}}
\newcommand{\htl}{{}_\Lambda^3\rm{H}}
\newcommand{\hefl}{{}_\Lambda^4\rm{He}}
\newcommand{\hefil}{{}_\Lambda^5\rm{He}}
\newcommand{\lisl}{{}_\Lambda^7\rm{Li}}
\newcommand{\benl}{{}_\Lambda^9\rm{Be}}
\newcommand{\btl}{{}_\Lambda^{10}\rm{B}}
\newcommand{\bel}{{}_\Lambda^{11}\rm{B}}

\newcommand{\nfl}{{}_\Lambda^{15}\rm{N}}
\newcommand{\osl}{{}_\Lambda^{16}\rm{O}}
\newcommand{\ctl}{{}_\Lambda^{13}\rm{C}}
\newcommand{\pbtl}{{}_\Lambda^{208}\rm{Pb}}

\def\vector#1{\mbox{\boldmath$#1$}}
\newcommand{\Kpi}{(K^-,\pi^-)}
\newcommand{\piKz}{(\pi^-,K^0)}
\newcommand{\pPK}{(\pi^+,K^+)}
\newcommand{\pMK}{(\pi^-,K^+)}
\newcommand{\pPMK}{(\pi^{\pm},K^+)}

\newcommand{\eeK}{(e,e' K^+)}
\newcommand{\gK}{(\gamma + p \to \Lambda + K^+)}
\newcommand{\PiKL}{\pi^-  p \to K^0 \Lambda}
\newcommand{\multipi}{\pi^-  p \to \pi^-\pi^-\pi^+p}
\newcommand{\PiKX}{\pi^-  p \to K^0 X}
\newcommand{\PiKSM}{\pi^-  p \to K^+ \Sigma^-}
\newcommand{\pipKS}{\pi^{\pm}p \to K^+ \Sigma^{\pm}}
\newcommand{\pipKX}{\pi^{\pm}p \to K^+ X}
\newcommand{\pipLn}{\pi^- p \to \Lambda n}
\newcommand{\PiKS}{\pi^{-}p \to K^{0}\Sigma^{0}}

\newcommand{\kzdecay}{K^0 \to \pi^+ \pi^-}
\newcommand{\kzsd}{K^0_s \to \pi^+ \pi^-\ \rm{or}\ \pi^0 \pi^0}
\newcommand{\Ldecay}{\Lambda\to p\pM}
\newcommand{\scatldecay}{\Lambda'\to p\pM}




\newcommand{\triton}{{}^3\rm{H}}

\newcommand{\BB}{B_{8}B_{8}}
\newcommand{\SM}{\Sigma^{-}}
\newcommand{\SP}{\Sigma^{+}}
\newcommand{\Sz}{\Sigma^{0}}
\newcommand{\SMp}{\Sigma^{-}p}
\newcommand{\SMn}{\Sigma^{-}n}
\newcommand{\SPp}{\Sigma^{+}p}
\newcommand{\SPn}{\Sigma^{+}n}
\newcommand{\Sp}{\Sigma p}
\newcommand{\SPMp}{\Sigma^{\pm}p}
\newcommand{\SPM}{\Sigma^{\pm}}
\newcommand{\SPdecay}{\Sigma^+ \to \pi^0 p}
\newcommand{\SMdecay}{\Sigma^- \to \pi^- n}
\newcommand{\SMpLn}{\Sigma^- p \to \Lambda n}

\newcommand{\XM}{\Xi^{-}}
\newcommand{\Xz}{\Xi^{0}}

\newcommand{\pM}{\pi^{-}}
\newcommand{\pP}{\pi^{+}}
\newcommand{\pZ}{\pi^{0}}
\newcommand{\pPM}{\pi^{\pm}}
\newcommand{\KP}{K^{+}}
\newcommand{\KM}{K^{-}}
\newcommand{\Kz}{K^{0}}
\newcommand{\Lp}{\Lambda p}
\newcommand{\LpLX}{\Lambda p \to \Lambda X}

\newcommand{\LN}{\Lambda N}
\newcommand{\SN}{\Sigma N}
\newcommand{\LNtoSN}{\Lambda N\to\Sigma N}
\newcommand{\LS}{\Lambda - \Sigma}

%\newcommand{\dp}{\Delta p}
%\newcommand{\dE}{\Delta E}

\newcommand{\dcs}{d\sigma/d\Omega}
\newcommand{\fdcs}{\frac{d\sigma}{d\Omega}}
\newcommand{\dz}{\Delta z}
\newcommand{\dzkz}{\Delta z_{K^{0}}}


\newcommand{\bgct}{\beta\gamma c\tau}

\newcommand{\costp}{\cos{\theta_p}}
\newcommand{\costkz}{\cos{\theta_{K0,CM}}}
\newcommand{\costcm}{\cos{\theta}_{CM}}
\newcommand{\PL}{P_{\Lambda}}
\newcommand{\PLall}{P_{\Lambda,\ all}}
\newcommand{\PLsele}{P_{\Lambda,\ selected}}
\newcommand{\errPL}{\sigma(P_{\Lambda})}

\newcommand{\rud}{r_{ud}}
\newcommand{\errrud}{\sigma(\rud)}

\newcommand{\accPL}{\epsilon_{\PL}}
\newcommand{\erraccPL}{\sigma(\epsilon_{\PL})}

\newcommand{\PLscat}{P_{\Lambda'}}
\newcommand{\effPLw}{\epsilon_{\PL,\ w/}}
\newcommand{\erreffPLw}{\sigma(\epsilon_{\PL,\ w/})}
\newcommand{\effPLwo}{\epsilon_{\PL,\ w/o}}
\newcommand{\erreffPLwo}{\sigma(\epsilon_{\PL,\ w/o})}

\newcommand{\chisq}{\chi^{2}}

\newcommand{\centered}[1]{\begin{tabular}{l} #1 \end{tabular}}

\makeatother

\begin{document}


\chapter{Acknowledgements}
%\addcontentsline{toc}{chapter}{謝辞}

Throughout the writing of this dissertation, I have received a great deal of support and assistance. This has truly been a wonderful journey.

First, I would like to express my sincere gratitude to my advisor, Prof. Koji Miwa. He has been my mentor since 2018 and was the first to take me to the field of nuclear physics experiments. At that time, the J-PARC E40 experiment was just about to begin, and he allowed me to participate in the preparatory work, giving me many opportunities to interact with detectors and electronics. It is no exaggeration to say that I learned a lot here and was allowed to expand my curiosity, which will continue in the future. He is the most hard-working, insightful, and kindest person I have ever met. He always took the lead in the J-PARC E40 experiment and led to the successful measurement of the $\SPMp$ scattering cross-sections. Additionally, he was very supportive in providing essential advice in data analysis and paper writing and was always willing to take time out of his busy schedule to assist me. Over the past five years, he has always encouraged and nurtured me. Thank you for everything.

I want to express my deep gratitude to Prof. Hirokazu Tamura. He was the first person to lecture me on the fun of nuclear physics. During graduate school lectures, he always enjoyed talking about physics, which fascinated me. While promoting projects at J-PARC, he always helped and guided me in laboratory meetings with his great expertise and ability to explain complex physics topics. He is a very friendly, warm-hearted person with a great sense of humor and always cares about his students. I will never forget that he has been watching over me and raising me since 2018. I am sincerely grateful.

I would like to thank the J-PARC K1.8 beamline members and collaborators. In addition to the J-PARC E40 experiment, I also participated in many experiments conducted at the Hadron Experimental Facility, where I learned about detectors, experimental setups, and data acquisition systems. Thank everyone who kindly accepted my participation. The time I spent at the experimental facility was a very valuable experience and has become an irreplaceable memory.

I would like to thank the shift members who worked to ensure the stable operation of the detector 24 hours a day during the J-PARC E40 experiment. Especially during the $\SPp$ data-taking period, COVID-19 was raging, and the situation was very difficult, but everyone's support allowed us to obtain wonderful physical data. Thank you, Prof. Koji Miwa, Prof. Toshiyuki Takahashi, Prof. Kiyoshi Tanida, Dr. Yuya Akazawa, Dr. Manami Fujita, Prof. Shuhei Hayakawa, Prof. Yudai Ichikawa, Prof. Mifuyu Ukai, Dr. Takeshi O. Yamamoto, Dr. Takuya Nanamura, Dr. Yuji Ishikawa, Ms. Honoka Kanauchi, Mr. Tomomasa Kitaoka, and Mr. Shunsuke Wada.

In daily meetings at the laboratory, I had many discussions about analysis with the core members of the J-PARC E40 experiment. I received a lot of professional advice, which was very helpful. Thank you, Prof. Koji Miwa, Prof. Toshiyuki Takahashi, Prof. Ryotaro Honda, Dr. Yuya Akazawa, Prof. Shuhei Hayakawa, Prof. Mifuyu Ukai, Dr. Takeshi O .Yamamoto, Mr. Yoshiyuki Nakada, and Dr. Takuya Nanamura. 

I also thank the staff of the J-PARC accelerator and the Hadron Experimental Facility for their support in providing the beam during beam time.

I express my sincere gratitude to our laboratory members, Prof. Hirokazu Tamura, Prof. Satoshi N. Nakamura, Prof. Koji Miwa, Prof. Yudai Ichikawa, Prof. Mifuyu Ukai, Prof. Takeshi Koike, Prof. Masashi Kaneta, Prof. Sho Nagao, and Prof. Shuhei Hayakawa have provided careful guidance and warm support through my studies. Thank you for being so supportive. Your intelligence and enthusiasm for research will continue to be the best role models for me.

I thank Mr. Hiroo Umetsu, a technical staff member, for his dedicated efforts. He made parts for experimental equipment and took care of documentation to always support us. I thank Mrs. Ayumi Takahashi, a secretary member, for her vitality and politeness. She supported my time at the laboratory by taking on various tasks, from administering KAKENHI funds to handling university administrative procedures. Whenever I met her, she was full of energy, and she always encouraged me.

I have made many friendships with seniors, classmates, and juniors at this laboratory. They have always been there to lend a hand and enjoy physics when needed. Thank you, Dr. Manami Fujita, Ms. Honoka Kanauchi, Dr. Yuichi Toyama, Dr. Kosuke Itabashi, Ms. Anya Rogers, Mr. Kunpei Matsuda, Mr. Takashi Aramaki, Mr. Takeru Akiyama, Mr. Kazuki Okuyama, Mr. Shunsuke Kajikawa, Mr. Tomomasa Fujiwara, Mr. Kento Kamada, Mr. Masaya Mizuno, Mr. Fumiya Oura, Ms. Ryoko Kino, Mr. Tatsuhiro Ishige, Mr. Tomomasa Kitaoka, Mr. Shunsuke Wada, Mr. Taito Morino, Mr. Kazuma Ohashi, Mr. Chesu Son, Mr. Hayato Miyata, Mr. Daigo Watanabe, Mr. Ryo Imamoto, Mr. Rintaro Kurata, and Mr. Ryuta Saito.

Finally, and above all, I dedicate this paper to my family. I am truly grateful to my mother, Chizuko Sakao, and father, Masayuki Sakao. I couldn't have gotten here without their love and support. Thank you with love.

\section*{Funding resources}
The J-PARC E40 experiment was supported by JSPS KAKENHI Grant Number 23684011, 15H00838, 15H05442, 15H02079 and 18H03693. This experiment was also supported by Grants-in-Aid Number 24105003 and 18H05403 for Scientific Research from the Ministry of Education, Culture, Science and Technology (MEXT), Japan.

This paper was supported by JSPS KAKENHI for DC1 Grant Number 21J20154 and 22KJ0166.
%\end{document}

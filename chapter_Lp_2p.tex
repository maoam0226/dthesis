%\documentclass[a4paper,12pt,oneside,openany]{jsbook}
%%\setlength{\topmargin}{10mm}
%\addtolength{\topmargin}{-1in}
%\setlength{\oddsidemargin}{27mm}
%\addtolength{\oddsidemargin}{-1in}
%\setlength{\evensidemargin}{20mm}
%\addtolength{\evensidemargin}{-1in}
%\setlength{\textwidth}{160mm}
%\setlength{\textheight}{250mm}
%\setlength{\evensidemargin}{\oddsidemargin}

%\usepackage{ascmac}

\usepackage{color}
\usepackage{textcomp}
%\usepackage[dviout]{graphicx}
%\usepackage[dvipdfm]{graphicx,color}
\usepackage{wrapfig}
\usepackage{ccaption}
\usepackage{color}
%\usepackage{jumoline} %%行にまたがって下線を引ける、ダウンロードの必要有
\usepackage{umoline}
\usepackage{fancybox}
\usepackage{pifont}
\usepackage{indentfirst} %%最初の段落も字下げしてくれる

\usepackage{amsmath,amssymb,amsfonts}
\usepackage{bm}
%\usepackage{graphicx}
\usepackage[dvipdfmx]{graphicx}
%\usepackage[dvipsnames]{xcolor}
\usepackage{subfigure}
\usepackage{verbatim}
\usepackage{makeidx}
\usepackage{accents}
%\usepackage{slashbox} %%ダウンロードの必要有

\usepackage[dvipdfmx]{hyperref} %%pdfにリンクを貼る
\usepackage{pxjahyper}

\usepackage[flushleft]{threeparttable}
\usepackage{array,booktabs,makecell}

\usepackage{geometry}
\geometry{left=30mm,right=30mm,top=50mm,bottom=5mm}

\usepackage[super]{nth} %1st, 2nd ...を出力
\usepackage{dirtytalk} %クォーテーションマーク
\usepackage{amsmath} %行列が書ける
\usepackage{tikz} %\UTF{2460}などが書ける
\usepackage{cite} %複数の引用ができる

\usepackage[toc,page]{appendix}

%\graphicspath{{./pictures/}}

%\setlength{\textwidth}{\fullwidth}
\setlength{\textheight}{40\baselineskip}
\addtolength{\textheight}{\topskip}
\setlength{\voffset}{-0.55in}

\renewcommand{\baselinestretch}{1} %% 行間

%\setcounter{tocdepth}{5}  %% 目次section depth
\setcounter{secnumdepth}{5}
%\renewcommand{\bibname}{参考文献}

%%%%%%%%% accent.sty 設定 %%%%%%%%%
\makeatletter
  \def\widebar{\accentset{{\cc@style\underline{\mskip10mu}}}}
\makeatother

%%%%%%%%%  chapter 設定 %%%%%%%%%%%
%\makeatletter
%\def\@makechapterhead#1{%
%  \vspace*{1\Cvs}% 欧文は50pt 章上部の空白
%  {\parindent \z@ \raggedright \normalfont
%    \ifnum \c@secnumdepth >\m@ne
%      \if@mainmatter
%        \huge\headfont \@chapapp\thechapter\@chappos
%       \par\nobreak
%       \vskip \Cvs % 欧文は20pt
%         \hskip1zw
%      \fi
%    \fi
%    \interlinepenalty\@M
%    \centering \huge \headfont #1\par\nobreak
%    \vskip 3\Cvs}} % 欧文は40pt 章下部の空白
%\makeatother

%%%%%%%%%  chapter* 設定 %%%%%%%%%%%



%%%%%%%%%  chapter* 設定 %%%%%%%%%%%

%\makeatletter
%\def\@makeschapterhead#1{%
%  \vspace*{1\Cvs}
%  {\parindent \z@ \raggedright
%    \normalfont
%    \interlinepenalty\@M
%    \centering \huge \headfont #1\par\nobreak
%    \vskip 3\Cvs}}
%\makeatother

%%%%%%%%%  section 設定 %%%%%%%%%%%
\makeatletter
\renewcommand{\section}{%
  \@startsection{section}%
   {1}%
   {\z@}%
   {-3.5ex \@plus -1ex \@minus -.2ex}%
   {2.3ex \@plus.2ex}%
   {\normalfont\Large\bfseries}%
}%
\makeatother

%%%%%%%%%  subsection 設定 %%%%%%%%%%%
\makeatletter
\renewcommand{\subsection}{%
  \@startsection{subsection}%
   {2}%
   {\z@}%
   {-3.5ex \@plus -1ex \@minus -.2ex}%
   {2.3ex \@plus.2ex}%
   {\normalfont\large\bfseries}%
}%
\makeatother

%%%%%%%%%  subsubsection 設定 %%%%%%%%%%%
\makeatletter
\renewcommand{\subsubsection}{%
  \@startsection{subsubsection}%
   {3}%
   {\z@}%
   {-3.5ex \@plus -1ex \@minus -.2ex}%
   {2.3ex \@plus.2ex}%
   %{\normalfont\normalsize\bfseries$\blacksquare$}%
   {\normalfont\normalsize\bfseries}%
}%
\makeatother

%%%%%%%%%  paragraph 設定 %%%%%%%%%%%
\makeatletter
\renewcommand{\paragraph}{%
  \@startsection{paragraph}%
   {4}%
   {\z@}%
   {0.5\Cvs \@plus.5\Cdp \@minus.2\Cdp}
   {-1zw}
   {\normalfont\normalsize\bfseries $\blacklozenge$\ }%
  % {\normalfont\normalsize\bfseries $\Diamond$\ }%
}%
\makeatother

%%%%%%%%%  subparagraph 設定 %%%%%%%%%%%
\makeatletter
\renewcommand{\subparagraph}{%
  \@startsection{subparagraph}%
   {4}%
   {\z@}%
   {0.5\Cvs \@plus.5\Cdp \@minus.2\Cdp}
   {-1zw}
   {\normalfont\normalsize\bfseries $\Diamond$\ }%
}%
\makeatother

%%%%%%%%% caption 設定 %%%%%%%%%%%%
\makeatletter

\newcommand*\circled[1]{\tikz[baseline=(char.base)]{
            \node[shape=circle,draw,inner sep=2pt] (char) {#1};}}

\newcommand*{\rom}[1]{\expandafter\@slowromancap\romannumeral #1@}

\newcommand{\msolar}{M_\odot}

\newcommand{\anapow}{A_{y}(\theta)}
\newcommand{\depo}{D^{y}_{y}(\theta)}

\newcommand{\figcaption}[1]{\def\@captype{figure}\caption{#1}}
\newcommand{\tblcaption}[1]{\def\@captype{table}\caption{#1}}
\newcommand{\klpionn}{K_L \to \pi^0 \nu \overline{\nu}}
\newcommand{\kppipnn}{K^+ \to \pi^+ \nu \overline{\nu}}
\newcommand{\hfl}{{}_\Lambda^4\rm{H}}
\newcommand{\htl}{{}_\Lambda^3\rm{H}}
\newcommand{\hefl}{{}_\Lambda^4\rm{He}}
\newcommand{\hefil}{{}_\Lambda^5\rm{He}}
\newcommand{\lisl}{{}_\Lambda^7\rm{Li}}
\newcommand{\benl}{{}_\Lambda^9\rm{Be}}
\newcommand{\btl}{{}_\Lambda^{10}\rm{B}}
\newcommand{\bel}{{}_\Lambda^{11}\rm{B}}

\newcommand{\nfl}{{}_\Lambda^{15}\rm{N}}
\newcommand{\osl}{{}_\Lambda^{16}\rm{O}}
\newcommand{\ctl}{{}_\Lambda^{13}\rm{C}}
\newcommand{\pbtl}{{}_\Lambda^{208}\rm{Pb}}

\def\vector#1{\mbox{\boldmath$#1$}}
\newcommand{\Kpi}{(K^-,\pi^-)}
\newcommand{\piKz}{(\pi^-,K^0)}
\newcommand{\pPK}{(\pi^+,K^+)}
\newcommand{\pMK}{(\pi^-,K^+)}
\newcommand{\pPMK}{(\pi^{\pm},K^+)}

\newcommand{\eeK}{(e,e' K^+)}
\newcommand{\gK}{(\gamma + p \to \Lambda + K^+)}
\newcommand{\PiKL}{\pi^-  p \to K^0 \Lambda}
\newcommand{\multipi}{\pi^-  p \to \pi^-\pi^-\pi^+p}
\newcommand{\PiKX}{\pi^-  p \to K^0 X}
\newcommand{\PiKSM}{\pi^-  p \to K^+ \Sigma^-}
\newcommand{\pipKS}{\pi^{\pm}p \to K^+ \Sigma^{\pm}}
\newcommand{\pipKX}{\pi^{\pm}p \to K^+ X}
\newcommand{\pipLn}{\pi^- p \to \Lambda n}
\newcommand{\PiKS}{\pi^{-}p \to K^{0}\Sigma^{0}}

\newcommand{\kzdecay}{K^0 \to \pi^+ \pi^-}
\newcommand{\kzsd}{K^0_s \to \pi^+ \pi^-\ \rm{or}\ \pi^0 \pi^0}
\newcommand{\Ldecay}{\Lambda\to p\pM}
\newcommand{\scatldecay}{\Lambda'\to p\pM}




\newcommand{\triton}{{}^3\rm{H}}

\newcommand{\BB}{B_{8}B_{8}}
\newcommand{\SM}{\Sigma^{-}}
\newcommand{\SP}{\Sigma^{+}}
\newcommand{\Sz}{\Sigma^{0}}
\newcommand{\SMp}{\Sigma^{-}p}
\newcommand{\SMn}{\Sigma^{-}n}
\newcommand{\SPp}{\Sigma^{+}p}
\newcommand{\SPn}{\Sigma^{+}n}
\newcommand{\Sp}{\Sigma p}
\newcommand{\SPMp}{\Sigma^{\pm}p}
\newcommand{\SPM}{\Sigma^{\pm}}
\newcommand{\SPdecay}{\Sigma^+ \to \pi^0 p}
\newcommand{\SMdecay}{\Sigma^- \to \pi^- n}
\newcommand{\SMpLn}{\Sigma^- p \to \Lambda n}

\newcommand{\XM}{\Xi^{-}}
\newcommand{\Xz}{\Xi^{0}}

\newcommand{\pM}{\pi^{-}}
\newcommand{\pP}{\pi^{+}}
\newcommand{\pZ}{\pi^{0}}
\newcommand{\pPM}{\pi^{\pm}}
\newcommand{\KP}{K^{+}}
\newcommand{\KM}{K^{-}}
\newcommand{\Kz}{K^{0}}
\newcommand{\Lp}{\Lambda p}
\newcommand{\LpLX}{\Lambda p \to \Lambda X}

\newcommand{\LN}{\Lambda N}
\newcommand{\SN}{\Sigma N}
\newcommand{\LNtoSN}{\Lambda N\to\Sigma N}
\newcommand{\LS}{\Lambda - \Sigma}

%\newcommand{\dp}{\Delta p}
%\newcommand{\dE}{\Delta E}

\newcommand{\dcs}{d\sigma/d\Omega}
\newcommand{\fdcs}{\frac{d\sigma}{d\Omega}}
\newcommand{\dz}{\Delta z}
\newcommand{\dzkz}{\Delta z_{K^{0}}}


\newcommand{\bgct}{\beta\gamma c\tau}

\newcommand{\costp}{\cos{\theta_p}}
\newcommand{\costkz}{\cos{\theta_{K0,CM}}}
\newcommand{\costcm}{\cos{\theta}_{CM}}
\newcommand{\PL}{P_{\Lambda}}
\newcommand{\PLall}{P_{\Lambda,\ all}}
\newcommand{\PLsele}{P_{\Lambda,\ selected}}
\newcommand{\errPL}{\sigma(P_{\Lambda})}

\newcommand{\rud}{r_{ud}}
\newcommand{\errrud}{\sigma(\rud)}

\newcommand{\accPL}{\epsilon_{\PL}}
\newcommand{\erraccPL}{\sigma(\epsilon_{\PL})}

\newcommand{\PLscat}{P_{\Lambda'}}
\newcommand{\effPLw}{\epsilon_{\PL,\ w/}}
\newcommand{\erreffPLw}{\sigma(\epsilon_{\PL,\ w/})}
\newcommand{\effPLwo}{\epsilon_{\PL,\ w/o}}
\newcommand{\erreffPLwo}{\sigma(\epsilon_{\PL,\ w/o})}

\newcommand{\chisq}{\chi^{2}}

\newcommand{\centered}[1]{\begin{tabular}{l} #1 \end{tabular}}

\makeatother

\begin{document}


%\renewcommand{\labelitemi}{・}
%\renewcommand{\labelitemii}{・}
\graphicspath{{./pictures/chapter_Lp_2p}}

\chapter{Analysis \rom{3}: \\$\Lp$ scattering identification} 
\label{chap-Lp_2p}

%%%%%
\section{Overview}
\textcolor{red}{ In detection case \rom{2}, $\Lp$ scattering identification was performed after beam $\Lambda$ was tagged. As mentioned before, the \say{without $\pM$} mode assumes that CATCH detects recoil proton and proton from $\scatldecay$ decay. The \say{with $\pM$} mode assumes that CATCH detects all final state particles (recoil proton, $\scatldecay$ decay products). In each mode, $\Lp$ scattering events were identified by kinematical consistency analysis. In this analysis, the energy of the recoil proton or the momentum of $\Lambda'$ is calculated kinematically. $\Lp$ scattering events were identified by confirming whether the calculated value is almost the same as the measured value. }

The energy of the recoil proton was expected to be small enough to be stopped within the BGO calorimeter so its momentum and trajectory could be measured. The recoil proton energy has already been corrected by the proton's vector direction and the vertex position between beam $\Lambda$ and recoil proton, taking into account the energy deposit in the target material CFRP.

The main background, $pp$ elastic scattering, was removed by the opening angle between the two protons. Although the analyzed $\PiKL$ data is a by-product of J-PARC E40 and its statistics are low, it was concluded that $\Lp$ scattering events can be identified even when considering the contamination rate of $pp$ elastic scattering.

This chapter describes the analysis procedures of the $\Lp$ scattering identification performed in case \rom{2} with/without $\pM$ detection.

%In the $\Lp$ scattering analysis, the \say{without $\pM$} mode was performed first. After selecting the missing mass and the beam $\Lambda$ momentum, four cuts were applied: DOCA of the beam $\Lambda$ and recoil proton ({\bf C-2p-0}), $pp$ elastic scattering removal ({\bf C-2p-1}), $\cos{\theta_{CM}}$ selection ({\bf C-2p-2}), and the kinematical consistency check ({\bf C-2p-3}). Next, the \say{with $\pM$} mode was performed if the detected $\pM$ is not the same particle as the one derived from the $\Kz$ decay. Under the cut levels {\bf C-2p-0} and {\bf C-2p-1}, three additional cuts were applied: missing mass of the $\LpLX$ reaction ({\bf C-2p-4}), $\cos{\theta_{CM}}$ selection ({\bf C-2p-5}), and the kinematical consistency check ({\bf C-2p-6}).

%%%%%
\section{Background:\\ $pp$ elastic scattering and $\Ldecay$ decay}
The main background event in the detection case \rom{2} is $pp$ elastic scattering. This reaction occurs when protons from $\Ldecay$ decay collide with target protons. Figure \ref{fig-2p_pp_woPi} shows the case where CATCH detects only two protons when $pp$ elastic scattering occurs. Figure \ref{fig-2p_pp_wPi} shows the case where CATCH detects all final state particles of $pp$ elastic scattering. As you can see, the combination of final state particles is the same as $\Lp$ scattering, which affects the accuracy of $\Lp$ scattering identification.

Another background is $\Ldecay$ decay. As shown in Figure \ref{fig-2p_Ldecay}, there is only one proton in the final state. However, if CATCH simultaneously detects a proton from an accidental background event, it will slightly contribute to the analysis of the $\Lp$ scattering event identification.

Appropriate event cuts removed these background events. Especially regarding the removal of $pp$ elastic scattering, please refer to Sec. \ref{sec-ppRemoval}.

\begin{figure}[!h]
  \begin{center}
    \includegraphics[width=12cm]{2p_pp_woPi.eps}
    \caption{Schematic of $pp$ elastic scattering, the main background in detection case \rom{2} without $\pM$ mode.}
    \label{fig-2p_pp_woPi}
  \end{center}
\end{figure}

\begin{figure}[!h]
  \begin{center}
    \includegraphics[width=12cm]{2p_pp_wPi.eps}
    \caption{Schematic of $pp$ elastic scattering, the main background in detection case \rom{2} with $\pM$ mode. }
    \label{fig-2p_pp_wPi} 
  \end{center}
\end{figure}

\begin{figure}[!h]
  \begin{center}
    \includegraphics[width=12cm]{2p_Ldecay.eps}
    \caption{Schematic of $\Ldecay$ decay, the background in detection case \rom{2}. }
    \label{fig-2p_Ldecay}
  \end{center}
\end{figure}


%%%%%
\clearpage
\section{\nth{1} level cuts}
\label{sec-Lp_common}

After tagging beam $\Lambda$ by the missing mass method, the \nth{1} level cuts to remove $pp$ elastic scattering and to select $\Lp$ scattering events geometrically were applied, as follows. They are common in \say{with/without $\pM$} modes.
\begin{enumerate}
  \item {\bf Removal of $pp$ elastic scattering ({\bf C-2p-1}) } \\
  A kinematical consistency analysis and opening angle between the two protons removed $pp$ elastic scattering events. 
  \item {\bf Selection of $\Lp$ scattering ({\bf C-2p-2}) } \\
  $\Lp$ scattering events were selected by requiring $z-$vertex and the closest distance between the beam $\Lambda$ and recoil proton to be consistent with the simulation results.
\end{enumerate}

%After applying the cuts above, the background subtraction was performed for the $\Delta E$ spectrum. The background structure in the $\Delta E$ spectrum was estimated by normalizing the contamination from the low/high sidebands in the missing mass of the $\PiKX$ reaction, which remained under the cut levels from {\bf C-2p-1} to {\bf C-2p-5}. The normalizing constant for the sideband contamination was calculated by fitting the missing mass. The fitting function was set as a sum of a Gaussian and a linear function. The normalized structure was subtracted from the primary $\Delta E$ spectrum, considering the statistical error from the background subtraction. \textcolor{red}{The systematic error was estimated by changing the selection range of the scattering angle of scattered $\Lambda$.} 


%%%%
\subsection{Removal of $pp$ elastic scattering ({\bf C-2p-1})}
\label{sec-ppRemoval}

The contamination of $pp$ elastic scattering was removed by kinematical consistency analysis and measurement of the opening angle between two protons.

In the kinematical consistency analysis, we assumed that the two protons in the detection case \rom{2} derive from $pp$ elastic scattering. In this case, the momentum of the recoil proton can be calculated as
\begin{align}
  p_{calc} &= \frac{A p_{decay\ p} \cos{\theta_{pp'}} + (E_{decay\ p} + m_{p}) \sqrt{B}}{2((E_{decay\ p} + m_{p})^{2} - p^{2}_{decay\ p} \cos{\theta_{pp'}})}, \\
  \nonumber \\
  A &= m^{2}_{decay\ p} + m^{2}_{p} + m^{2}_{p'} - m_{p''} + 2 E_{decay\ p} m_{p}, \\
  B &= 4 m^{2}_{p'} ( p^{2}_{decay\ p} \cos{\theta_{pp'}} - (E_{decay\ p} + m_{p})^{2} ) + A^{2},
\end{align}
where $p_{decay\ p}$ and $E_{decay\ p}$ are the momentum and energy of the decay proton, $\theta_{pp'}$ is the scattering angle of the recoil proton, and $m_{p}$, $m_{decay\ p}$, $m_{p'}$, and $m_{p''}$ are the masses of target proton, decay proton, recoil proton, and scattered proton, respectively. On the other hand, the measured momentum value ($p_{meas}$) can be obtained by the missing momentum. 
By calculating the difference between $p_{meas}$ and $p_{calc}$ ($\Delta p$), $pp$ elastic scattering was identified. If the scattering indeed occurs, ideally, $\Delta p$ should be 0. This method is called \say{$\Delta p$ method}. 

In addition to that, the opening angle between the recoil proton and scattered proton ($\theta_{p'p''}$) can be measured by CATCH. Removal of $pp$ elastic scattering was performed after confirming the correlation between $\theta_{p'p''}$ versus $\Delta p$.

The correlations between the opening angle ($\theta_{p'p''}$) versus $\Delta p$ for the simulations of the $pp$ elastic scattering, $\Lp$ scattering, and $\Ldecay$ decay are shown in Figure \ref{fig-thetapp_dp_sim}. The upper left shows the sum of all reactions, the upper right shows the $pp$ elastic scattering, the lower left shows the $\Lp$ scattering, and the lower right shows the results of $\Ldecay$ decay scaled to 1/100. The $pp$ elastic scattering events make a locus around the range of $85^{\circ}<\theta_{p'p''}<95^{\circ}$ and $|\Delta p|<0.05$ GeV/$c$. For $\Lp$ scattering, it is also found that it is distributed in the range of $|\Delta p|<0.1$ GeV/$c$, but the major difference from $pp$ elastic scattering is that the opening angle is distributed widely in the range of $50^{\circ}-110^{\circ}$. $\Ldecay$ decay events would no longer be significant, but we found they could also distribute at the range of $\Delta p>0.1$ GeV/$c$. 

Figure \ref{fig-thetapp_dp} shows the same correlation taken in E40 data. Figure \ref{fig-dp_pp} and Figure \ref{fig-thetapp} also show the $y$ projection ($\Delta p$) and the $x$ projection ($\theta_{p'p''}$). As seen in Figure \ref{fig-thetapp}, a clear peak around $90^{\circ}$ appears, which corresponds to the remaining $pp$ elastic scattering contamination. To remove this, the events where $85^{\circ} < \theta_{p'p''} < 95^{\circ}$ were removed. The survival ratio of $\Lp$ scattering for this cut estimated by the simulation was $\sim66.05\%$. This is cut level {\bf C-2p-1}.

%\begin{comment}

%thetapp vs. dp (sim)
\begin{figure}[h]
  \centering
  \includegraphics[width=15cm]{thetapp_dp_sim.eps}
  \caption{Simulations of the correlation plots between the opening angle ($\theta_{p'p''}$) versus $\Delta p$. The upper left shows the sum of all reactions, the upper right shows the $pp$ elastic scattering, the lower left shows the $\Lp$ scattering, and the lower right shows the results of $\Ldecay$ decay scaled to 1/100. }
  \label{fig-thetapp_dp_sim}
\end{figure}

%e40 
\begin{figure}[!h]
  \centering
  \includegraphics[width=12cm]{thetapp_dp_2p.eps}
  \caption{Correlation plot between the opening angle ($\theta_{pp}$) versus $\Delta p$ for the $pp$ elastic scattering, taken in E40 data.}
   \label{fig-thetapp_dp}
\end{figure}

%e40 dp
\begin{figure}[!h]
  \centering
  \includegraphics[width=12cm]{dppp_2p.eps}
  \caption{$\Delta p$ for the $pp$ elastic scattering, taken in E40 data.}
   \label{fig-thetapp}
\end{figure}

%e40 thetapp
\begin{figure}[!h]
  \centering
  \includegraphics[width=12cm]{thetapp_2p.eps}
  \caption{Opening angle between recoil proton and scattered proton assuming $pp$ elastic scattering, taken in E40 data. A cut for $\theta_{p'p''}$ was made at $85^{\circ} < \theta_{p'p''} < 95^{\circ}$, as represented by the red solid lines. The survival ratio of $\Lp$ scattering for this cut was estimated as $\sim66.05\%$. This is cut level {\bf C-2p-1}.}
   \label{fig-dp_pp}
\end{figure}

%\end{comment}


\clearpage
%%%%
\subsection{Selection of $\Lp$ scattering ({\bf C-2p-2})}
The $z-$vertex and closest distance between beam $\Lambda$ and recoil proton were measured by their momentum vectors. Cuts to select $\Lp$ scattering were made referring to the simulation results. 

%%%
\subsubsection{$z$-vertex between the beam $\Lambda$ and recoil proton}

Ideally, when $\Lp$ scattering occurs, its reaction vertex should be located inside the target space. Figure \ref{fig-zvertLp_sim} shows the simulated $z$-vertex between beam $\Lambda$ and recoil proton taken in the sum of all reactions generated (black solid line), $pp$ elastic scattering (red solid line), $\Lp$ scattering (green solid line), and $\Ldecay$ decay scaled to 1/100 after applying cut level {\bf C-2p-1}. For $\Lp$ scattering simulation, the $z$-verteces ($z$) is distributed within the range of $-150< z\ {\rm (mm)} <150$ (target beam axial length). Therefore, a cut to select $\Lp$ scattering was made at the same range for E40 data, as represented by the red solid lines in Figure \ref{fig-zvertLp}.

\begin{figure}[h]
  \centering
  \includegraphics[width=12cm]{zvertLp_sim.eps}
  \caption{Simulation of the $z$-vertex between the beam $\Lambda$ and recoil proton under cut level {\bf C-2p-1}, taken in simulation data. The black solid line represents the results of the sum of all reactions generated, the red one represents the results of the $pp$ elastic scattering, the green one represents the results of the $\Lp$ scattering, and the blue one represents the results of $\Ldecay$ decay scaled to 1/100. Almost all $\Lp$ scattering events would distribute within the target space ($\pm150$ mm).}
  \label{fig-zvertLp_sim}
\end{figure}

\begin{figure}[h]
  \centering
  \includegraphics[width=12cm]{zvertLp.eps}
  \caption{$z$-vertex between the beam $\Lambda$ and recoil proton under cut level {\bf C-2p-1}, taken in E40 data. A cut was made at $-150 < z {\rm mm} < 150$, as represented by the red solid lines.}
  \label{fig-zvertLp}
\end{figure}


%%%
\subsubsection{Closest distance between the beam $\Lambda$ and recoil proton}

If $\Lp$ scattering occurs, ideally the closest distance between the beam $\Lambda$ and the recoil proton should be 0. However, in reality, this distribution has a width due to the position resolution of the detectors and the influence of multiple $\pi$ production.

Figure \ref{fig-cdistLp_sim} shows the simulation results of the closest distance for each reaction (black solid line: simulation sum, red solid line: $pp$ elastic scattering, green solid line: $\Lp$ scattering, blue solid line: $\Ldecay$ decay). From this, we found that most of the $\Lp$ scattering events are distributed in a region $\leq20$ mm. Therefore, in the E40 data analysis, a cut was applied to select only events distributed in this region, as represented by the red solid line in Figure \ref{fig-cdistLp}.

\begin{figure}[h]
  \centering
  \includegraphics[width=12cm]{cdistLp_sim.eps}
  \caption{Simulation of the closest distance between the beam $\Lambda$ and recoil proton under cut level {\bf C-2p-1}, taken in simulation data. The black solid line represents the simulation sum, the red solid line $pp$ elastic scattering, the green solid line $\Lp$ scattering, and the blue solid line $\Ldecay$ decay.}
  \label{fig-cdistLp_sim}
\end{figure}

\begin{figure}[h]
  \centering
  \includegraphics[width=12cm]{cdistLp.eps}
  \caption{Closest distance between the beam $\Lambda$ and recoil proton under cut level {\bf C-2p-1}, taken in E40 data. A cut to select $\Lp$ scattering was made under 20 mm, as represented by the red solid line.}
  \label{fig-cdistLp}
\end{figure}


%%========================without pi mode ここから===================================%%
\clearpage
%%%%%
\section{Without $\pM$ mode}
\label{sec-Lp_2p_woPi}

After applying the \nth{1} level cut in Sec. \ref{sec-Lp_common}, kinematical consistency analysis was performed to investigate the specific scattering angle range suitable for $\Lp$ scattering event identification. %As a result, it was found that most of the $\Lp$ scattering events are distributed in the scattering angle region of $-0.6<\costcm<0.5$.
After selecting the scattering angle region, we estimated the number and identification accuracy of $\Lp$ scattering events using the index ($\Delta E$) determined by kinematical consistency analysis.

%%%%
\subsection{Kinematical consistency analysis and selection of the scattering angle range}
\label{sec-2p_wo_kine}

Using the four-momentum vectors of beam $\Lambda$ and recoil proton ($\bm{P_{\Lambda}}$ and $\bm{P_{p'}}$), the four-momentum vector of $\Lambda'$ ($\bm{P_{\Lambda'}}$) can be obtained as 
\begin{equation}
  \bm{P_{\Lambda'}} = \bm{P_{\Lambda}} - \bm{P_{p'}}. 
\end{equation}
%Then, the scattering angle of $\Lambda'$ in the CM frame ($\theta_{CM}$) was measured. 
Here, we performed the kinematical consistency analysis to select the $\Lp$ scattering events with better S/N. This analysis solves the kinematics of $\Lp$ scattering by assuming that one of the detected protons is the recoil proton. By calculating the difference between the recoil proton's kinetic energy measured by CATCH ($E_{meas}$) and the one kinematically calculated by the scattering angle of recoil proton $\theta_{p'}$ ($E_{calc}$), $\Lp$ scattering events were identified. Ideally, if $\Lp$ scattering occurs, the difference $\Delta E$ should be 0 ($\Delta E = E_{meas} - E_{calc} = 0$). This analysis is called \say{$\Delta E$ method.}

The kinetic energy of the recoil proton can be obtained as 
\begin{align}
  E_{p'\ cal} &= \sqrt{m_{p'}^{2} + p_{p'}^{2}} - m_{p'}, \\
  p_{p'} &= \frac{2m_{p}(E_{\Lambda} + m_{p})p_{\Lambda} \cos^{2}{\theta_{\Lambda p'}}} {(E_{\Lambda} + m_{p})^{2} - p_{\Lambda}^{2} \cos^{2}{\theta_{\Lambda p'}}},
  \label{eq-dE_recop}
\end{align}
where $m_{p}$ and $m_{p'}$ are the masses of the target proton and one of the recoil proton, $p_{p'}$ is the momentum of recoil proton, $E_{\Lambda}$ and $p_{\Lambda}$ are the energy and momentum of beam $\Lambda$, and $\theta_{\Lambda p'}$ is the scattering angle of recoil proton. 

After applying the \nth{1} level cuts, if this analysis is applied to the $pp$ elastic scattering, $\Lp$ scattering, and $\Ldecay$ decay-only events generated in the Monte Carlo simulation, the correlation between the scattering angle of $\Lambda'$ in the CM frame ($\costcm$) versus $\Delta E$ could be like Figure \ref{fig-costdE_sim}. The upper left indicates the sum of all reactions, the upper right $pp$ elastic scattering, the lower left $\Lp$ scattering, and the lower right $\Ldecay$ decay scaled to 1/100. \textcolor{red}{ The remaining $pp$ elastic scattering events make a locus around the range of $0.1<\costcm<0.8$ and $0<\Delta E<30$ MeV. The $\Lp$ scattering events make a locus around the range of $-0.6<\costcm<0.5$ and $|\Delta E|<10$ MeV. The $\Ldecay$ decay events are mainly distributed in the backward angular range ($\costcm<-0.8$). }

Also, the $y$ projection of the correlation above (i.e., $\Delta E$) in each $\Kz$ scattering angle in the CM frame ($\costkz$) were shown in Figure \ref{fig-dEeach_sim}. 
Each colored solid line represents the same reaction as other simulation results.
\textcolor{red}{ In the range of $-1.0<\costkz<-0.7$, the contamination of $\Ldecay$ decay events remains. In the range of $0.5<\costkz<0.8$, the contamination of $pp$ elastic scattering becomes significant. In the range of $0.7<\costkz<1.0$, the $\Lp$ scattering peak shifts and is no longer located in the center. This is because the energy correction for proton and $\pM$ becomes difficult due to the scattering angle being small. }

From this, a cut for $\costcm$ was made at the range of $-0.6<\costcm<0.5$, as represented by the red solid lines in Figure \ref{fig-costdE}. This is cut level {\bf C-2p-3}.

%LpのcostdE相関(sim)
\begin{figure}[!h]
  \begin{center}
    \includegraphics[width=15cm]{costdE_wo_sim.eps}
    \caption{Correlation between the scattering angle of $\Lambda'$ in the CM frame ($\costcm$) versus $\Delta E$ assuming recoil proton in case \rom{2} without $\pM$ mode under {\bf C-2p-1, C-2p-2}, taken in simulation data. The upper left shows the sum of all reactions, the upper right shows the $pp$ elastic scattering, the lower left shows the $\Lp$ scattering, and the lower right shows the results of $\Ldecay$ decay scaled to 1/100.}
    \label{fig-costdE_sim}
  \end{center}
\end{figure}

%LpのdE(sim) 各K0角度毎
\begin{figure}[!h]
  \begin{center}
    \includegraphics[width=15cm]{dEeach_wo_sim.eps}
    \caption{$\Delta E$ assuming recoil proton in each $\Kz$ scattering angle in the CM frame ($\costkz$) in case \rom{2} without $\pM$ mode under {\bf C-2p-1, C-2p-2}, taken in simulation data. Each colored solid line represents the same reaction as other simulation results.}
    \label{fig-dEeach_sim}
  \end{center}
\end{figure}

%LpのcostdE相関(e40)
\begin{figure}[!h]
  \begin{center}
    \includegraphics[width=12cm]{costdE_wo.eps}
    \caption{Correlation plot between the scattering angle of $\Lambda'$ in the CM frame ($\costcm$) versus $\Delta E$ assuming recoil proton in case \rom{2} without $\pM$ mode under {\bf C-2p-1, C-2p-2}, taken in E40 data. A cut was made at $-0.6<\costcm<0.5$, as represented by the red solid lines. This is cut level {\bf C-2p-3}.}
    \label{fig-costdE}
  \end{center}
\end{figure}


\begin{comment}
%%%%%%%%%%%%%%%%%%%%%%%%%%%%%%%%%%%%%%%%%%%%%%
\textcolor{red}{2p解析ではBG除去を行わない方針になったので、この節は省く。1p1pi, Pl解析におけるBG除去をどうするか決まったら、ここの記述を参考に各チャプターに載せる。}
\clearpage
%%%%
\subsection{Background subtraction}
\label{sec-bgsub_2p_woPi}
The remaining background contamination was estimated by normalizing the sum of events on the low/high sidebands. The normalizing constant ($x_{nor\ 2p}$) can be calculated by measuring the background structure of the missing mass spectrum \textcolor{red}{with cut levels ****} as 
\begin{equation} 
  x_{nor\ 2p} = \frac{N_{\Lambda\ bg}}{N_{low} + N_{high}},
  \label{eq-norcon}
\end{equation}
where $N_{\Lambda\ bg}$ is the yield of background under $\Lambda$ peak in the missing mass, and $N_{low} (N_{high})$ is the yield of low (high) sideband in the missing mass. Each component was measured by fitting the missing mass as shown in Figure \ref{fig-MM_2p_fit}. The fitting range was set to be $1.0 - 1.17$ GeV/$c^{2}$, not to include the $\Sigma^{0}$ peak. The red solid line represents the fitting function for $\Lambda$ peak and background, and the blue shade does the estimated background structure. $N_{\Lambda\ bg}$, $N_{low}$, and $N_{high}$ were measured by integrating the missing mass range of $1.07 - 1.15$ GeV/$c^{2}$, $1.0 - 1.07$ GeV/$c^{2}$, and $1.15 - 1.17$ GeV/$c^{2}$ for each. The fitting results were summarized in Table \ref{Table-MM_2p_fit}. \textcolor{red}{This fitting estimated the normalizing constant as $x_{nor\ 2p} = **$.} 

%2pのMM fitting
\begin{figure}[!h]
  \begin{center}
    \includegraphics[clip,width=14cm]{MM_2p_fit.png}
    \caption{Fitting result of the missing mass of $\PiKX$ reaction in case (1) without $\pM$ mode. The fitting range was set to be $1.0 - 1.17$ GeV/$c^{2}$, not to include the $\Sigma^{0}$ peak. The red solid line represents the fitting function for $\Lambda$ peak and background, and the blue shade does the estimated background structure.}
    \label{fig-MM_2p_fit}
  \end{center}
\end{figure}

\textcolor{red}{fitting結果を表にまとめて。}


The yield of the realistic background of the $\Delta E$ on the $\Lambda$ range of the missing mass can be calculated by the yield of $\Delta E$ on the low/high sidebands of the missing mass ($N_{\Delta E\ low}$ and $N_{\Delta E\ high}$) as
\begin{align}
  N_{\Delta E\ bg,\ est} &= x_{nor\ 2p} (N_{\Delta E\ low} + N_{\Delta E\ high}), \\
  \sigma(N_{\Delta E\ bg,\ est}) &= \sqrt{N_{\Delta E\ low} + N_{\Delta E\ high}},
  \label{eq-estbgdE_2p}
\end{align}
where $\sigma(N_{\Delta E\ bg,\ est})$ is the statistical error of $N_{\Delta E\ bg,\ est}$.

Then the background subtraction for the $\Delta E$ of $\Lp$ scattering was performed. The corrected yield of $\Delta E$ ($N_{\Delta E\ \Lambda,\ cor}$) can be calculated by the estimated yield of background ($N_{\Delta E\ bg,\ est}$) as 
\begin{align}
  N_{\Delta E\ \Lambda,\ cor} &= N_{\Delta E\ \Lambda} - N_{\Delta E\ bg,\ est}, \\
  \sigma(N_{\Delta E\ \Lambda,\ cor}) &= \sqrt{N_{\Delta E\ \Lambda} + \sigma(N_{\Delta E\ bg,\ est})^{2}}, \\
  							 &= \sqrt{N_{\Delta E\ \Lambda} + N_{\Delta E\ low} + N_{\Delta E\ high}},
  \label{eq-cordE_2p}
\end{align}
where $N_{\Delta E\ \Lambda}$ is the yield of original $\Delta E$ on the $\Lambda$ range of the missing mass. Figure \ref{fig-dE_2p_woPi_bgsub} (Left) shows the $\Delta E$ of the events on $\Lambda$ region of the missing mass before the subtraction with estimated background structure. The green solid line represents the events on the $\Lambda$ region of the missing mass, and the gray shade does the estimated background structure (i.e. $N_{\Delta E\ bg,\ est}$). Figure \ref{fig-dE_2p_woPi_cor} (Right) shows the $\Delta E$ on the $\Lambda$ range of the missing mass, which was corrected by the background subtraction (i.e. $N_{\Delta E\ \Lambda,\ cor}$). \textcolor{red}{This is cut level **.}

%補正前後のLpのdE
\begin{figure}[!h]
  \begin{minipage}[t]{0.48\columnwidth}
    \centering
    \includegraphics[clip,width=\columnwidth]{dE_2p_woPi_bgsub.png}
    \caption{$\Delta E$ of the events on $\Lambda$ region of the missing mass before the subtraction with estimated background structure. The green solid line represents the events on the $\Lambda$ region of the missing mass, and the gray shade does the estimated background structure (i.e. $N_{\Delta E\ bg,\ est}$). \textcolor{red}{予想BGはエラーバー付きで描く}}
    \label{fig-dE_2p_woPi_bgsub}
  \end{minipage}
  \hspace{0.04\columnwidth} % ここで隙間作成
  \begin{minipage}[t]{0.48\columnwidth}
    \centering
    \includegraphics[clip, width=\columnwidth]{dE_2p_woPi_cor.png}
    \caption{$\Delta E$ on the $\Lambda$ range of the missing mass, which was corrected by the background subtraction (i.e. &N_{\Delta E\ \Lambda,\ cor}$). \textcolor{red}{補正されたΔEはエラーバー付きで描く}}
    \label{fig-dE_2p_woPi_cor}
  \end{minipage}
\end{figure}
%%%%%%%%%%%%%%%%%%%%%%%%%%%%%%%%%%%%%%%%%%%%%%
\end{comment}




%この段階で残っているMMをfittingし、Lp収量の試算を示す。
\clearpage
%%%%
\subsection{Identification of $\Lp$ scattering events}
\label{sec-2p_wo_id}

After applying the \nth{1} level cuts and selecting specific scattering angle region ({\bf C-2p-3}), the identification of $\Lp$ scattering in the \say{without $\pM$} mode of detection case \rom{2} was performed in two steps, as follows. 

\begin{enumerate}
  \item {\bf Missing mass of the $\PiKX$ reaction ({\bf C-2p-4}) } \\
  The yield of $\Lp$ scattering was roughly estimated by the missing mass of $\PiKX$ reaction.
  \item {\bf $\Delta E$ assuming recoil proton ({\bf C-2p-5}) } \\
  Precise estimation of $\Lp$ scattering yield was performed by $\Delta E$.
\end{enumerate}

%%
\subsubsection{Missing mass of the $\PiKX$ reaction ({\bf C-2p-4})}

\textcolor{red}{ The yield and ratio of $\Lp$ scattering, $pp$ elastic scattering, and $\Ldecay$ decay events remaining after applying the cut levels {\bf C-2p-1} to {\bf C-2p-3} can be estimated from the missing mass of the $\PiKX$ reaction. Figure \ref{fig-mm_wopi_sim} shows the simulation result. Here, by counting each bin content in the $\Lambda$ region (the $1.071-1.163$ GeV/$c^{2}$ range defined in Figure \ref{fig-mm_lam}), we found that the numbers of $\Lp$ scattering, $pp$ elastic scattering, and $\Ldecay$ decay events were 26169, 2527, and 121. In other words, when the ratio of $\Lp$ scattering is 1, the ratio of each reaction is 0.0966 for $pp$ elastic scattering and 0.00462 for $\Ldecay$ decay. The S/N at this time was 9.883. }

\textcolor{red}{ The same spectrum taken in E40 data is shown in Figure \ref{fig-mm_wopi}. The event number within the $\Lambda$ region (red solid lines) was approximately 59. Considering that the S/N estimated by the simulation was 9.883, the yield of $\Lp$ scattering events included here was expected at most 53. A cut to make S/N better was made at this $\Lambda$ region. This is cut level {\bf C-2p-4}.}

%%
\subsubsection{$\Delta E$ assuming recoil proton ({\bf C-2p-5})}

\textcolor{red}{ To more precisely estimate the yield of $\Lp$ scattering events, we analyzed the $\Delta E$ spectrum of events in which the missing mass is distributed in the $\Lambda$ region.
This $\Delta E$ is the same component as the one explained in Sec. \ref{sec-2p_wo_kine}.
The simulation result of $\Delta E$ of events in which the missing mass is distributed in the $\Lambda$ region is shown in Figure \ref{fig-dE_wopi_sim}. Here, the yields of $\Lp$ scattering, $pp$ elastic scattering, and $\Ldecay$ decay in the $|\Delta E|<20$ MeV region were calculated to be 24080, 1981, and 81, respectively. In other words, when the ratio of $\Lp$ scattering is 1, $pp$ elastic scattering is 0.0823 and $\Ldecay$ decay is 0.00336. The S/N at this time was 11.678. }

\textcolor{red}{ The same spectrum taken in E40 data is shown in Figure \ref{fig-dE_wopi}. The event number within the $|\Delta E|<20$ MeV region (red solid lines) was approximately 43. Considering that the S/N estimated by the simulation was 11.678, the yield of $\Lp$ scattering events included here was expected at most 39. Requiring $|\Delta E|<20$ MeV is the final cut level {\bf C-2p-5}.}

%cut1-3後のmm
\begin{figure}[!h]
  \begin{center}
    \includegraphics[width=12cm]{mm_wopi_sim.eps}
    \caption{Missing mass of the $\PiKX$ reaction under {\bf C-2p-1} to {\bf C-2p-3}, taken in simulation data. Each colored solid line represents the same reaction as other simulation results. The estimated S/N was 9.883.}
    \label{fig-mm_wopi_sim}
  \end{center}
\end{figure}

\begin{figure}[!h]
  \begin{center}
    \includegraphics[width=12cm]{mm_wopi.eps}
    \caption{Missing mass of the $\PiKX$ reaction under {\bf C-2p-1} to {\bf C-2p-3}, taken in E40 data. The red solid lines represent the $\Lambda$ region ($1.0707-1.1626$ GeV/$c^{2}$). The event number within the $\Lambda$ region was approximately 59. Considering that the S/N estimated by the simulation was 9.883, the yield of $\Lp$ scattering events included here was expected at most 53. A cut to make S/N better was made at this $\Lambda$ region ({\bf C-2p-4}).}
    \label{fig-mm_wopi}
  \end{center}
\end{figure}

%cut1-4後のdE
\begin{figure}[!h]
  \begin{center}
    \includegraphics[width=12cm]{dE_wopi_sim.eps}
    \caption{$\Delta E$ of events where the missing mass is distributed in the $\Lambda$ region under {\bf C-2p-1} to {\bf C-2p-4}, taken in simulation data. Each colored solid line represents the same reaction as other simulation results. The estimated S/N was 11.678.}
    \label{fig-dE_wopi_sim}
  \end{center}
\end{figure}

\begin{figure}[!h]
  \begin{center}
    \includegraphics[width=12cm]{dE_wopi.eps}
    \caption{$\Delta E$ of events where the missing mass is distributed in the $\Lambda$ region under {\bf C-2p-1} to {\bf C-2p-4}, taken in E40 data. The red solid lines represent the final cut. The event number within the $|\Delta E|<20$ MeV region was approximately 43. Considering that the S/N estimated by the simulation was 11.678, the yield of $\Lp$ scattering events included here was expected at most 39. Requiring $|\Delta E|<20$ MeV is the final cut level {\bf C-2p-5}.}
    \label{fig-dE_wopi}
  \end{center}
\end{figure}




\begin{comment}

\begin{figure}[!h]
  \begin{minipage}[t]{0.48\columnwidth}
    \centering
    \includegraphics[width=\columnwidth]{mm_wopi_sim.png}
    \caption{Missing mass of the $\PiKX$ reaction under {\bf C-2p-1} to {\bf C-2p-3}, taken in simulation data. Each colored solid line represents the same reaction as other simulation results. The estimated S/N was 9.883.}
    \label{fig-mm_wopi_sim}
  \end{minipage}
  \hspace{0.04\columnwidth} % ここで隙間作成
  \begin{minipage}[t]{0.48\columnwidth}
    \centering
    \includegraphics[width=\columnwidth]{mm_wopi.png}
    \caption{Missing mass of the $\PiKX$ reaction under {\bf C-2p-1} to {\bf C-2p-3}, taken in E40 data. The red solid lines represent the $\Lambda$ region ($1.0707-1.1626$ GeV/$c^{2}$). The event number within the $\Lambda$ region was approximately 59. Considering that the S/N estimated by the simulation was 9.883, the yield of $\Lp$ scattering events included here was expected at most 53. A cut to make S/N better was made at this $\Lambda$ region ({\bf C-2p-4}).}
    \label{fig-mm_wopi}
  \end{minipage}
\end{figure}

%cut1-4後のdE
\begin{figure}[!h]
  \begin{minipage}[t]{0.48\columnwidth}
    \centering
    \includegraphics[width=\columnwidth]{dE_wopi_sim.png}
    \caption{$\Delta E$ of events where the missing mass is distributed in the $\Lambda$ region under {\bf C-2p-1} to {\bf C-2p-4}, taken in simulation data. Each colored solid line represents the same reaction as other simulation results. The estimated S/N was 11.678.}
    \label{fig-dE_wopi_sim}
  \end{minipage}
  \hspace{0.04\columnwidth} % ここで隙間作成
  \begin{minipage}[t]{0.48\columnwidth}
    \centering
    \includegraphics[width=\columnwidth]{dE_wopi.png}
    \caption{$\Delta E$ of events where the missing mass is distributed in the $\Lambda$ region under {\bf C-2p-1} to {\bf C-2p-4}, taken in E40 data. The red solid lines represent the final cut. The event number within the $|\Delta E|<20$ MeV region was approximately 43. Considering that the S/N estimated by the simulation was 11.678, the yield of $\Lp$ scattering events included here was expected at most 39. Requiring $|\Delta E|<20$ MeV is the final cut level {\bf C-2p-5}.}
    \label{fig-dE_wopi}
  \end{minipage}
\end{figure}

\end{comment}


%%========================with pi mode ここから===================================%%
\clearpage
%%%%%
\section{With $\pM$ mode}
\label{sec-Lp_2p_wPi}

After applying the \nth{1} level cut in Sec. \ref{sec-Lp_common}, kinematical consistency analyses were performed to investigate the specific scattering angle range suitable for $\Lp$ scattering event identification. %As a result, it was found that most of the $\Lp$ scattering events are distributed in the scattering angle region of $-0.6<\costcm<0.5$.
After selecting the scattering angle region, we estimated the number and identification accuracy of $\Lp$ scattering events using two indices ($\Delta E$ and $\Delta p$) determined by kinematical consistency analyses.

The analysis flow is the almost same as the \say{without $\pM$} mode in Sec. \ref{sec-Lp_2p_woPi}. However, this \say{with $\pM$} mode has better accuracy for the $\Lp$ scattering identification since two kinematical consistency analyses, which assume not only recoil proton but $\Lambda'$, are available. 



%%%%
\subsection{Kinematical consistency analyses and selection of the scattering angle range}
\label{sec-2p_w_kine}

\textcolor{red}{ In this section, $\pM$s determined not to be a decay product of $\Kz$ are regarded as derived from the $\scatldecay$ decay. 
The $\pM$ momentum was recalculated so that the invariant mass of the detected proton and $\pM$ agrees with the mass of $\Lambda$. This is because if the energy of $\pM$ is high, it will penetrate the BGO calorimeter, and the momentum cannot be measured correctly.  }
Here, we assumed the kinematic of the $\scatldecay$ decay. The calculated $\pM$ momentum ($p_{\pM\ calc}$) can be obtained as
\begin{align}
  p_{\pM\ calc} &= \frac{A p_{p} \cos{\theta_{p\pM}} + \sqrt{B}}{E_{p}^{2} - p_{p}^{2} \cos^{2}{\theta_{p\pM}}}, \\
  \nonumber \\
  A &= \frac{m_{\Lambda'}^{2} - (m_{p}^{2} + m_{\pM}^{2})}{2}, \\
  B &= (A p_{p} \cos{\theta_{p\pM}})^{2} - (E_{p}^{2} - p_{p}^{2} \cos^{2}{\theta_{p\pM}}) (E_{p}^{2} m_{\pM}^{2} - A^{2}), \\
  E_{p} &= \sqrt{p_{p}^{2} + m_{p}^{2}},
  \label{eq-pMmomrecal}
\end{align}
where $p_{p}$, $E_{p}$ are the momentum and energy of decay proton, $\theta_{p\pM}$ is the opening angle between decay proton and $\pM$, and $m_{\Lambda'}$, $m_{p}$, and $m_{\pM}$ are the masses of $\Lambda'$, decay proton, and decay $\pM$.


\textcolor{red}{ Then, we performed two kinematical consistency analyses: The $\Delta E$ method and the $\Delta p$ method. The former is the same analysis as in Sec. \ref{sec-2p_wo_kine}. The latter identifies $\Lp$ scattering by calculating the difference between the $\Lambda'$ momentum measured by proton and $\pM$ ($p_{meas}$) and the one calculated kinematically assuming the $\Lp$ scattering ($p_{calc}$). If $\Lp$ scattering occurs, the difference $\Delta p$ ideally should be 0 ($\Delta p = p_{meas} - p_{calc} = 0$). }

The momentum of $\Lambda'$ can be obtained as
\begin{align}
  p_{cal} &= \frac{A p_{\Lambda} \cos{\theta_{\Lambda\Lambda'}} + (E_{\Lambda} + m_{p}) \sqrt{B}} {2 ((E_{\Lambda} + m_{p})^{2} + p_{\Lambda}^{2} \cos{\theta_{\Lambda\Lambda'}})}, \\
  \nonumber \\
  A &= m_{\Lambda}^{2} + m_{p}^{2} + m_{\Lambda'}^{2} - m_{p'} + 2 E_{\Lambda} m_{p}, \\
  B &= 4 m_{\Lambda'}^{2} (p_{\Lambda}^{2} \cos{\theta_{\Lambda\Lambda'}} - (E_{\Lambda} + m_{p})^{2}) + A^{2},
  \label{eq-dp_scatL}
\end{align}
where $p_{\Lambda}$ and $E_{\Lambda}$ are momentum and energy of beam $\Lambda$, $\theta_{\Lambda\Lambda'}$ is the scattering angle of $\Lambda'$, and $m_{p}$, $m_{\Lambda}$, and $m_{\Lambda'}$ are masses of target proton, beam $\Lambda$, and $\Lambda'$.

\textcolor{red}{ When the $\Delta p$ analysis was applied to the $pp$ elastic scattering, $\Lp$ scattering, and $\Ldecay$ decay events generated in the Monte Carlo simulation, the correlation between the scattering angle of $\Lambda'$ in the CM frame ($\costcm$) versus $\Delta p$ could be like Figure \ref{fig-costdp_w_sim}. 
The upper left indicates the sum of all reactions, the upper right $pp$ elastic scattering, the lower left $\Lp$ scattering, and the lower right $\Ldecay$ decay scaled to 1/100.
The remaining $pp$ elastic scattering events are relatively less but make a locus around the range of $0.2<\costcm<0.8$ and $0<\Delta p<0.2$ GeV/$c$. The $\Lp$ scattering events make a locus around the range of $-0.6<\costcm<0.6$ and $|\Delta p|<0.05$ GeV/$c$. The $\Ldecay$ decay events are distributed in the backward angular range ($\costcm<-0.8$). }

\textcolor{red}{ Also, the $y$ projection of the correlation above (i.e., $\Delta p$) in each $\Kz$ scattering angle in the CM frame ($\costkz$) were shown in Figure \ref{fig-dpeach_w_sim}. 
Each colored solid line represents the same reaction as other simulation results.
In the range of $\costkz<-0.8$, the contamination of the $\Ldecay$ decay events remains. The contamination of $pp$ elastic scattering becomes significant only in the range of $0.5<\costkz$. In the range of $0.7<\costkz$, the peak structure of $\Lp$ scattering events distorts. }

\textcolor{red}{ The $\Delta E$ was also calculated in the same way in Sec. \ref{sec-2p_wo_kine}. The simulation results of each reaction are shown in Figure \ref{fig-costdE_w_sim}. The remaining $pp$ elastic scattering events make a small locus around the range of $0.4<\costcm<0.8$ and $10<\Delta E<30$ MeV. The $\Lp$ scattering events make a locus around the range of $-0.6<\costcm<0.5$ and $|\Delta E|<10$ MeV. The $\Ldecay$ decay events are distributed in the backward angular range ($\costcm<-0.8$). }

The $y$ projection of the correlation above (i.e., $\Delta E$) in each $\Kz$ scattering angle in the CM frame ($\costkz$) were shown in Figure \ref{fig-dEeach_w_sim}. Each colored solid line represents the same reaction as other simulation results.
In the range of $\costkz<-0.8$, the contamination of the $\Ldecay$ decay events remains. The contamination of $pp$ elastic scattering appears only in the range of $0.5<\costkz$. In the range of $0.6<\costkz$, the peak structure of $\Lp$ scattering events distorts and shifts. In such an angular region, the reconstructed $\Kz$ momentum is estimated to be smaller than the real. Then, the beam $\Lambda$ momentum is estimated to be larger than real. As a result, $\Delta E$ shifts to the right. 

\textcolor{red}{ From this, a cut for $\costcm$ was made at the range of $-0.6<\costcm<0.5$, as represented by the red solid lines in Figure \ref{fig-costdp_w} and Figure \ref{fig-costdE_w}. This is cut level {\bf C-2p-6}. }


%Lpのcostdp相関(sim)
\begin{figure}[!h]
  \begin{center}
    \includegraphics[width=15cm]{costdp_w_sim.eps}
    \caption{Correlation between the scattering angle of $\Lambda'$ in the CM frame ($\costcm$) versus $\Delta p$ assuming $\Lambda'$ in case \rom{2} with $\pM$ mode under {\bf C-2p-1, C-2p-2}, taken in simulation data. The upper left shows the sum of all reactions, the upper right shows the $pp$ elastic scattering, the lower left shows the $\Lp$ scattering, and the lower right shows the results of $\Ldecay$ decay scaled to 1/100.}
    \label{fig-costdp_w_sim}
  \end{center}
\end{figure}

%Lpのdp(sim) 各K0角度毎
\begin{figure}[!h]
  \begin{center}
    \includegraphics[width=15cm]{dpeach_w_sim.eps}
    \caption{$\Delta p$ assuming $\Lambda'$ in each $\Kz$ scattering angle in the CM frame ($\costkz$) in case \rom{2} without $\pM$ mode under {\bf C-2p-1, C-2p-2}, taken in E40 data. Each colored solid line represents the same reaction as other simulation results.}
    \label{fig-dpeach_w_sim}
  \end{center}
\end{figure}

%LpのcostdE相関(sim)
\begin{figure}[!h]
  \begin{center}
    \includegraphics[width=15cm]{costdE_w_sim.eps}
    \caption{Correlation between the scattering angle of $\Lambda'$ in the CM frame ($\costcm$) versus $\Delta E$ assuming recoil proton in case \rom{2} with $\pM$ mode under {\bf C-2p-1, C-2p-2}, taken in simulation data. The upper left shows the sum of all reactions, the upper right shows the $pp$ elastic scattering, the lower left shows the $\Lp$ scattering, and the lower right shows the results of $\Ldecay$ decay scaled to 1/100.}
    \label{fig-costdE_w_sim}
  \end{center}
\end{figure}

%LpのdE(sim) 各K0角度毎
\begin{figure}[!h]
  \begin{center}
    \includegraphics[width=15cm]{dEeach_w_sim.eps}
    \caption{$\Delta E$ assuming recoil proton in each $\Kz$ scattering angle in the CM frame ($\costkz$) in case \rom{2} with $\pM$ mode under {\bf C-2p-1, C-2p-2}, taken in simulation data. Each colored solid line represents the same reaction as other simulation results.}
    \label{fig-dEeach_w_sim}
  \end{center}
\end{figure}


%Lpのcostdp & costdE相関(e40)
\begin{figure}[!h]
  \begin{center}
    \includegraphics[width=12cm]{costdp_w.eps}
    \caption{Correlation plot between the scattering angle of $\Lambda'$ in the CM frame ($\costcm$) versus $\Delta p$ assuming recoil proton in case \rom{2} with $\pM$ mode under {\bf C-2p-1, C-2p-2}, taken in E40 data. The red solid line represents the cut level {\bf C-2p-6}.}
    \label{fig-costdp_w}
  \end{center}
\end{figure}

\begin{figure}[!h]
  \begin{center}
    \includegraphics[width=12cm]{costdE_w.eps}
    \caption{Correlation plot between the scattering angle of $\Lambda'$ in the CM frame ($\costcm$) versus $\Delta E$ assuming recoil proton in case \rom{2} with $\pM$ mode under {\bf C-2p-1, C-2p-2}, taken in E40 data. The red solid line represents the cut level {\bf C-2p-6}.}
    \label{fig-costdE_w}
  \end{center}
\end{figure}

\begin{comment}

\begin{figure}[!h]
  \begin{minipage}[t]{0.48\columnwidth}
    \centering
    \includegraphics[width=\columnwidth]{costdp_w.eps}
    \caption{Correlation plot between the scattering angle of $\Lambda'$ in the CM frame ($\costcm$) versus $\Delta p$ assuming recoil proton in case \rom{2} with $\pM$ mode under {\bf C-2p-1, C-2p-2}, taken in E40 data. The red solid line represents the cut level {\bf C-2p-6}.}
    \label{fig-costdp_w}
  \end{minipage}
  \hspace{0.04\columnwidth} % ここで隙間作成
  \begin{minipage}[t]{0.48\columnwidth}
    \centering
    \includegraphics[width=\columnwidth]{costdE_w.eps}
    \caption{Correlation plot between the scattering angle of $\Lambda'$ in the CM frame ($\costcm$) versus $\Delta E$ assuming recoil proton in case \rom{2} with $\pM$ mode under {\bf C-2p-1, C-2p-2}, taken in E40 data. The red solid line represents the cut level {\bf C-2p-6}.}
    \label{fig-costdE_w}
  \end{minipage}
\end{figure}

\end{comment}

%この段階で残っているMMをfittingし、Lp収量の試算を示す。
\clearpage
%%%%
\subsection{Identification of $\Lp$ scattering events}
\label{sec-2p_w_id}

After applying the \nth{1} level cuts and selecting specific scattering angle region ({\bf C-2p-6}), the identification of $\Lp$ scattering in the \say{with $\pM$} mode of detection case \rom{2} was performed in two steps, as follows. 

\begin{enumerate}
  \item {\bf Missing mass of the $\PiKX$ reaction ({\bf C-2p-7}) } \\
  The yield of $\Lp$ scattering was roughly estimated by the missing mass of $\PiKX$ reaction.
  \item {\bf Correlation between $\Delta E$ assuming recoil proton versus $\Delta p$ assuming $\Lambda'$ ({\bf C-2p-8}) } \\
  Precise estimation of $\Lp$ scattering yield was performed by the correlation between $\Delta E$ versus $\Delta p$.
\end{enumerate}


%%%
\subsubsection{Missing mass of the $\PiKX$ reaction ({\bf C-2p-7})}

The yield and ratio of $\Lp$ scattering, $pp$ elastic scattering, and $\Ldecay$ decay events remaining after applying the cut levels {\bf C-2p-1, C-2p-2, C-2p-6} can be estimated from the missing mass of the $\PiKX$ reaction. Figure \ref{mm_wpi_sim} shows the simulation result. Here, by counting each bin content in the $\Lambda$ region (the $1.071-1.163$ GeV/$c^{2}$ range defined in Figure \ref{fig-mm_lam}), we found that the numbers of $\Lp$ scattering, $pp$ elastic scattering, and $\Ldecay$ decay events were 16328, 1315, and 37. In other words, when the ratio of $\Lp$ scattering is 1, $pp$ elastic scattering is 0.0805 and $\Ldecay$ decay is 0.00227. The S/N at this time was 12.077. 

The same spectrum taken in E40 data is shown in Figure \ref{fig-mm_wpi}. The event number within the $\Lambda$ region (red solid lines) was approximately 21. Considering that the S/N estimated by the simulation was 12.077, the yield of $\Lp$ scattering events included here was expected at most 19. A cut to make S/N better was made at this $\Lambda$ region. This is cut level {\bf C-2p-7}.

\subsubsection{Correlation between $\Delta E$ versus $\Delta p$ ({\bf C-2p-8})}

To more precisely estimate the yield of $\Lp$ scattering events, we analyzed the correlation between the $\Delta E$ and $\Delta p$ of events in which the missing mass is distributed in the $\Lambda$ region.
These $\Delta E$ and $\Delta p$ are the same components as the one explained in Sec. \ref{sec-2p_w_kine}.
The simulation result of the correlation is shown in Figure \ref{fig-dEdp_wpi_sim}. Here, the yields of $\Lp$ scattering, $pp$ elastic scattering, and $\Ldecay$ decay in the range of $|\Delta E|<20$ MeV and $|\Delta p|<0.05$ GeV/$c$ were calculated to be 13376, 730, and 14, respectively. In other words, when the ratio of $\Lp$ scattering is 1, $pp$ elastic scattering is 0.0546 and $\Ldecay$ decay is 0.00105. The S/N at this time was 17.979.

The same plot taken in E40 data is shown in Figure \ref{fig-dEdp_wpi}. The event number within the range of $|\Delta E|<20$ MeV and $|\Delta p|<0.05$ GeV/$c$ (red solid lines) was approximately 15. Considering that the S/N estimated by the simulation was 17.979, the yield of $\Lp$ scattering events included here was expected at most 14. Requiring $|\Delta E|<20$ MeV and $|\Delta p|<0.05$ GeV/$c$ is the final cut level {\bf C-2p-8}.

Although it is expected that there are still unknown background events that the simulation cannot predict, it is considered that it is possible to identify $\Lp$ scattering events with a high S/N using the analysis method developed in this study. When we obtain high statistics in future J-PARC E86 experiments, we can rigorously fit $\Delta p$ and $\Delta E$ spectrums and estimate the yield of $\Lp$ scattering events. 

%cut1,2,6後のmm
\begin{figure}[!h]
  \begin{center}
    \includegraphics[width=12cm]{mm_wpi_sim.eps}
    \caption{Missing mass of the $\PiKX$ reaction under {\bf C-2p-1, C-2p-2, C-2p-6}, taken in simulation data. Each colored solid line represents the same reaction as other simulation results. The estimated S/N was 12.077.}
    \label{fig-mm_wopi_sim}
  \end{center}
\end{figure}

\begin{figure}[!h]
  \begin{center}
    \includegraphics[width=12cm]{mm_wpi.eps}
    \caption{Missing mass of the $\PiKX$ reaction under {\bf C-2p-1, C-2p-2, C-2p-6}, taken in E40 data. The red solid lines represent the $\Lambda$ region ($1.0707-1.1626$ GeV/$c^{2}$). The event number within the $\Lambda$ region was approximately 21. Considering that the S/N estimated by the simulation was 12.077, the yield of $\Lp$ scattering events included here was expected at most 19. A cut to make S/N better was made at this $\Lambda$ region ({\bf C-2p-7}).}
    \label{fig-mm_wopi}
  \end{center}
\end{figure}

%cut1,2,6,7後のdEdp
\begin{figure}[!h]
  \begin{center}
    \includegraphics[width=15cm]{dEdp_wpi_sim.eps}
    \caption{Correlation between $\Delta E$ versus $\Delta p$ of events where the missing mass is distributed in the $\Lambda$ region under {\bf C-2p-1, C-2p-2, C-2p-6, C-2p-7}, taken in simulation data. The upper left shows the sum of all reactions, the upper right shows the $pp$ elastic scattering, the lower left shows the $\Lp$ scattering, and the lower right shows the results of $\Ldecay$ decay scaled to 1/100. The estimated S/N was 17.979.}
    \label{fig-dEdp_wopi_sim}
  \end{center}
\end{figure}

\begin{figure}[!h]
  \begin{center}
    \includegraphics[width=12cm]{dEdp_wpi.eps}
    \caption{Correlation between $\Delta E$ versus $\Delta p$ of events where the missing mass is distributed in the $\Lambda$ region under {\bf C-2p-1, C-2p-2, C-2p-6, C-2p-7}, taken in E40 data. The red solid lines represent the final cut. The event number within the $|\Delta E|<20$ MeV and $|\Delta p|<0.05$ GeV/$c$ region was approximately 15. Considering that the S/N estimated by the simulation was 17.979, the yield of $\Lp$ scattering events included here was expected at most 14. Requiring $|\Delta E|<20$ MeV and $|\Delta p|<0.05$ GeV/$c$ is the final cut level {\bf C-2p-8}.}
    \label{fig-dEdp_wopi}
  \end{center}
\end{figure}


\begin{comment}

%cut1,2,6後のmm
\begin{figure}[!h]
  \begin{minipage}[t]{0.48\columnwidth}
    \centering
    \includegraphics[width=\columnwidth]{mm_wpi_sim.png}
    \caption{Simulation of the missing mass of the $\PiKX$ reaction after applying the cuts {\bf C-2p-1}, {\bf C-2p-2}, and {\bf C-2p-4} (with $\pM$ mode). The definition of each solid line is the same as other simulation results. The estimated S/N was 12.0769.}
    \label{fig-mm_wpi_sim}
  \end{minipage}
  \hspace{0.04\columnwidth} % ここで隙間作成
  \begin{minipage}[t]{0.48\columnwidth}
    \centering
    \includegraphics[width=\columnwidth]{mm_wpi.png}
    \caption{Missing mass of the $\PiKX$ reaction after applying the cuts {\bf C-2p-1}, {\bf C-2p-2}, and {\bf C-2p-4} (with $\pM$ mode), taken in E40 data. The dotted magenta lines represent the $\Lambda$ region ($1.0707-1.1626$ GeV/$c^{2}$). The number of bins within the $\Lambda$ region was approximately 21.}
    \label{fig-mm_wpi}
  \end{minipage}
\end{figure}

%cut1,2,4後のdE
\begin{figure}[!h]
  \centering
  \includegraphics[width=15cm]{dEdp_wpi_sim.png}
  \caption{Simulation of the correlation plot between $\Delta p$ and $\Delta E$ after applying the cuts {\bf C-2p-1}, {\bf C-2p-2}, and {\bf C-2p-4} using only events where the missing mass is distributed in the $\Lambda$ region (with $\pM$ mode). The definition of each solid line is the same as other simulation results. The estimated S/N was 17.9785.}
  \label{fig-dEdp_wpi_sim}
\end{figure}  
  
\begin{figure}
  \centering
  \includegraphics[width=12cm]{dEdp_wpi.png}
  \caption{Correlation plot between $\Delta p$ and $\Delta E$ after applying the cuts {\bf C-2p-1}, {\bf C-2p-2}, and {\bf C-2p-4} using only events where the missing mass is distributed in the $\Lambda$ region (with $\pM$ mode), taken in E40 data. The red solid lines represent the final cut. The number of bins within the final cut was approximately 15.}
  \label{fig-dEdp_wpi}
\end{figure}

\end{comment}



\clearpage
%%%%%
\section{Summary of $\Lp$ scattering identification}
\label{sec-2p_summary}

Using the $\PiKL$ reaction data taken in the J-PARC E40 experiment, we verified the $\Lp$ scattering identification requiring CATCH to detect two protons (detection case \rom{2}). In this detection case, the number of tagged $\Lambda$ was $2.72\times10^{3}$ with S/N$=1.47$ (see Sec. \ref{sec-beamLID} \textcolor{red}{(Λビーム同定の章)}). 

The analysis was divided into cases depending on whether or not $\pM$ derived from $\scatldecay$ decay was detected by CATCH. $\Lp$ scattering events were identified from kinematical consistency analysis ($\Delta p$ and $\Delta E$ methods) after applying cuts to remove the main background events (multiple $\pi$ production, $pp$ elastic scattering, and $\Ldecay$ decay).

Table \ref{tab-2p_summary} shows the numerical values of the yield of $\Lp$ scattering identified in E40 data and the S/N estimated by the Geant4 Monte Carlo simulation. Note that these results were obtained in the detection case \rom{2} with/without $\pM$ detection. In either mode, we demonstrated that the analysis method developed in this paper can identify $\Lp$ scattering events with high accuracy. Once we obtain more highly statistical data in the J-PARC E86 experiment, we can predict background structures and derive a cross-section more precisely.

%=====table=====%
\begin{table}[!tbph]
  \begin{center}
    \caption{The numerical values of the yield of $\Lp$ scattering identified in E40 data and the S/N estimated by the Geant4 Monte Carlo simulation. }
    \begin{tabular}{ccc}
      Mode & Yield of $\Lp$ scattering in E40 data& Estimated S/N \\ \hline \hline
      without $\pM$ & 39 & 11.678 \\ \hline
      with $\pM$ & 14 & 17.979 \\
    \end{tabular}
    \label{tab-2p_summary}
  \end{center}
\end{table}


%%
%2023/11/14
%%

%%%%%%%%%%%%
%%%%%%%%%%%%
%\end{document}
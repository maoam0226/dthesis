%\documentclass[a4paper,12pt,oneside,openany]{jsbook}
%%\setlength{\topmargin}{10mm}
%\addtolength{\topmargin}{-1in}
%\setlength{\oddsidemargin}{27mm}
%\addtolength{\oddsidemargin}{-1in}
%\setlength{\evensidemargin}{20mm}
%\addtolength{\evensidemargin}{-1in}
%\setlength{\textwidth}{160mm}
%\setlength{\textheight}{250mm}
%\setlength{\evensidemargin}{\oddsidemargin}

%\usepackage{ascmac}

\usepackage{color}
\usepackage{textcomp}
%\usepackage[dviout]{graphicx}
%\usepackage[dvipdfm]{graphicx,color}
\usepackage{wrapfig}
\usepackage{ccaption}
\usepackage{color}
%\usepackage{jumoline} %%行にまたがって下線を引ける、ダウンロードの必要有
\usepackage{umoline}
\usepackage{fancybox}
\usepackage{pifont}
\usepackage{indentfirst} %%最初の段落も字下げしてくれる

\usepackage{amsmath,amssymb,amsfonts}
\usepackage{bm}
%\usepackage{graphicx}
\usepackage[dvipdfmx]{graphicx}
%\usepackage[dvipsnames]{xcolor}
\usepackage{subfigure}
\usepackage{verbatim}
\usepackage{makeidx}
\usepackage{accents}
%\usepackage{slashbox} %%ダウンロードの必要有

\usepackage[dvipdfmx]{hyperref} %%pdfにリンクを貼る
\usepackage{pxjahyper}

\usepackage[flushleft]{threeparttable}
\usepackage{array,booktabs,makecell}

\usepackage{geometry}
\geometry{left=30mm,right=30mm,top=50mm,bottom=5mm}

\usepackage[super]{nth} %1st, 2nd ...を出力
\usepackage{dirtytalk} %クォーテーションマーク
\usepackage{amsmath} %行列が書ける
\usepackage{tikz} %\UTF{2460}などが書ける
\usepackage{cite} %複数の引用ができる

\usepackage[toc,page]{appendix}

%\graphicspath{{./pictures/}}

%\setlength{\textwidth}{\fullwidth}
\setlength{\textheight}{40\baselineskip}
\addtolength{\textheight}{\topskip}
\setlength{\voffset}{-0.55in}

\renewcommand{\baselinestretch}{1} %% 行間

%\setcounter{tocdepth}{5}  %% 目次section depth
\setcounter{secnumdepth}{5}
%\renewcommand{\bibname}{参考文献}

%%%%%%%%% accent.sty 設定 %%%%%%%%%
\makeatletter
  \def\widebar{\accentset{{\cc@style\underline{\mskip10mu}}}}
\makeatother

%%%%%%%%%  chapter 設定 %%%%%%%%%%%
%\makeatletter
%\def\@makechapterhead#1{%
%  \vspace*{1\Cvs}% 欧文は50pt 章上部の空白
%  {\parindent \z@ \raggedright \normalfont
%    \ifnum \c@secnumdepth >\m@ne
%      \if@mainmatter
%        \huge\headfont \@chapapp\thechapter\@chappos
%       \par\nobreak
%       \vskip \Cvs % 欧文は20pt
%         \hskip1zw
%      \fi
%    \fi
%    \interlinepenalty\@M
%    \centering \huge \headfont #1\par\nobreak
%    \vskip 3\Cvs}} % 欧文は40pt 章下部の空白
%\makeatother

%%%%%%%%%  chapter* 設定 %%%%%%%%%%%



%%%%%%%%%  chapter* 設定 %%%%%%%%%%%

%\makeatletter
%\def\@makeschapterhead#1{%
%  \vspace*{1\Cvs}
%  {\parindent \z@ \raggedright
%    \normalfont
%    \interlinepenalty\@M
%    \centering \huge \headfont #1\par\nobreak
%    \vskip 3\Cvs}}
%\makeatother

%%%%%%%%%  section 設定 %%%%%%%%%%%
\makeatletter
\renewcommand{\section}{%
  \@startsection{section}%
   {1}%
   {\z@}%
   {-3.5ex \@plus -1ex \@minus -.2ex}%
   {2.3ex \@plus.2ex}%
   {\normalfont\Large\bfseries}%
}%
\makeatother

%%%%%%%%%  subsection 設定 %%%%%%%%%%%
\makeatletter
\renewcommand{\subsection}{%
  \@startsection{subsection}%
   {2}%
   {\z@}%
   {-3.5ex \@plus -1ex \@minus -.2ex}%
   {2.3ex \@plus.2ex}%
   {\normalfont\large\bfseries}%
}%
\makeatother

%%%%%%%%%  subsubsection 設定 %%%%%%%%%%%
\makeatletter
\renewcommand{\subsubsection}{%
  \@startsection{subsubsection}%
   {3}%
   {\z@}%
   {-3.5ex \@plus -1ex \@minus -.2ex}%
   {2.3ex \@plus.2ex}%
   %{\normalfont\normalsize\bfseries$\blacksquare$}%
   {\normalfont\normalsize\bfseries}%
}%
\makeatother

%%%%%%%%%  paragraph 設定 %%%%%%%%%%%
\makeatletter
\renewcommand{\paragraph}{%
  \@startsection{paragraph}%
   {4}%
   {\z@}%
   {0.5\Cvs \@plus.5\Cdp \@minus.2\Cdp}
   {-1zw}
   {\normalfont\normalsize\bfseries $\blacklozenge$\ }%
  % {\normalfont\normalsize\bfseries $\Diamond$\ }%
}%
\makeatother

%%%%%%%%%  subparagraph 設定 %%%%%%%%%%%
\makeatletter
\renewcommand{\subparagraph}{%
  \@startsection{subparagraph}%
   {4}%
   {\z@}%
   {0.5\Cvs \@plus.5\Cdp \@minus.2\Cdp}
   {-1zw}
   {\normalfont\normalsize\bfseries $\Diamond$\ }%
}%
\makeatother

%%%%%%%%% caption 設定 %%%%%%%%%%%%
\makeatletter

\newcommand*\circled[1]{\tikz[baseline=(char.base)]{
            \node[shape=circle,draw,inner sep=2pt] (char) {#1};}}

\newcommand*{\rom}[1]{\expandafter\@slowromancap\romannumeral #1@}

\newcommand{\msolar}{M_\odot}

\newcommand{\anapow}{A_{y}(\theta)}
\newcommand{\depo}{D^{y}_{y}(\theta)}

\newcommand{\figcaption}[1]{\def\@captype{figure}\caption{#1}}
\newcommand{\tblcaption}[1]{\def\@captype{table}\caption{#1}}
\newcommand{\klpionn}{K_L \to \pi^0 \nu \overline{\nu}}
\newcommand{\kppipnn}{K^+ \to \pi^+ \nu \overline{\nu}}
\newcommand{\hfl}{{}_\Lambda^4\rm{H}}
\newcommand{\htl}{{}_\Lambda^3\rm{H}}
\newcommand{\hefl}{{}_\Lambda^4\rm{He}}
\newcommand{\hefil}{{}_\Lambda^5\rm{He}}
\newcommand{\lisl}{{}_\Lambda^7\rm{Li}}
\newcommand{\benl}{{}_\Lambda^9\rm{Be}}
\newcommand{\btl}{{}_\Lambda^{10}\rm{B}}
\newcommand{\bel}{{}_\Lambda^{11}\rm{B}}

\newcommand{\nfl}{{}_\Lambda^{15}\rm{N}}
\newcommand{\osl}{{}_\Lambda^{16}\rm{O}}
\newcommand{\ctl}{{}_\Lambda^{13}\rm{C}}
\newcommand{\pbtl}{{}_\Lambda^{208}\rm{Pb}}

\def\vector#1{\mbox{\boldmath$#1$}}
\newcommand{\Kpi}{(K^-,\pi^-)}
\newcommand{\piKz}{(\pi^-,K^0)}
\newcommand{\pPK}{(\pi^+,K^+)}
\newcommand{\pMK}{(\pi^-,K^+)}
\newcommand{\pPMK}{(\pi^{\pm},K^+)}

\newcommand{\eeK}{(e,e' K^+)}
\newcommand{\gK}{(\gamma + p \to \Lambda + K^+)}
\newcommand{\PiKL}{\pi^-  p \to K^0 \Lambda}
\newcommand{\multipi}{\pi^-  p \to \pi^-\pi^-\pi^+p}
\newcommand{\PiKX}{\pi^-  p \to K^0 X}
\newcommand{\PiKSM}{\pi^-  p \to K^+ \Sigma^-}
\newcommand{\pipKS}{\pi^{\pm}p \to K^+ \Sigma^{\pm}}
\newcommand{\pipKX}{\pi^{\pm}p \to K^+ X}
\newcommand{\pipLn}{\pi^- p \to \Lambda n}
\newcommand{\PiKS}{\pi^{-}p \to K^{0}\Sigma^{0}}

\newcommand{\kzdecay}{K^0 \to \pi^+ \pi^-}
\newcommand{\kzsd}{K^0_s \to \pi^+ \pi^-\ \rm{or}\ \pi^0 \pi^0}
\newcommand{\Ldecay}{\Lambda\to p\pM}
\newcommand{\scatldecay}{\Lambda'\to p\pM}




\newcommand{\triton}{{}^3\rm{H}}

\newcommand{\BB}{B_{8}B_{8}}
\newcommand{\SM}{\Sigma^{-}}
\newcommand{\SP}{\Sigma^{+}}
\newcommand{\Sz}{\Sigma^{0}}
\newcommand{\SMp}{\Sigma^{-}p}
\newcommand{\SMn}{\Sigma^{-}n}
\newcommand{\SPp}{\Sigma^{+}p}
\newcommand{\SPn}{\Sigma^{+}n}
\newcommand{\Sp}{\Sigma p}
\newcommand{\SPMp}{\Sigma^{\pm}p}
\newcommand{\SPM}{\Sigma^{\pm}}
\newcommand{\SPdecay}{\Sigma^+ \to \pi^0 p}
\newcommand{\SMdecay}{\Sigma^- \to \pi^- n}
\newcommand{\SMpLn}{\Sigma^- p \to \Lambda n}

\newcommand{\XM}{\Xi^{-}}
\newcommand{\Xz}{\Xi^{0}}

\newcommand{\pM}{\pi^{-}}
\newcommand{\pP}{\pi^{+}}
\newcommand{\pZ}{\pi^{0}}
\newcommand{\pPM}{\pi^{\pm}}
\newcommand{\KP}{K^{+}}
\newcommand{\KM}{K^{-}}
\newcommand{\Kz}{K^{0}}
\newcommand{\Lp}{\Lambda p}
\newcommand{\LpLX}{\Lambda p \to \Lambda X}

\newcommand{\LN}{\Lambda N}
\newcommand{\SN}{\Sigma N}
\newcommand{\LNtoSN}{\Lambda N\to\Sigma N}
\newcommand{\LS}{\Lambda - \Sigma}

%\newcommand{\dp}{\Delta p}
%\newcommand{\dE}{\Delta E}

\newcommand{\dcs}{d\sigma/d\Omega}
\newcommand{\fdcs}{\frac{d\sigma}{d\Omega}}
\newcommand{\dz}{\Delta z}
\newcommand{\dzkz}{\Delta z_{K^{0}}}


\newcommand{\bgct}{\beta\gamma c\tau}

\newcommand{\costp}{\cos{\theta_p}}
\newcommand{\costkz}{\cos{\theta_{K0,CM}}}
\newcommand{\costcm}{\cos{\theta}_{CM}}
\newcommand{\PL}{P_{\Lambda}}
\newcommand{\PLall}{P_{\Lambda,\ all}}
\newcommand{\PLsele}{P_{\Lambda,\ selected}}
\newcommand{\errPL}{\sigma(P_{\Lambda})}

\newcommand{\rud}{r_{ud}}
\newcommand{\errrud}{\sigma(\rud)}

\newcommand{\accPL}{\epsilon_{\PL}}
\newcommand{\erraccPL}{\sigma(\epsilon_{\PL})}

\newcommand{\PLscat}{P_{\Lambda'}}
\newcommand{\effPLw}{\epsilon_{\PL,\ w/}}
\newcommand{\erreffPLw}{\sigma(\epsilon_{\PL,\ w/})}
\newcommand{\effPLwo}{\epsilon_{\PL,\ w/o}}
\newcommand{\erreffPLwo}{\sigma(\epsilon_{\PL,\ w/o})}

\newcommand{\chisq}{\chi^{2}}

\newcommand{\centered}[1]{\begin{tabular}{l} #1 \end{tabular}}

\makeatother

\begin{document}

\graphicspath{{./pictures/chapter1/}}

%EN
\chapter{Introduction} 
\label{chap1}
The nuclear force is the interaction that binds protons and neutrons in all atomic nuclei. It is solidly repulsive at short range ($\leq0.5$ fm), and attractive at long range ($\geq1.5$ fm). This feature is crucial to stabilizing nuclei. The study of nuclear force has been performed experimentally and theoretically globally because it can lead to understanding the origin \textcolor{red}{and evolution} of matters. 

The meson exchange model \textcolor{red}{of the nuclear force}, constructed with abundant nucleon-nucleon ($NN$) scattering data, \textcolor{red}{explains} the long-range attraction of \textcolor{red}{the} nuclear force. Since $\pi$ \textcolor{red}{meson} is the lightest meson, it contributes to the long-distance region. $\sigma$ \textcolor{red}{meson} with a mass of $\sim520$ MeV and decay width of $\sim700$ MeV contributes attraction at intermediate distances. $\omega$(782) and $\rho$(770) vector mesons dominate the exchange in the short-distance region. In contrast, treating repulsive cores of nuclear force appearing in a short range at high energies is highly model-dependent and requires further study. 

They say six kinds of quarks with different flavors exist: up ($u$), down ($d$), charm ($c$), strange ($s$), top ($t$), and bottom ($b$), as shown in Figure \ref{fig-quarks}. Nucleons are the ground-state baryons, which consist of $u$ and $d$ quarks with spin 1/2. The key to understanding the mechanism of the short-range nuclear force is the interaction between baryons, including other flavors, since nuclear force should be described from the same viewpoint as the nuclear force when considering that they originate from strong interaction. 
\begin{figure}[h!]
  \begin{center}
  \includegraphics[clip,width=12cm]{quarks.png}
  \caption{Six kinds of quarks with different flavors. From the top left, up, down, charm, strange, top, and bottom quarks are listed.}
  \label{fig-quarks}
  \end{center}
\end{figure}

%%%%%
\section{Baryon-Baryon interaction in the SU(3) flavor symmetry}
\label{sec-BBint}

Since the masses of $u$, $d$, and $s$ quarks are lighter than the scale parameter of Quantum Chromo-Dynamics (QCD), the symmetry \textcolor{red}{between} them is expected to hold. Because the mass of the $s$ quark is relatively similar to $u$, $d$ quarks, the interactions between two baryons composed of these quarks \textcolor{red}{are expected to} be understood \textcolor{red}{under} the SU(3) flavor symmetry (SU(3)$_f$). To study united Baryon-Baryon interactions considering quark contributions, it is essential to expand the system into SU(3)$_f$, which is the spin composition of three quarks $u$, $d$, and $s$, \textcolor{red}{although it is said that this symmetry is slightly broken from the quark picture.}

The quark model, in which baryons are composed of three quarks, was proposed by M. Gell-Mann \cite{mgm-1956} \cite{mgm-1961}. %Gluons are colored and anticolored at the same time, and according to group theory, $3\times\overline{3}$ color combinations produce two kinds of multiplets: singlet and octet. 
Here, \say{color confinement} occurs, in which only hadrons, whose final synthesized color is colorless, exist as free particles. A colorless (\say{white}) state can be synthesized by combining color with its anticolor or by adding three different colors, as shown in Figure \ref{fig-colors}. Considering the quark-antiquark combination under the color SU(3) symmetry, the meson of $q\bar{q}$ and the baryon of $qqq$ contain a color singlet without a color charge \textcolor{red}{with the minimum quark configuration}.
\begin{figure}[h!]
  \begin{center}
  \includegraphics[clip,width=15cm]{colors.png}
  \caption{Three examples of colorless (white) states synthesized by combining color with its anticolor or by adding three different colors.}
  \label{fig-colors}
  \end{center}
\end{figure}

In the following, we discuss \say{baryon}, a hadron composed of $u$, $d$, and $s$ quarks. The wavefunction of the whole baryon can be written as
\begin{equation}
  \psi = \xi_{space} \cdot \zeta_{flavor} \cdot \chi_{spin} \cdot \phi_{color}.
\end{equation}
Since the three quarks making up baryons are all fermions, the Pauli principle must be satisfied. That is, it must be antisymmetric when any two quarks are exchanged.  The following convention can also express the spin composition of three quarks with spin $\frac{1}{2}$.
\begin{equation}
    {\bf \frac{1}{2}\otimes\frac{1}{2}\otimes\frac{1}{2}=(1\oplus0)\otimes\frac{1}{2}=\frac{3}{2}\otimes\frac{1}{2}}.
    \label{eq-baryon}
\end{equation}
From Equation (\ref{eq-baryon}), hadrons are classified into an octet of spin $\frac{1}{2}$ and a decuplet of spin $\frac{3}{2}$. The octet baryons composed of $u$, $d$, and $s$ quarks are $n$, $p$, $\Lambda$, $\SM$, $\Sz$, $\SP$, $\XM$, and $\Xz$. The latter six baryons include at least one $s$ quark and are called \say{Hyperons ($Y$)}. Figure \ref{fig-B8} shows the baryon octet in the SU(3)$_f$ with spin $\frac{1}{2}$. Horizontal axis $I_{3}$ denotes the $z$ component of isospin, and vertical axis $Y$ denotes hypercharge, defined by $Y = B + S$ ($B$ is baryon number and $S$ is strangeness). 
\begin{figure}[h]
  \begin{center}
  \includegraphics[clip,width=12cm]{B8.png}
  \caption{Baryon octet appearing in the SU(3)$_f$ symmetry with spin $\frac{1}{2}$. Horizontal axis $I_{3}$ denotes the $z$ component of isospin, and vertical axis $Y$ denotes hypercharge, defined by $Y = B + S$ using baryon number $B$ and strangeness $S$.}
  \label{fig-B8}
  \end{center}
\end{figure}

The Baryon-Baryon interactions in the baryon octet ($\BB$ interaction) in SU(3)$_f$ can be expressed as
\begin{equation}
  {\bf 8\otimes 8 = 27\oplus 10\oplus 10^*\oplus 8_{\rm{s}}\oplus 8_{\rm{a}}\oplus 1}. 
  \label{eq-BBint}
\end{equation}
Figure \ref{fig-class} shows the particle classification by $\BB$ composition. Each $\BB$ interaction is represented by a combination of linear couplings in six irreducible representations in Equation (\ref{eq-BBint}).

\begin{figure}[h]
 \begin{center}
   \includegraphics[clip,width=12cm]{class.png}
   \caption{Classification by baryon octuplet composition.}
   \label{fig-class}
 \end{center}
\end{figure}

\begin{figure}[h]
 \begin{center}
   \includegraphics[clip,width=15cm]{BB_Swave.png}
   \caption{Potential of $\BB$ interactions diagonal in the flavor-irreducible representation basis \cite{QCD-2019}.}
   \label{fig-BB_Swave}
 \end{center}
\end{figure}

Since baryons are fermions like quarks, symmetric and antisymmetric representations of flavors are allowed depending on the orbital angular momentum $L$ and spin $S$, denoted by $^{2S+1}(-1)^{L}$ such as singlet-even $(^1E)$, triplet-odd $(^3O)$, triplet-even $(^3E)$, and singlet-odd $(^1O)$, as shown in Table \ref{table-BBint}. %In this table, spin states $S$ and parity of the orbital angular momentum $L$ are denoted as ${}^{2S+1}(-1)^{L}$, for example, singlet-even (${}^{1}E$) and triplet-odd (${}^{3}O$). 
${\bf (27), (8_{\rm_{s}})}$ and ${\bf (1)}$ are symmetric, and ${\bf (10), (10^*)}$ and ${\bf (8_{\rm{a}})}$ are asymmetric under the exchange of two baryons. By introducing $s$ quark in the SU(2), all multiplets but ${\bf (27), (10^*)}$ newly appear in the Hyperon-Nucleon ($YN$) or Hyperon-Hyperon ($YY$) channels. 

\begin{table}[h]
  \begin{center}
    \caption{Combination of linear couplings in six irreducible representations of $\BB$ interactions between the isospin and SU(3)$_f$ bases. Spin states $S$ and parity of the orbital angular momentum $L$ are denoted as ${}^{2S+1}(-1)^{L}$, for example, singlet-even (${}^{1}E$) and triplet-odd (${}^{3}O$).}
    \begin{tabular}{cccc} \hline \hline
      Strangeness & $B_{8}B_{8}(I)$ & $^1E$ or $^3O$ & $^3E$ or $^1O$ \\ \hline
      0 & $NN (I=0)$ & --- & ${\bf (10^*)}$ \\
      0 & $NN (I=1)$ & ${\bf (27)}$ & --- \\ \hline
      -1 & $\Sigma N (I=1/2)$ & $\frac{1}{\sqrt{10}}[(3({\bf 8_{\rm{s}}})-{\bf (27)}]$ & $\frac{1}{\sqrt{2}}[({\bf 8_{\rm{a}}})+{\bf (10^*)}]$ \\
      -1 & $\Sigma N (I=3/2)$ & ${\bf (27)}$ & ${\bf (10)}$ \\ \hline
      -1 & $\Lambda N (I=1/2)$ & $\frac{1}{\sqrt{10}}[({\bf 8_{\rm{s}}})+3{\bf (27)}]$ & $\frac{1}{\sqrt{2}}[-({\bf 8_{\rm{a}}})+{\bf (10^*)}]$ \\ \hline\hline
   \end{tabular}
   \label{table-BBint}
   \end{center}
\end{table}

%The quark-cluster model (QCM) \cite{Oka-2000} predicts that two $\SN$ channels in the $S-$wave, $(S = 0, I = 1/2)$ and $(S = 1, I = 3/2)$ have almost forbidden states. 
\textcolor{red}{ ${\bf (8_{\rm{s}})}$ and ${\bf (10)}$ are predicted to be strongly repulsive due to the Pauli forbidden states at a quark level. In contrast, it is said that ${\bf (8_{\rm{a}})}$ has a weak repulsive core, and ${\bf (1)}$ has an attractive pocket. These features were recently reproduced by Lattice Quantum Chromo-Dynamics (Lattice QCD) simulations \cite{QCD-2019} (see more details in Sec. \ref{sec-LatticeQCD}). } Especially, the $\SN(I=3/2)$ channel ($\SPp$ and $\SMn$) are expected to have a large repulsive core originated from the quark Pauli effect since it only includes ${\bf (27)}$ and ${\bf (10)}$. The $YN$ channel is also suitable for investigating ALS forces. The reason is that while the SLS force is symmetric with respect to spin swapping, the ALS force is antisymmetric, so the picture of the ALS force can only be investigated in the $YN$ channel that includes both singlet and triplet terms. 


\begin{comment}


From this viewpoint, the J-PARC E40 experiment ($\Sp$ scattering experiment) measured the differential cross-sections of $\SPp\ (I = 3/2)$ and $\SMp\ (I = 1/2)$ channels with better accuracy than the past data. They are essential inputs to fix the current $\BB$ interaction models. So, the understanding of ${\bf (8_{\rm{s}})}$ and ${\bf (8_{\rm{a}})}$ would make progress near future, following ${\bf (27)}$ and ${\bf (10^*)}$ which have been well studied by abundant $NN$ scattering data. 

The wavefunction of a many-particle system consisting of fermions of the same kind must be antisymmetric concerning the exchange of particles. Therefore, the $NN$ channel is forbidden to be spin-antisymmetric ($S=0$) when it is isospin-antisymmetric ($I=0$) and spin-symmetric ($S=1$) when it is isospin-symmetric ($I=1$). The $\BB$ effective interaction is generally expressed as
\begin{align}
  V(\bm{r}) =& V_{\rm c} \\
  	  +& V_{\sigma}(\bm{s_a} \cdot \bm{s_b}) \\
	  +& V_{\rm LS}(\bm{s}\cdot\bm{L}) \\
	  +& V_{\tau\tau}(\bm{\tau_a}\cdot\bm{\tau_b}) \\
	  +& V_{\rm T}S_{12} \\
	  +& \cdots
\end{align}
where $V_{\rm c}$ is the central force, not including the interaction between the spins $\bm{s_a}$ and $\bm{s_b}$ and the spatial relative coordinates $\bm{r = r_a - r_b}$, $V_{\sigma}(\bm{s_a} \cdot \bm{s_b})$ is the spin-spin interaction depending on the relative orientation of the spins $\bm{s_a}$ and $\bm{s_b}$, $V_{\rm LS}(\bm{s}\cdot\bm{L})$ is the spin-orbit angular momentum interaction (LS force) that depends on the orientation between spin and orbital angular momentum, $V_{\tau\tau}(\bm{\tau_a}\cdot\bm{\tau_b})$ is the isospin-isospin interaction depending on the relative orientation of isospin $\bm{\tau_a}$ and $\bm{\tau_b}$, and $V_{\rm T}S_{12}$ is the potential $V_{\rm T}$ and its tensor operator $S_{12}$, which represents the tensor force involving the interaction between spins and spatial relative coordinates.

The LS force is composed of $\bm{L}\cdot(\bm{s_a}+\bm{s_b})$ symmetric to the spin exchange and $\bm{L}\cdot(\bm{s_a}-\bm{s_b})$ anti-symmetric to the spin exchange. In particular, examining channels containing spin-singlet and spin-triplet terms is necessary to understand the antisymmetric LS force. As Table \ref{table-BBint} shows, the $NN$ channel for each isospin contains either spin-singlet or spin-triplet terms so that no restrictions can be imposed on the antisymmetric LS force. Therefore, to examine the antisymmetric LS force, it is necessary to investigate the $\SN$ and $\LN$ channels that contain both spin-singlet and spin-triplet terms.

%ALS in LN channel
\textcolor{red}{ This unique feature in the $YN$ channels, an ALS force, is proportional to the difference between the spins of two different baryons. } Although ALS force vanishes in the SU(2) limit, it survives in the SU(3)$_f$ symmetry \cite{Morimatsu-1984} \cite{Tani-1996}. Since ALS force changes the flavor symmetry of two baryons, only the ${\bf (8_{\rm_{s}})-(8_{\rm_{a}})}$ transition is nonzero in the SU(3)$_f$ symmetry. In the $\BB$ representation, ALS force leads relations among the $\BB$ matrix elements \cite{Oka-2000} as
\begin{align}
  \langle \LN {}^1P_1 |{\rm ALS}| \SN^{(1/2)} {}^3P_1 \rangle &= - \langle \SN^{(1/2)} {}^1P_1 |{\rm ALS}| \SN^{(1/2)} {}^3P_1 \rangle \\
  3 \langle \LN {}^1P_1 |{\rm ALS}| \LN {}^3P_1 \rangle &= -3 \langle \SN^{(1/2)} {}^1P_1 |{\rm ALS}| \LN {}^3P_1 \rangle
\end{align}

In the meson exchange model, only the interference term of the vector and tensor couplings of the vector meson exchange can contribute to ALS force in the SU(3)$_f$ symmetry, so ALS force becomes weak \cite{OBE-1999}. 
%
In contrast, the QCM predicts a large ALS force in the $YN$ channels. Its strength is comparable to the ordinary symmetric spin-orbit (SLS) force. The single particle spin-orbit force on the $\Lambda$ in a nucleus is determined by the sum of SLS and ALS forces over the nucleons surrounding $\Lambda$. Then the spin-orbit interaction for the $\Lambda$ spin, a sum of ALS and SLS forces, is relevant. SLS and ALS forces have different signs, so the resulting spin-orbit force may become small. Recent experimental studies of the structures of light hypernuclei suggest that the LS force for $\Lambda$ is one order smaller than that for the nucleon. This result is explained qualitatively by a large ALS force and its cancellation of the SLS force \cite{Oka-2000}. Ref. \cite{Ishikawa-2004} predicts ALS force in the $\Lp$ system is stronger than that in the $\SPp$ system, and analyzing power (polarization) of the polarized beam in the $\Lp$ scattering, which is equivalent to respective cross-section asymmetries, is sensitive to ALS force. Therefore, measurements of that analyzing power would give a clear-cut examination of such characteristic features of the LS forces in $YN$ channels. The J-PARC E86 experiment ($\Lp$ scattering experiment) will measure this component using a highly-polarized $\Lambda$ beam produced by the $\PiKL$ reaction at the K1.1 beamline in the J-PARC near future. Since all channels are related to each other in the SU(3)$_f$ symmetry, combining experimental data of $YN$ scattering is indispensable to improve theoretical models for two-body Baryon-Baryon interactions.


\end{comment}

\clearpage
%%%%%
\section{Two-body $YN$ interaction} 
\label{sec-YNint}

First, this section introduces several theoretical models describing the two-body interaction in SU(3)$_f$. Second, the current experimental progress in the study of two-body $YN$ interaction.
%%%%
\subsection{Theoretical models in SU(3)$_f$}
%%%
\subsubsection{Boson exchange model}
The Boson Exchange Potential is based on the boson field, which results from the boson exchange between two baryons. The Nijmegen model is one of the widely used models in \textcolor{red}{hypernuclear physics}. The Nijmegen Soft-Core (NSC) models express the repulsive core by the pomeron exchange, representing the phenomenological multiple gluon exchange. New versions of NSC, Nijmegen Extended Soft-Core (ESC) models describe $NN$, $YN$, and $YY$ interactions under SU(3) symmetry breaking in a unified manner. It consists of OBE, Two-Pseudo-Scalar meson exchange (TPS), Meson Pair Exchange (MPE), and local and nonlocal potentials.

The latest version of ESC (ESC16) \cite{ESC16} incorporates several meson exchange effects to describe the small spin-orbit splitting expected in the $\Lambda$ hypernucleus, and their validity should be tested from precise measurements of the analyzing power $A_y$ in the $\Lp$ channel. ESC16 also describes the repulsion phenomenologically as a sum of \say{pure} pomeron exchange and \say{pomeron-like} exchange. This, too, needs to be validated by precise $YN$ scattering data.
%ESC16論文から図を入れる?

%%%
\subsubsection{Quark model}
The Quark-Cluster Models (QCM), such as FSS and fss2, describe the $\BB$ interactions at the quark level \cite{Fujiwara-2007}. Their study of the $\BB$ interactions started by applying the three-quark ($3q$) model for the nucleon to the \textcolor{red}{$qqq$-$qqq$} system to obtain a microscopic understanding of the short-range repulsion of the $NN$ interaction. The effect of antisymmetrization among quarks is considered in the framework of a microscopic nuclear cluster model, the Resonating-Group Method (RGM), including the color degree of freedom of quarks. This model consists of the phenomenological confinement potential, the color Fermi-Breit (FB) interaction with explicit Flavor-Symmetry Breaking (FSB), and effective-meson exchange potentials of scalar-, pseudoscalar-, and vector-meson types. 

This framework incorporates both quark and mesonic degrees of freedom. Since the meson-exchange effects are the non-perturbative aspect of QCD, these are described by the Effective-Meson Exchange Potentials (EMEPs) acting between quarks. The color-magnetic term of the FB interaction and the Pauli effect at the quark level describe the short-range repulsion of the $\BB$ interactions. In FSS \cite{FSS}, only the scalar-meson and pseudoscalar-meson exchanges are introduced as the EMEPs, while in fss2 \cite{fss2}, the vector-meson exchanges are also introduced. All standard terms used in the non-relativistic OBEP are included. They also try to study the charge dependence of the $\BB$ interactions by applying the pion-Columb correction in the particle basis. 

The fss2 version reproduces the $NN$ phase shifts at non-relativistic energies. However, the $\Lambda$ single-particle potential has a depth of 48 MeV (fss2) and 46 MeV (FSS) if the continuous prescription is used for intermediate energy spectra. These values are slightly more attractive than the experimental values of $\Lambda$-hypernuclei \cite{Bando-1990}. It also predicts strong repulsion of the $\Sigma$ single-particle potential due to the strong Pauli effect in the $\SN\ (S = 1, I = 3/2)$ state. The estimated strengths of the potentials are about 8 MeV (fss2) and 20 MeV (FSS). %Vector-meson exchanges might cause this difference but need to be carefully investigated, referring to ultra-precise $\Sigma$-hypernuclei experimental data. 
Additionally, fss2 describes the LS force generated from the scalar meson EMEP. If this contribution is significant, canceling the SLS and ALS force components from the FB interaction becomes less prominent. The estimated ratios of the Scheerbaum factors, $S_{\Lambda}/S_N$, are $\approx1/4$ (fss2) and $\approx1/12$ (FSS). The ALS force information derived from the analyzing power of the $\Lp$ scattering data should also check this difference. 

Their three-cluster formalism would be the first step to solving the few-body systems that interact via the quark-model $\BB$ interactions, considering the essential features of RGM, the non-locality, the energy dependence, and the pairwise Pauli-forbidden state.


%%%
\subsubsection{Chiral effective field theory}
The Weinberg power counting derives the chiral Baryon-Baryon potentials for the strangeness sector. The Leading Order (LO) potential comprises four-baryon contact terms without derivatives and one-pseudoscalar-meson exchanges. In contrast, at Next-to-Leading Order (NLO), contact terms with two derivatives arise, together with contributions from (irreducible) two-pseudoscalar-meson exchanges. The contributions from pseudoscalar-meson exchanges (the Goldstone bosons $\pi$, $\eta$, $K$ of QCD's spontaneously broken chiral symmetry) are completely fixed by the assumed SU(3)$_f$. On the other hand, the strength parameters associated with contact terms, the Low-Energy Constants (LECs), need to be determined in a fit to data. 

An early study of $\LN$ and $\SN$ scatterings up to NLO in SU(3) chiral Effective Field Theory ($\chi$EFT) in 2013 demonstrated that one could comprehensively describe the available low-energy $\LN$ and $\SN$ data, following closely earlier analogous investigations of the $NN$ interaction \cite{chiEFT-2013}. The $YN$ potential up to NLO in SU(3) $\chi$EFT consists of contributions from one- and two-pseudoscalar-meson exchange diagrams (involving the Goldstone bosons) and from four-baryon contact terms without and with two derivatives \cite{chiEFT-2020}. As mentioned, the strength of the contact interactions is characterized by the various LECs. In the present work \cite{chiEFT-2020}, the new potential for the $\LN$ and $\SN$ interactions at NLO in SU(3) $\chi$EFT was established to reduce the number of LECs by inferring some of them from the $NN$ sector via the underlying SU(3) symmetry. The difference between this new potential and the initial NLO one published in 2013 \cite{chiEFT-2013} appears in the strength of the $\LNtoSN$ transition potential, while these two potentials yield equivalent results for $\LN$ and $\SN$ scattering observables. Figure \ref{fig-chiEFT_3S1_LN} shows the outcome of the effect from the $\LNtoSN$ transition for the $\LN$ $^3S_1$ phase shift \cite{chiEFT-2020}. The results on the left side are for the full (coupled channel) calculation, and it is obvious that the phase shifts for the NLO13 and NLO19 potentials lie on top of each other, at least up to momenta of $p_{lab}$, 400 MeV/$c$. On the right side, the results without channel coupling are shown.
\begin{figure}[h]
 \begin{center}
   \includegraphics[clip,width=12cm]{chiEFT_3S1_LN.png}
   \caption{$^3S_1$ $\LN$ phase shift with (left) and without (right) $\SN$ coupling \cite{chiEFT-2020}. (Left) The full (coupled channel) calculation. The phase shifts for the NLO13 and NLO19 potentials lie on each other, at least up to momenta of $p_{lab}$, 400 MeV/$c$. (Right) The same calculation without considering the $\LN - \SN$ coupling.}
   \label{fig-chiEFT_3S1_LN}
 \end{center}
\end{figure}

It is suggested that the effect of the variation in the strength of the $\LNtoSN$ coupling ($\Lambda$-$\Sigma$ conversion) is sizable in the case of the matter properties. %To fully constrain the $\LNtoSN$ transition potential, more $YN$ scattering data covering the momentum range of this transition is necessary, besides consistent three-body forces to compensate for the differences in few- and many-body systems.
Since the statistics of $YN$ scattering data are insufficient, LECs at NLO and higher order and the $\LNtoSN$ transition potentials still have ambiguity. Measuring the $YN$ cross sections in a higher energy range is now required to constrain the $P$-wave contribution. %Hence, the experimental values of differential cross section and polarization observables in the $\Lambda$ beam momentum range of $400-800$ MeV/$c$ would play an important role in studying $\chi$EFT at NLO and higher order. 

%%%
\subsubsection{Lattice QCD}
\label{sec-LatticeQCD}
Lattice Quantum Chromo-Dynamics (Lattice QCD) is a numerical simulation established independently of the above theoretical model. QCD, a field theory that introduces color degrees of freedom into the quark picture, is set up on a lattice. The Hadrons to Atomic nuclei from Lattice (HAL) QCD method was established in 2006 to extract $NN$ interactions from QCD on a lattice \cite{QCD-2006}. HAL QCD method has been applied successfully to $YN$ systems \cite{QCD-YN}, general $\BB$ systems \cite{QCD-BB}, meson-baryon systems \cite{QCD-MB}, and so on. This means obtaining hyperon interactions without requiring experimental data, but relying on QCD is possible. 

HAL QCD collaboration also performed a Lattice QCD numerical simulation and extracted $\BB$ interactions from QCD using the HAL QCD method in 2019 \cite{QCD-2019}. Since they employed a full QCD gauge configuration ensemble, the masses of the hadrons in their Lattice QCD simulation are nearly physical; for example, the $\pi$ meson mass was 146 MeV, the $K$ meson mass was 525 MeV, and the nucleon mass was 958 MeV. 

Although SU(3)$_f$ is approximate in the physical world, the symmetry is useful in the Lattice QCD simulation. If expressing the $\BB$ interactions in the flavor irreducible representation basis, the diagonal parts of the flavor-basis potential, where the upper three figures correspond to flavor-symmetric two-baryon systems in the $^1S_0$ partial wave, and the lower six figures correspond to flavor-anti-symmetric two-baryon systems in the $^3S_1$$-$$^3D_1$ partial waves can be obtained as shown in Figure \ref{fig-BB_Swave}. Here, the entire attraction in ${\bf (1)}$ and the strong repulsion in ${\bf (8s)}$ and ${\bf (10)}$ are characteristic, which means the phenomenologically reasonable nuclear force can be derived from Lattice QCD simulation.


\clearpage
%%%%
\subsection{Current experimental progress}
%%%
%%%
\subsubsection{$\Lambda$ hypernuclei}
Under the lack of $YN$ scattering data, alternative information was obtained from studying hypernuclei, bound systems composed of nucleons and one or more hyperons. Speaking of $\Lambda$ hyperon, it has been well studied through $\Lambda$ hypernuclei. Since $\Lambda$ hyperon will not be affected by Pauli blocking by the other nucleons, it can get into the nuclear interior and form deeply bound hypernuclear states. It is expected that new forms of hadronic many-body systems can be investigated using this new degree of freedom, \say{strangeness}, and also comparing the $YN$ and $NN$ interactions. New nuclear structure, which cannot be studied in ordinary nuclei consisting only of nucleons, could provide indispensable information on the SU(3)$_f$ system for baryonic matter \cite{Hashi-Tamu}.

From the 1950s to 1970s, the binding energies of light ($A\leq16$) $\Lambda$ hypernuclei were measured from their weak decays by emulsion. However, experimental data are limited to ground-state binding energies, and excited states cannot be investigated except in some cases. 

In the 1970s, the counter experiments using the $\Kpi$ reaction began to study excited states of hypernuclei at CERN and later Brookhaven National Laboratory (BNL). However, statistics were often insufficient due to the low kaon beam intensity and limited spectral energy resolution. Furthermore, since the $\Kpi$ reaction has \say{magic momentum} where momentum transfer to the recoil hypernuclear becomes 0, only a limited number of hypernuclear states were studied \cite{Kpi-1973, Kpi-1975, Kpi-1976, Kpi-1978, Kpi-1979, Kpi-1981, Kpi-1979-2, Kpi-1981-2, Kpi-1973-2}.

From the mid-1980s, experiments using $\pPK$ reactions began at the Alternating Gradient Synchrotron (AGS) at BNL, USA \cite{pPK-1985, pPK-1991}. Furthermore, new experimental facilities became available at the 12 GeV proton synchrotron (PS) at Japan High Energy Accelerator Research Organization (KEK), and experiments were actively conducted \cite{PS-1991, PS-1995, PS-1996, PS-1998, PS-2001}. Especially, the Superconducting Kaon Spectrometer (SKS) played an important role in the exploration of $\Lambda$ hypernuclear spectroscopy via $\pPK$ reactions, which led to the establishment of hypernuclear spectroscopy.

Experiemental data of light hypernuclei, such as $\htl$, $\hefl$ and $\hefil$, have been used to investigate the contributions of $^1S_0$ and $^3S_1$ partial waves in the $\Lp$ interaction \cite{chiEFT-2020}\cite{chiEFT-2020_2}. As discussed in Ref. \cite{chiEFT-2020_2}, the fact that shell-model calculations indicate that the spin-orbit interaction contributes to the excitation energies seems to contradict the prospect of $\chi$EFT that $YN$ $P$-waves do not significantly. However, the $P$-wave interactions are identical in all NLO forces, so more detailed studies, including variations of the $\Lp$ interaction in higher partial waves, are required to understand this issue better. To know the $\Lp$ $^1S_0$ partial wave, finding ways to disentangle the detailed information on spin dependence is a crucial issue on the experimental side. 

Heavier $\Lambda$ hypernuclei, such as $\ctl$ to $\pbtl$, have been the probes to compute the $\Lambda$ single particle orbitals and to study the spin-orbit interaction in Ref. \cite{chiEFT-2009}. Ref. \cite{shell-2005} also discussed the spin dependence of the $\LN$ $S$-wave interactions by the experimental values of $\lisl$, $\benl$, $\btl$, $\bel$, $\nfl$, and $\osl$. However, as discussed in Ref. \cite{Lp-1968_Alex}, determining the scattering lengths of the $^1S_0$ and $^3S_1$ $\Lp$ partial waves was performed using only the scattering data. %Still, the error bars were large due to limited statistics, so there is almost no room to change the $S$-wave interactions drastically since the $\BB$ interaction models have to reproduce the scattering data and the hypernuclear data simultaneously.

As discussed above, though hypernuclear data have studied $YN$ interaction, it is still difficult to extract $P$-wave interactions due to the following matters. First, the binding energies of light hypernuclei are insensitive to $P$- and higher partial waves \cite{chiEFT-2020_2}. Second, it is said that the $P$-wave is indistinguishable from the many-body structure, such as middle-heavy hypernuclei \cite{Isaka-2017}. 

%%%
\subsubsection{Femtoscopy \textcolor{red}{新たに追加}}
Because the correlation function is sensitive to low-energy interactions, the femtoscopy that uses a two-particle momentum correlation function from high-energy collisions is useful for studying resonance states of exotic hadrons near the threshold. In femtoscopy, various interactions in the strangeness sector have been studied theoretically \cite{fem-th-1, fem-th-2, fem-th-3, fem-th-4, fem-th-5, fem-th-6, fem-th-7, fem-th-8, fem-th-9} and experimentally \cite{fem-ex-1, fem-ex-2, fem-ex-3, fem-ex-4, fem-ex-5, fem-ex-6, fem-ex-7, fem-ex-8, fem-ex-9, fem-ex-10, fem-ex-11, fem-ex-12}. The source size dependence of the correlation function has been found to be useful in distinguishing between the presence or absence of hadronic bound states \cite{fem-th-7, fem-th-8}. In addition to that, recently, the ALICE collaboration measured the $D^{-}p$ correlation function \cite{fem-ex-13}, paving the way for femtoscopy in the charm field. 


%%%
\subsubsection{$YN$ scattering}
To understand the \textcolor{red}{nuclear} properties, such as energies scheme and level structures in the quark picture, establishing the generalized $\BB$ interaction \textcolor{red}{considering the precise two-body $YN$ interactions} is crucial. \textcolor{red}{ Hyperon's lifetime is extremely short, on the order of $\sim0.1$ ns, so sometimes it decays before scattering occurs, and even if it does scatter, it is difficult to detect it directly. Therefore, historically, $YN$ scattering experiments have been rarely performed compared to $NN$ scattering experiments. }

%簡潔なE40の紹介
\textcolor{red}{ For the $\SN$ channels, recently we pioneered a new $\SPMp$ scattering experiment with high statistics. Here, we generate hyperons by irradiating the target with the J-PARC secondary meson beam to use as a tertiary beam to cause $YN$ scattering. %The generated hyperons are tagged by the missing mass method, and scattering events are indirectly identified by kinematical consistency analysis. 
The J-PARC E40 experiment was performed to achieve the high-statistic $\SPMp$ scattering measurement to investigate the validity of the meson-exchange model in the $\SN(I=1/2)$ channel and the quark Pauli effect in the $\SN(I=3/2)$ channel. It finally measured the differential cross-sections of $\SMp$ scattering \cite{Miwa-SMp} (Figure \ref{fig-dcs_SMp}), $\SMpLn$ inelastic scattering \cite{Miwa-SMLn} (Figure \ref{fig-dcs_SMpLn}), and $\SPp$ scattering \cite{Nana-SPp} (Figure \ref{fig-dcs_SPp}). Ref. \cite{Nana-SPp} also measured the phase shifts $\delta_{^3S_1}$ and $\delta_{^1P_1}$, as shown in Figure \ref{fig-phaseshift_SPp}. }

%Lpについて
For the $\LN$ channel, theoretical models were built based on the past $\Lp$ scattering data in a lower momentum range ($\leq400$ MeV/$c$). Figure \ref{fig-totcsLp_old} \cite{chiEFT-2020} shows the past measurements of the total cross-sections of the $\Lp$ scattering with theoretical models. The red (dark) band represents the result for NLO13 \cite{NLO13} including cutoff variations, the cyan (light) band for the alternative version NLO19, the dashed curve for the J\"{u}lich '04 meson-exchange model \cite{Julich04}, and the dotted curve for the Nijmegen NSC97f potential \cite{NSC97f}. The experimental cross-sections are taken from Refs. \cite{Lp-1968_Sechi} (filled circles), \cite{Lp-1968_Alex} (filled squares), \cite{Lp-1967_Herndon} (open triangles), \cite{Lp-1971} (open squares), and \cite{Lp-1977_Hauptman} (open circles). Below the $400$ MeV/$c$ range, several cross-section measurements with hydrogen bubble chambers were used to determine the $S$-wave contribution. In contrast, the $\Lp$ scattering data over the $400$ MeV/$c$ range are still lacking, which makes it difficult to study higher wave contributions such as $P$- and $D$-waves. Although the CLAS collaboration reported the new total cross-section data \cite{Lp-2021} in 2021 (Figure \ref{fig-Lp_totcs_CLAS}), the experimental data is still insufficient to restrict the theoretical models. As shown in Figure \ref{fig-Lp_dcs_models} \cite{chiEFT-2020}, models show different behaviors \textcolor{red}{in differential cross-section} in each beam momentum (i.e., $p_{lab}=500$ MeV/$c$, or $p_{lab}=633$ MeV/$c$). Here, both $\chi$EFT version of NLO13 and NLO19 suggest that the $S-$wave contributes drastically at 500 MeV/$c$, while at 633 MeV/$c$ the $\SPn$ threshold makes a strong angular dependence. the J\"{u}lich '04 meson-exchange model predicts the $P-$wave contribution even at 500 MeV/$c$. In contrast, according to Ref. \cite{OBE-1999}, the large angular dependence in the Nijmegen NSC97f potential is derived by $^3D_1$ contribution. 

%以下の文章入れる?要検討。
%\textcolor{red} {To establish the precise two-body $\LN$ interaction, we will perform the next-generation $\Lp$ scattering experiment at J-PARC which follows the same experimental technique as J-PARC E40. It aims to measure not only differential cross-section but spin observables such as analyzing power and depolarization. }

%以下は記載なしの方向で。
%The combination of linear couplings in six irreducible representations of $\BB$ interactions for $\LN$ channel has ${\bf (8_{\rm_{s}}), (8_{\rm_{a}}), (10)^*}$ and ${\bf (27)}$ (see table \ref{table-BBint}). So, ideally, they should be studied more carefully after the cross-section data of the $\SPp$ and $\SMp$ scatterings from the J-PARC E40 experiment restrict theoretical models. Furthermore, the information on ALS force, newly appearing in the $YN$ channels, should be studied for the first time by the spin observables data of $\Lp$ scatterings.


%三輪さんと七村さんの論文の図
\begin{figure}[!h]
  \begin{center}
   \includegraphics[width=15cm]{dcs_SMp.png}
   \caption{Differential cross-sections of the $\SMp$ elastic scattering obtained from the J-PARC E40 experiment (black points) \cite{Miwa-SMp}. The error bars and boxes show statistical and systematic uncertainties, respectively. The red points represent averaged differential cross-sections in the momentum range of $400-700$ MeV/$c$ obtained in KEK-PS E289. The dotted (magenta), dot-dashed (blue), and solid (yellow) lines represent the calculated cross-sections by the Nijmegen ESC08c model based on the boson-exchange picture, the fss2 model including QCM, and the extended $\chi$EFT model, respectively.}
   \label{fig-dcs_SMp}
 \end{center}
\end{figure}

\begin{figure}[!h]
  \begin{center}
   \includegraphics[width=15cm]{dcs_SMpLn.png}
   \caption{Differential cross-sections of the $\SMpLn$ inelastic scattering obtained from the J-PARC E40 experiment (black points) \cite{Miwa-SMLn}. The error bars and boxes show the statistical and systematic uncertainties, respectively. The dotted magenta and green lines represent the calculated cross-sections by the Nijmegen ESC08c \cite{OBP2} and ESC16 \cite{ESC16} based on the boson-exchange model. The dot-dashed (blue) line represents the calculation by the fss2 model, including QCM \cite{Fujiwara-2007}. The solid orange and red lines represent the calculations by two versions of the extended $\chi$EFT model: NLO13 \cite{NLO13} and NLO19 \cite{chiEFT-2020}. In both cases, the cutoff value of 600 MeV was used.}
   \label{fig-dcs_SMpLn}
 \end{center}
\end{figure}

\begin{figure}[!h]
  \begin{center}
   \includegraphics[width=15cm]{dcs_SPp.png}
   \caption{Differential cross-sections of the $\SPp$ elastic scattering obtained from the J-PARC E40 experiment (black points) \cite{Nana-SPp}. The error bars and boxes show the statistical and systematic uncertainties, respectively. The red boxes and blue triangles represent the data of past measurements, KEK E251 \cite{KEK1} and KEK E289 \cite{KEK3}, respectively. The blue dotted and dot-dashed lines represent the calculated cross-sections by FSS and fss2 \cite{Fujiwara-2007}, respectively. The green solid lines and black dashed lines represent the calculations by the Nijmegen NSC97f \cite{OBE-1999} and ESC08s \cite{OBP2}, respectively. The orange and red dot-dashed lines represent the calculations by the $\chi$EFT NLO13 \cite{NLO13} and NLO19 \cite{chiEFT-2020}.}
   \label{fig-dcs_SPp}
 \end{center}
\end{figure}

\begin{figure}[!h]
  \begin{minipage}[t]{0.48\columnwidth}
    \centering
    \includegraphics[width=\columnwidth]{phaseshift_SPp_3S1.png}
  \end{minipage}
  \hspace{0.04\columnwidth} % ここで隙間作成
  \begin{minipage}[t]{0.48\columnwidth}
    \centering
    \includegraphics[width=\columnwidth]{phaseshift_SPp_1P1.png}
  \end{minipage}
  \caption{Phase shifts $\delta_{^3S_1}$ (left) and $\delta_{^1P_1}$ (right) in the $\SPp$ system as a function of the incident $\SP$ momentum obtained from the J-PARC E40 experiment \cite{Nana-SPp}. The dashed (black), solid (green), and dotted (blue) lines represent the calculated phase shifts by ESC16 \cite{ESC16}, NSC97f \cite{OBE-1999}, and fss2 \cite{Fujiwara-2007}, respectively.}
  \label{fig-phaseshift_SPp}
\end{figure}

%
\begin{figure}[h!]
 \begin{center}
   \includegraphics[clip,width=8cm]{totcsLp_old.png}
   \caption{Cross-section for $\Lp$ scattering as a function of beam $\Lambda$ momentum ($p_{lab}$) \cite{chiEFT-2020}. The red (dark) band represents the result for NLO13 \cite{NLO13} including cutoff variations, the cyan (light) band for the alternative version NLO19, the dashed curve for the J\"{u}lich '04 meson-exchange model \cite{Julich04}, and the dotted curve for the Nijmegen NSC97f potential \cite{NSC97f}. The experimental cross-sections are taken from Refs. \cite{Lp-1968_Sechi} (filled circles), \cite{Lp-1968_Alex} (filled squares), \cite{Lp-1967_Herndon} (open triangles), \cite{Lp-1971} (open squares), and \cite{Lp-1977_Hauptman} (open circles).}
   \label{fig-totcsLp_old}
 \end{center}
\end{figure}

\begin{figure}[h!]
 \begin{center}
   \includegraphics[clip,width=12cm]{Lp_totcs_CLAS.png}
   \caption{Total cross sections of $\Lp$ elastic scattering as the function of $\Lambda$ momentum in the Laboratory system  \cite{Lp-2021}. Blue solid boxes represent present results by CLAS, large circles represent averaged one over a wide momentum range, and small circles represent previous results. The band represents the uncertainty within the $\chi$EFT model. The shaded region at the bottom shows the systematic error.}
   \label{fig-Lp_totcs_CLAS}
 \end{center}
\end{figure}

\begin{figure}[h]
 \begin{center}
   \includegraphics[clip,width=15cm]{Lp_dcs_models.png}
   \caption{Differential cross section for $\Lp$ scattering at 500 MeV/$c$ and at 633 MeV/$c$ \cite{chiEFT-2020}. The red (dark) band represents the result for NLO13 \cite{chiEFT-2013}, including cutoff variations, and the cyan (light) band for the alternative version NLO19. The dashed curve is the result of the J\"{u}lich '04 meson-exchange model \cite{chiEFT-2005}, and the dotted curve is that of the Nijmegen NSC97f potential \cite{NSC97f}.} 
   \label{fig-Lp_dcs_models}
 \end{center}
\end{figure}


%%%%%%%%%%%%%%%%%%%%%%%%%%%%%%%%%%%%%%
%Hyperon puzzle of neutron stars
%%%%%%%%%%%%%%%%%%%%%%%%%%%%%%%%%%%%%%
%%%%%
\clearpage
\section{Hyperon puzzle of neutron stars}
During stellar evolution, the $\rm{^{56}Fe}$ core of massive stars, over $10\msolar$, collapses by photodisintegration as 
\begin{equation}
  \rm{^{56}Fe} + \gamma \to 13\rm{^4He} + 4n 
\end{equation}
followed by 
\begin{equation}
  \rm{^4He} + \gamma \to 2p + 2n.
\end{equation}
At the same time, electrons change protons to neutrons by electron capture as 
\begin{equation}
  e^- + p \to n + \nu_e.
\end{equation}
Then, such stars no longer withstand gravity, which triggers the supernova. After the cooling and neutrino emission of proto-neutron stars, neutron stars or black holes (if the star's mass is over $30 \msolar$) are left as the remains. 
Recently, it has been said that $YN$ and $YY$ interactions could solve the \say{hyperon puzzle} of neutron stars. In the core of neutron stars, nucleon changes to hyperon energetically if its chemical potential becomes larger than the mass difference between nucleon and hyperon. That means hyperons appear at the density $2-3 \rho_0$ ($\rho_0$ is a nuclear saturation density) as shown in Figure \ref{fig-hypapp} \cite{RMF-2008}. Due to the attraction of the single-particle potential of $\Lambda$ $U_\Lambda\sim-30$ MeV, it is expected to appear first in such a high-density region of neutron stars. If hyperons are present, the fermi pressure will be decreased, and the Equation of State (EoS) becomes softer, reducing the maximum mass value of the star. The degrees of the EoS softening and the maximum mass value reduction depend on the $YN$ and $YY$ interactions.
With hyperons, the neutron star's maximum mass value is expected to be up to $1.5\msolar$ due to decreasing neutron degeneracy pressure by hyperon mixing \cite{BHF-2008}. In contrast, the observations of massive neutron stars ($m\sim2\msolar$) \cite{NS-1} \cite{NS-2} indicates that the knowledge of current hypernuclear physics has room for improvement (see Figure \ref{fig-NS}). To solve this problem, so-called \say{hyperon puzzle} of neutron stars, an additional repulsion mechanism is necessary to make the EoS stiffer to support the massive neutron stars. 

In the last few years, some claim that the repulsive many-body $YN$ force, such as $\Lambda NN$ and $\Lambda NNN$, could be the candidate. Ref. \cite{Schmidt-1999} presented the first Quantum Monte Carlo analysis of infinite matter composed of neutrons and $\Lambda$ particles to investigate the effect of repulsive three-body $\Lambda NN$ force in Hyper Neutron Matter (HNM) \cite{Diego-2015}. In this calculation, the auxiliary field diffusion Monte Carlo (AFDMC) algorithm, successfully applied to investigate the properties of Pure Neutron Matter (PNM), was employed. This letter has not considered hyperons other than $\Lambda$ due to the lack of potential models suitable for quantum Monte Carlo calculations. Considering two different models that successfully describe the binding energy of medium mass hypernuclei, it reported that one of their repulsive three-body $\Lambda NN$ force (type \rom{2}) could make the mass-radius relation to explaining the heavy pulsars observations in HNM as shown in Figure \ref{fig-TBF}.

Therefore, understanding three-body $YNN$ interactions could be a foothold to study its behavior in a high-density situation like a neutron star interior. \textcolor{red}{To clarify the mechanism of the three-body $YNN$ interaction, we must first precisely determine the two-body $\LN$ interaction from scattering experiments, considering the spin component. The next section will introduce the spin observables of $YN$ scattering experimentally measurable.}

\begin{figure}[h]
  \begin{center}
   \includegraphics[clip,width=12cm]{hypapp.png}
   \caption{The fraction of baryons and leptons in neutron star matter based on an RMF calculation \cite{RMF-2008}.}
   \label{fig-hypapp}
 \end{center}
\end{figure}

\begin{figure}[h]
  \begin{center}
   \includegraphics[clip,width=12cm]{NS.png}
   \caption{Observed $(1.97\pm0.04)M_{\odot}$ neutron star (J1614-2230) and other observed neutron star masses and main EOS theoretical model curves \cite{NS-1}. The models that allow the hyperon appearance inside neutron stars (GS1, GM3) do not reach the newly observed massive neutron stars (J1614-2230 and J1903+0327).}
   \label{fig-NS}
 \end{center}
\end{figure}

\begin{figure}[h]
  \begin{center}
   \includegraphics[clip,width=12cm]{TBF.png}
   \caption{Mass-radius relations calculated by AFDMC considering $\LN$ (red), $\Lambda NN$ (I) (blue), $\Lambda NN$ (II) (black), and PNM (green) \cite{Diego-2015}. Horizontal bands at $\sim2\msolar$ are the heavy pulsars observations (PSR J1614-2230 \cite{NS-1}, and PSR J0348+0432 \cite{NS-2}). The grey-shaded region is the excluded part of the plot due to causality.}
   \label{fig-TBF}
 \end{center}
\end{figure}





%%%%%%%%%%%%%%%%%%%%%%%%%%%%%%%%%%%%%%
%Measurable physics observables
%%%%%%%%%%%%%%%%%%%%%%%%%%%%%%%%%%%%%%

%%%%%
\clearpage
\section{Measurable spin observables}
%%%%

\begin{comment}
\subsection{Differential cross-section}
To experimentally obtain information on the two-body interaction potential, measuring the differential cross-section $d\sigma/d\Omega$ and determining its scattering phase difference $\delta$ is necessary. The differential cross-section derived from the scattering amplitude $f(\theta)$ expanded by the orbital angular momentum $l$ is given by 
\begin{equation}
  \frac{d\sigma}{d\Omega}(\theta) = |f(\theta)|^2 = \left| \cfrac{1}{k}\sum_{l=0}^\infty (2l+1)e^{i\delta_l}\sin\delta_l P_l (\cos\theta)\right|^2,
  \label{eq: potential}
\end{equation}
where $k$ is the wavenumber, $P_l$ is the $l^{\rm th}$ Legendre polynomial, and spin weights are not considered. $\delta_l$ is the phase difference between the waves scattered by the potential of a partial wave with angular momentum $l$ and unscattered waves. When $\delta_l<0$, the potential is repulsive, and when $\delta_l>0$, the potential is attractive.

For the $\LN$ interaction, the contribution of the (${\bf 27}$) term in the spin-singlet can be estimated from the abundant experimental data of $NN$ scattering as in the $\SN$ interaction. The theoretical indeterminacy of the phase difference in the (${\bf 27}$) is expected to be very small. The spin-singlet and spin-triplet contributions can be constrained if the J-PARC E40 experiment determines the $\SN$ interaction.
\end{comment}

%%%%
%\subsection{Spin observables}
The spin observables for the two-body scattering $ab\to a'b'$ are the polarization of $a'$ relative to the scattering plane ($P_a$), the symmetry of the scattering emission angle of $a'$ \textcolor{red}{for the polarized beam $a$} (i.e., analyzing power, $A_y$), and the change in polarization of $a$ before and after scattering (i.e., depolarization, $D^y_y$). Here, the spin quantization axis of $a$ is defined as the normal vector to the scattering plane. 

\begin{comment}
Several models describe the $YN$ interaction. For example, there are 
\begin{itemize}
  \item {\bf Extended-Soft Core model (ESC16)} describes the short-range repulsion derived from the quark-Pauli effect phenomenologically using Pomeron exchange, etc.
  \item {\bf Quark-cluster model (QCM)} is a theory of interquark interactions to describe the repulsion.
  \item {\bf Chiral EFT} has the potential to be composed of meson exchange and contact interactions (parameterized by low-energy constants (LECs), the values of which need to be determined by fitting the scattering data) with the contribution of pseudoscalar octuplets. 
\end{itemize}
Compared to the abundance of $NN$ scattering data, $YN$ scattering data are still scarce. A robust model dependence appears in the amount of spin observed in SU(3)$_f$ space with the introduction of $s$ quarks in the model above. We plan to constrain these models by experimentally measuring the $ \Lp$ spin observables.
\end{comment}

For the $YN$ elastic scattering $ab\to a'b'$, the $T$ matrix can be represented in terms of spin-independent, spin-spin, SLS, ALS, and tensor components as
\begin{align}
  \bm{M} = V_{\rm c} \\
  &+ V_{\sigma}(\bm{s_a} \cdot \bm{s_b}) \\
  &+ V_{{\rm SLS}}(\bm{s_a} + \bm{s_b}) \cdot \bm{L} \\
  &+ V_{{\rm ALS}}(\bm{s_a} - \bm{s_b}) \cdot \bm{L} \\
  &+V_{{\rm T}}([\bm{s_a} \otimes \bm{s_b}]^{(2)} \cdot \bm{Y_2}(\bm{r})),
  %rのhatがうまくかけない
\end{align}
where $\bm{L}$ is the $a$-$b$ relative orbital angular momentum, $\bm{r}$ is the $a$-$b$ relative coordinate, and $V$s are form-factor functions, which include the higher-order effects, for the spin-independent central interaction $V_{\rm c}$, the spin-spin interaction $V_{\sigma}$, the SLS interaction $V_{\rm SLS}$, the ALS interaction $V_{\rm ALS}$, and the tensor interaction $V_{\rm T}$. %As mentioned, exchange effects due to strangeness transfers between the particles are included in $V_{\rm ALS}$. 
For the $NN$ scattering, $V_{\rm ALS}$ is eliminated because of the equivalence of $a$ and $b$ \cite{Ishikawa-2004}. 

The $T$ matrix can also describe the differential cross-section as
\begin{equation}
  \left( \frac{d\sigma}{d\Omega} \right) = \frac{1}{4}{\rm Tr}(MM^{\dagger}) = |U_{\alpha}|^{2} + \frac{3}{16}|U_{\beta}|^{2} 
  + \frac{1}{2}(|S_{\rm SLS}|^{2} + |S_{\rm ALS}|^{2}) + \frac{1}{4}|T_{1}|^{2} + \frac{1}{2}(|T_{2}|^{2} + |T_{3}|^{2}).
  \label{eq-dcs-theo}
\end{equation}
Each term in Equation (\ref{eq-dcs-theo}), new scalar amplitudes, vector amplitudes, and tensor amplitudes can be defined as follows. The scalar amplitudes:
\begin{align}  
  U_{\alpha} &\equiv <\bm{k_f}|V_{\rm c}|\bm{k_i}> ,\\
  U_{\beta} &\equiv <\bm{k_f}|V_{\rm \sigma}|\bm{k_i}>.
  \label{eq-scalarAmp}
\end{align}
The vector amplitudes:
\begin{align}  
  S_{\rm ALS} &\equiv <\bm{k_f}|V_{\rm ALS}L_{1}|\bm{k_i}>, \\
  S_{\rm SLS} &\equiv <\bm{k_f}|V_{\rm SLS}L_{1}|\bm{k_i}>.
  \label{eq-vectorAmp}
\end{align}
The tensor amplitudes for $j=1,2,3$:
\begin{align}  
  T_{j} &= \frac{1}{2} <\bm{k_f}|V_{\rm T}Y_{2,\ j-1}|\bm{k_i}>, \\
  T_{\alpha} &= \frac{1}{\sqrt{6}}T_{1} + T_{3}, \\
  T_{\beta} &= \frac{1}{\sqrt{6}}T_{1} - T_{3}, \\
  T_{2} &= -\tan{\theta} (\frac{1}{2} T_{\alpha} + T_{\beta}).
  \label{eq-tensorAmp}
\end{align}
Therefore, spin observables for $YN$ scattering are essential to determine each component's behavior separately in Equation (\ref{eq-dcs-theo}). In the $YN$ elastic scattering $ab\to a'b'$, the analyzing power ($A_{y}(Y)$) can be written by the amplitudes defined above as
\begin{equation}
  A_{y}(Y) = -\frac{1}{\sqrt{2}\sigma(\theta)} {\rm Im} \left\{ 
  (U_{\alpha} + \frac{1}{4}U_{\beta}^{*})S_{\rm SLS} 
  + (U_{\alpha} - \frac{1}{4}U_{\beta}^{*})S_{\rm ALS} 
  - \frac{1}{2}T_{\alpha}^{*}(S_{\rm SLS} - S_{\rm ALS}) \right\},
\end{equation}
where $\sigma(\theta)$ represents the differential cross-section. As one can notice here, the analyzing power is sensitive to the LS and ALS forces. The depolarization ($D^{y}_{y}(Y)$) can also be written as
\begin{equation}
\begin{split}
  D^{y}_{y}(Y) =& -\frac{1}{\sigma(\theta)} {\rm Re} \Biggl\{ 
  \frac{1}{2\sqrt{3}} \left( U_{0} + \frac{1}{\sqrt{3}}U_{1} \right)^{*}U_{1} + 
  \frac{1}{2} \left( U_{0} - \frac{1}{\sqrt{3}}U_{1} \right)^{*} \left( \frac{1}{\sqrt{6}}T_{1} + T_{3} \right) \\
  & - S^{*}_{1}S_{2} + \frac{1}{2}|S_{3}|^{2} - \frac{1}{\sqrt{6}}T^{*}_{1} \left( \frac{1}{\sqrt{6}}T_{1} - T_{3} \right) - \frac{1}{2}|T_{2}|^{2} \Biggl\}.
\end{split}
\end{equation}
Although it would be difficult to determine each spin-dependent term separately, Ref. \cite{chiEFT-1992} predicts the depolarization is sensitive to the tensor force. 

Current theoretical models for $YN$ systems have been built by fitting the parameters to the only existing total cross-section data. As a result, the predictions of some spin observables for $\Lp$ scattering by current theoretical models definitely differ, as shown in Figure \ref{fig-theopred}. \textcolor{red}{To impose strong constraints on the models, we decided to perform a new $\Lp$ scattering experiment at J-PARC to measure spin observables.}

%dcs, pol, depol理論予想
\begin{figure}[h!]
 \begin{center}
   \includegraphics[width=15cm]{theopred.png}
   \caption{Differential cross-sections, polarization, and depolarization for $\Lp$ scattering at 633 MeV/$c$ (i.e., at the $\Sigma^{+}n$ threshold) \cite{chiEFT-2021}. Predictions for NLO13 (600) (solid line), NLO19 (600) (dash-dotted), J\"{u}lich '04 (dashed), and Nijmegen NSC97f (dotted) are presented.}
   \label{fig-theopred}
 \end{center}
\end{figure}


%%%%%%%%2023/10/31 12:23
%%%%
\subsection{Analyzing power}
\label{sec-anapow}

Analyzing power for the two-body $\Lp$ scattering event can be obtained by the differential cross-section of the left-scattered $\Lambda$ against the $\Lambda$ spin axis $\left( \frac{d\sigma}{d\Omega} \right)_{L}$, and the one of the right-scattered $\Lambda$ $\left( \frac{d\sigma}{d\Omega} \right)_{R}$ as
\begin{equation}
  A_{y}(\theta) = \frac{\pi}{2\PL} \frac{\left( \frac{d\sigma}{d\Omega} \right)_{L} - \left( \frac{d\sigma}{d\Omega} \right)_{R}}{\left( \frac{d\sigma}{d\Omega} \right)_{L} + \left( \frac{d\sigma}{d\Omega} \right)_{R}},
  \label{eq-anapow}
\end{equation}
where $\theta$ is the scattering angle of scattered $\Lambda$, $\PL$ is the beam $\Lambda$ polarization. Each left/right differential cross-section can be defined as 
\begin{align}
  \left( \frac{d\sigma}{d\Omega} \right)_{L} &= \frac{1}{\pi} \left( \frac{d\sigma}{d\Omega} \right)_{0} \int^{\pi/2}_{-\pi/2} (1+A_{y}(\theta)\PL\cos{\phi})d\phi, \\
  &= \left( \frac{d\sigma}{d\Omega} \right)_{0} (1+\frac{2\PL}{\pi}A_{y}(\theta)), \\
  & \nonumber \\
   \left( \frac{d\sigma}{d\Omega} \right)_{R} &= \frac{1}{\pi} \left( \frac{d\sigma}{d\Omega} \right)_{0} \int^{3\pi/2}_{\pi/2} (1+A_{y}(\theta)\PL\cos{\phi})d\phi, \\
  &= \left( \frac{d\sigma}{d\Omega} \right)_{0} (1-\frac{2\PL}{\pi}A_{y}(\theta)), \\
  & \nonumber \\
  \left( \frac{d\sigma}{d\Omega} \right)_{0} &= \frac{1}{2} \left(\left( \frac{d\sigma}{d\Omega} \right)_{L} + \left( \frac{d\sigma}{d\Omega} \right)_{R} \right)
\end{align}
where $\left( \frac{d\sigma}{d\Omega} \right)_{0}$ is the differential cross-section for unpolarized beam $\Lambda$, and $\phi$ is the crossing angle between the normal vector of $\Lambda$ production plane and $\Lp$ scattering plane. The schematic of these planes with the definitions of $\phi$, and $\theta$ can be seen in Figure \ref{fig-planes}. Here, the differential cross-section for polarized beam $\Lambda$ can also be obtained as 
\begin{equation}
  \left( \frac{d\sigma}{d\Omega} \right) = \left( \frac{d\sigma}{d\Omega} \right)_{0} (1+A_{y}(\theta)\PL\cos{\phi}).
  \label{eq-dcs_theo}
\end{equation}
Experimentally, final result for $\left( \frac{d\sigma}{d\Omega} \right)_{L}$ is calculated as the weighted mean of $ \left( \frac{d\sigma}{d\Omega} \right)_{L}$ derived from the left-scattered $\Lambda$, and the one from the right-scattered recoil proton. Final result for $\left( \frac{d\sigma}{d\Omega} \right)_{R}$ can be calculated by the same way. 

%散乱平面の図
\begin{figure}[h!]
 \begin{center}
   \includegraphics[width=12cm]{planes.png}
   \caption{Schematic of the $\Lambda$ production and $\Lp$ scattering planes. $\phi$ is the crossing angle between the normal vector of $\Lambda$ production plane and $\Lp$ scattering plane. $\theta$ is the scattering angle of scattered $\Lambda$. $\theta_{p}$ is the scattering angle of decay proton derived from the scattered $\Lambda$.}
   \label{fig-planes}
 \end{center}
\end{figure}

%%%%
\subsection{Depolarization}
\label{sec-depo}

Depolarization for the two-body $\Lp$ scattering event can be written with the polarization of scattered $\Lambda$ ($\PLscat$) derived from polarized beam $\Lambda$ ($\PL$), the polarization of scattered $\Lambda$ derived from unpolarized beam $\Lambda$ $P$, and the polarization of beam $\Lambda$ in the scattering plane $\PL\cos{\phi}$ as
\begin{align}
  \PLscat &= \frac{P+\depo\PL\cos{\phi}}{1+P\PL\cos{\phi}}, \\
  &= \frac{2}{\alpha} \frac{N_{U}-N_{D}}{N_{U}+N_{D}},
  \label{eq-depo}
\end{align}
where $\alpha$ is the asymmetry parameter, and $N_{U}$ ($N_{D}$) is the number of decay proton emitted to the up (down) side on the $\Lp$ scattering plane. The polarization of beam $\Lambda$ in the scattering plane can be written as $\PL\cos{\phi}$ depending on the definition of the $\Lambda$ polarization axis of the $\Lp$ scattering plane. Experimentally, we will measure the $N_{U}$ and $N_{D}$ to calculate the polarization of scattered $\Lambda$ derived from polarized beam $\Lambda$ $\PLscat$. 

%L beam polarizationについては偏極度測定の章で述べることを書くこと。
%%%
\subsubsection{Beam $\Lambda$ polarization}
In Equation (\ref{eq-anapow}), (\ref{eq-dcs_theo}), and (\ref{eq-depo}), the beam $\Lambda$ polarization ($\PL$) is essential to derive the analyzing power ($\anapow$) and depolarization ($\depo$). \textcolor{red}{In this paper, for the future $\Lp$ scattering experiment, we established a beam $\Lambda$ polarization measurement method and $\Lp$ scattering events identification method using the $\PiKL$ reaction data obtained in the J-PARC E40 experiment.}


%%%%%%%%%%%%%%%%%%%%%%%%%%%%%%%%%%%%%%
%Present $YN$ scattering experiment: J-PARC E40
%%%%%%%%%%%%%%%%%%%%%%%%%%%%%%%%%%%%%%
%%%%%
\clearpage
\section{Present $\SPMp$ scattering experiment: \\J-PARC E40}
\textcolor{red}{The purpose of the J-PARC E40 experiment was originally to derive the differential cross section of $\SPMp$ scattering. See Figure \ref{fig-dcs_SMp}, Figure \ref{fig-dcs_SMpLn}, and Figure \ref{fig-dcs_SPp} for the measured differential cross sections. This paper focused on $\PiKL$ reaction data included in $\SM$ run data and conducted basic research on next-generation $\Lp$ scattering experiments at J-PARC.}

%%%%%
\subsection{Outline of this paper}
The setup for this experiment (J-PARC E40) is described in Chapter \ref{chap-exp}. The overall analysis flow of $\Lp$ scattering event identification and beam $\Lambda$ polarization measurement is briefly summarized in Chapter \ref{chap-Anaflow}. The identification of $\Lambda$ derived from the $\PiKL$ reaction is explained in Chapter \ref{chap-Lbeam}. In chapter \ref{chap-Pl}, the beam $\Lambda$ polarization measurement is described. In chapter \ref{chap-Lp_2p}, the $\Lp$ scattering identification is described. Finally, the results and considerations toward the next-generation $\Lp$ scattering experiment are summarized in chapter \ref{chap-summary}.


\begin{comment}

The J-PARC E40 experiment was performed at K1.8 beamline in the J-PARC hadron experimental facility in the separated periods of June 2018, February to March 2019 (for $\SMp$ scattering data taking), April 2019, and May to June 2020 (for $\SPp$ scattering data taking). It aimed to achieve the high-statistic $\SPMp$ scattering production to investigate the validity of the meson-exchange model in the $\SN (I = 1/2)$ channel and the quark Pauli effect in the $\SN (I = 3/2)$ channel. \textcolor{red}{ Especially in the $^3S_1$ state in the $\SPp$ system, if two $u$ quarks have the same spin and color, at least one $u$ quark must be excited to the higher orbit to satisfy the Pauli principle. Some expect this mechanism to cause a strong short-range repulsion, which makes $^3S_1$ state in the $\SN (I = 3/2)$ channel have a large repulsive core. }    \textcolor{red}{ This experiment also measured the differential cross-section of the $\SMpLn$ inelastic scattering using the $\SM$ beams produced by the $\PiKSM$ reaction. }The high intensity $\pi$ beams (20 M/spill, 5.2 seconds cycle with a beam duration of 2 seconds) were irradiated to a liquid hydrogen (LH$_2$) target. The incident $\SP$s were produced with a momentum range of $0.44-0.80$ GeV/$c$ using 1.41 GeV/$c$ $\pP$ beams. The incident $\SM$s were produced with a momentum range of $0.47-0.85$ GeV/$c$ using 1.33 GeV/$c$ $\pM$ beams. 

%簡単な実験セットアップの特徴説明
The $\Sigma^{\pm}$ beams were tagged by missing mass of the $\pipKX$ reaction, where the momenta and tracks of $\pi$ beams and outgoing $K^+$ were analyzed by K1.8 beamline and the forward magnetic spectrometer (KURAMA), respectively. Recoil protons knocked out by $\Sigma^{\pm}p$ scatterings were detected by the CATCH system \cite{Aka-2020}, which is a cylindrical detector cluster consisting of a cylindrical scintillation fiber tracker (CFT), a BGO calorimeter, and a scintillator hodoscope (PiID). \textcolor{red}{ The CATCH system has sufficient rate tolerance for high-intensity beam conditions and acceptance ($-1<\costcm<0.5$) for recoil protons from $\SPMp$ scatterings. } The CFT measures the track direction of scattered particles, the BGO calorimeter measures the energy deposit of scattered particles, and PiID checks whether particles penetrate the BGO calorimeter. The $\Sigma^{\pm}p$ elastic scattering can be identified by requiring the kinematical consistency referring to the kinetic energy of the recoil proton ($\Delta E$ method) and the momentum of the scatted $\Sigma^{\pm}$ ($\Delta p$ method). This analysis technique allows us to study the $\Sigma^{\pm}p$ elastic scattering without detecting the $\Sigma^{\pm}$ beams produced inside the LH$_2$ target. 

Finally, the J-PARC E40 experiment accumulated $\sim70 \rm{M}\ \SP, \sim17 \rm{M}\ \SM$, about 100 times more than the past KEK experiment \cite{KEK1}\cite{KEK2}\cite{KEK3}. \textcolor{red}{ Approximately 4500 $\SMp$ elastic scattering events and 2300 $\SMpLn$ inelastic scattering events were identified, and their differential cross-sections were already reported in Ref. \cite{Miwa-SMp}\cite{Miwa-SMLn}. The derived differential cross-sections of $\SMp$ elastic scattering and $\SMpLn$ inelastic scattering are shown in Figure \ref{fig-dcs_SMp} and Figure \ref{fig-dcs_SMpLn}. The differential cross-sections were measured at angular intervals of $d\costcm=0.1$ for each momentum region, and the statistical error was approximately 10\%, achieving high accuracy suitable for constraining the theoretical model. These data are useful for examining ${\bf (8_{\rm{s}})}$ and ${\bf (8_{\rm{a}})}$, as seen in Table \ref{table-BBint}. In addition to that, using approximately 2400 $\SPp$ elastic scattering events identified, the differential cross-section measurement and phase shift analysis in the $^3S_1$ and $^1P_1$ states were performed in Ref. \cite{Nana-SPp}. The derived differential cross-sections of $\SPp$ elastic scattering and the phase shifts $\delta_{^3S_1}$ and $\delta_{^1P_1}$ are shown in Figure \ref{fig-dcs_SPp} and Figure \ref{fig-phaseshift_SPp}. The differential cross-sections were measured at angular intervals of $d\costcm=0.1$ for each momentum region, and the statistical error was approximately 20\%. The phase shifts of the $^3S_1$ and $^1P_1$ states were derived using the phase-shift analysis for the obtained differential cross-sections. Since the $\delta_{^3S_1}$ is related to the strength of repulsive force due to the quark Pauli effect, these precise phase shift data are essential to make current $\BB$ interaction models more realistic. }

%バイプロとしてPiKL反応データを取得したこと。
We identified the charge of scattered particles by bending the tracks of charged particles by the magnetic field applied when they pass through the KURAMA magnet. That makes only positive-charge particles join the KURAMA tracking analysis. During the $\SM$ production, to tag the ($\pM$, $K^+$) reaction, Scattered-Aerogel Counter (SAC), placed in front of the KURAMA magnet, vetoed $\pP$s. SAC has a refractive index of 1.10, which emits the Cherenkov light for the $\pP$s (typical momentum range of $0.9-1.4$ GeV/$c$). SAC has a $\pP$ rejection efficiency over 99\% under the data-taking rate of 300 kHz \cite{Koba-2016}. As a result, 1\% of $\pP$ data SAC accepted were accumulated simultaneously. \textcolor{red}{ The data include, for example, the multiple $\pi$ production (i.e., the $\multipi$ reaction) and the $\PiKL$ reaction when $\pP$ from the $\kzdecay$ decay passed through the KURAMA magnet. In this work, as a basic study for the J-PARC E86 (a next-generation $\Lp$ scattering experiment), the beam $\Lambda$ polarization measurement and the establishment of $\Lp$ scattering analysis method were performed using the $\PiKL$ reaction data accumulated in the J-PARC E40. }

\end{comment}

%%%%%%%%%%%%%%%%%%%%%%%%%%%%%%%%%%%%%%
%Next-generation $YN$ scattering experiment: J-PARC E86
%%%%%%%%%%%%%%%%%%%%%%%%%%%%%%%%%%%%%%
%%%%%
%\clearpage
\section{Next-generation $\Lp$ scattering experiment: \\J-PARC E86}
%E86モチベ
The next-generation $\Lp$ scattering experiment, J-PARC E86 \cite{Miwa-LpProp}, is scheduled to measure the $\Lp$ scattering differential cross-section and spin observables, such as analyzing power and depolarization. These spin observables would be the first experimental input in the $\Lp$ channel. %Considering that the interaction distance that can be resolved by elastic scattering is equivalent to the wavelength of the incident particle ($1/k = \hbar/p_{CM}$), J-PARC E86 can investigate the interaction distance range of $\sim0.63-1.14$ fm in the CM frame using the beam $\Lambda$ with a momentum range of $0.4-0.8$ GeV/$c$ in the Laboratory frame. 

Since $\Lambda$ and proton have isospin $I = 0$ and $I = 1/2$ respectively, the $\Lp$ channel has only the state of total isospin $I = 1/2$. Therefore, it is suitable to investigate the contribution of ALS force and the structure of $\Lp$ interaction around the $\SN$ threshold caused by the tensor force of $^3S_1 - ^3D_1$ coupling. %Many of data from past $\Lp$ scattering experiments were mainly from the era when bubble chambers were used, and their momentum regions were below the range targeted by J-PARC E86. Due to this limited experimental information, theoretical models of the $\LN$ interaction, despite being intensively studied from the $\Lambda$ hypernuclei, show quite different predictions for the $P-$wave and higher-wave momentum regions. 
In the $P$-wave and higher-wave regions, the strength of spin-dependent interactions, such as the LS force, becomes very large. Therefore, the spin observables measurement in J-PARC E86 is extremely important in placing strong constraints on the theoretical models. 

%簡単な実験セットアップの特徴説明
%At the K1.1 beamline of the J-PARC hadron experimental facility, we will use the intensity of 30 M/spill (5.2 seconds cycle with a beam duration of 2 seconds) $\pM$ beam with the momentum of 1.05 GeV/$c$ and liquid hydrogen (LH$_2$) target \textcolor{red}{with 30 cm long}. The incident $\Lambda$ produced by the $\PiKL$ reaction will be tagged by the missing mass of the $\PiKX$ reaction. The total cross-section of the $\PiKL$ reaction was expected to be large ($\sim900\ \mu$b) at this momentum range \cite{Baker}. 

\textcolor{red}{The $\Lambda$ beam is produced via the $\PiKL$ reaction with a total cross-section $\sim900\ \mu$b \cite{Baker} using $\pM$ beam momentum of 1.05 GeV/$c$. According to Ref. \cite{Baker}, the $\Lambda$ produced in the $\pM$ beam momentum range $0.931-1.334$ GeV/$c$ is $\sim100\%$ polarized at the forward $\Kz$ scattering angle, as shown in Figure \ref{fig-PL_Baker}. However, this data had the problem of low statistics and remained uncertain. Therefore, in this paper, we confirmed that the beam $\Lambda$ is automatically polarized, and verified whether it can be used for the spin observable measurement in the J-PARC E86 experiment, using the $\PiKL$ reaction data obtained in the J-PARC E40 experiment.}

\begin{comment}

Figure \ref{fig-E86setup} shows the experimental setup of J-PARC E86 \cite{Miwa-LpProp}. To reconstruct $\Kz$, $\pP$ and $\pM$ from the $\kzdecay$ decay will be detected by the forward magnetic spectrometer (SKS) \cite{K1.8} and the CATCH system, respectively. \textcolor{red}{ The momentum resolution of SKS is $\Delta p/p = 10^{-3}$ (FWHM), which is 10 times better than KURAMA, which should improve the missing mass resolution. As a result, the S/N ratio of $\Lambda$ identification is expected to be improved. }

\textcolor{red}{ To suppress the occurrence of multi-hit tracks and multi-signals on one wire as much as possible, J-PARC E86 plans to install a new 12-layer fiber tracker (BFT-D) downstream of the SKS magnet instead of the drift chamber. BFT-D measures beam timing with a timing resolution of $\sim1$ ns ($\sigma$). At this time, to select the correct beam track, we will analyze events that have passed through the hit segment of a timing hodoscope (BH2) with a time resolution of $\sim100$ ps ($\sigma$), which is also located downstream of the magnet. To reconstruct the beam momentum, the beam trajectory will be reconstructed using the inverse transfer matrix so that the track reconstructed by BH2 and BFT-D connects to the hit position of the upstream fiber tracker (BFT-U). The timing resolution of BFT-U is also $\sim1$ ns ($\sigma$). To prevent connections between non-corresponding hits upstream and downstream of the magnet, we will again require the reconstructed upstream track to pass through the hit segments of the timing hodoscope BH1, which has a timing resolution of $\sim100$ ps ($\sigma$). However, to prevent multiple track events in the same BH1 segment, the BH1 will be designed with a segment width of $\sim5$ mm and a total segment count of $\sim45$ to cover a beam expanse of $\sim200$ mm. Furthermore, the position resolution at the target for the beam track is important for improving the accuracies of the beam particle analysis and the incident $\Lambda$ identification. If this position resolution is improved, it is expected that the measurement accuracy of the flight length of $\Kz$ will be improved. As a result, the resolution of the missing mass used for beam $\Lambda$ identification will be improved. Therefore, a new fiber tracker VFT1 (X and UV fiber configuration) will be installed in J-PARC E86 just upstream of the LH$_2$ target to measure the $xy$ position of the beam track at the target accurately. The momentum resolution of the K1.1 beamline is expected to be $\Delta p/p=3.8\times10^{-3}$ \cite{Miwa-LpProp}. The $\Lp$ scattering event will be identified by detecting the final state particles by the CATCH system. We will use kinematic consistency analyses to identify $\Lp$ scattering events more accurately, as we did in the J-PARC E40 experiment. The SKS spectrometer has better momentum resolution $\Delta p/p=10^{-3}$ (FWHM) by bending the charged particle's track 100 degrees. The $\pM$ beam will not penetrate the downstream detectors because its track will be bent to the opposite side against the positive-charged scattered particles. This would reduce the counting rate and allow us to easily identify the $\pP$ events derived from the $\kzdecay$ decay. These experimental techniques make earning a vast amount of the scattering events possible. }

We requested a 34-day beam time, including a 29-day production run and 5-day commissioning and calibration runs, as the first stage to measure the differential cross-sections and the analyzing power by accumulating 50 M momentum-tagged beam $\Lambda$s. We also requested an additional 34-day beam time as the second stage, including the production, commissioning, and calibration runs with the same ratio to measure the depolarization and improve the differential cross-section accuracy and analyzing power measurements. Since the momentum dependence of the differential cross-section is expected to become large toward the $\SN$ threshold range \cite{Miwa-LpProp}, the differential cross-section will be measured with the beam $\Lambda$ momentum step of $dp_{\Lambda}=50$ MeV/$c$. The analyzing power and depolarization will be measured with the beam $\Lambda$ momentum step of $dp_{\Lambda}=100$ MeV/$c$. 
\end{comment}


\begin{figure}[!h]
  \begin{center}
   \includegraphics[width=15cm]{PL_Baker.png}
   \caption{Induced polarization of $\Lambda$ in the $\PiKL$ reaction in the momentum range of $0.931-1.334$ GeV/$c$ measured in Ref. \cite{Baker}. The horizontal axis shows the scattering angle of $\Kz$ in the CM frame ($\costcm$). High polarization was obtained for all momentum ranges here.}
   \label{fig-PL_Baker}
 \end{center}
\end{figure}

\begin{comment}
\begin{figure}[!h]
  \begin{center}
   \includegraphics[width=15cm]{E86setup.png}
   \caption{Experimental setup of the J-PARC E86 \cite{Miwa-LpProp}. To accurately measure the beam timing and momentum, beam timing hodoscope BH1 and beam fiber tracker BFT-U will be installed upstream of the D magnet, and BFT-D and BH2 will be installed downstream of the magnet. To measure the beam position at the LH$_{2}$ target, VFT will be installed just upstream of the target.}
   \label{fig-E86setup}
 \end{center}
\end{figure}
\end{comment}

%\end{document}

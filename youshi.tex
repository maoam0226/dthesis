%\documentclass[a4paper,12pt,twoside,openany]{jsarticle}
\documentclass[a4paper,11pt,twoside,twocolumn]{article}
%\documentclass[a4paper,10.5pt]{jsarticle}


%
\usepackage{amsmath,amssymb}
\usepackage{bm}
\usepackage[dvipdfmx]{graphicx,color}
\usepackage[super]{nth} %1st, 2nd ...を出力
\usepackage{ascmac}
\usepackage{comment}
\usepackage{dirtytalk} %クォーテーションマーク
%
%\setlength{\textwidth}{\fullwidth}
\setlength{\textheight}{40\baselineskip}
\addtolength{\textheight}{\topskip}
\setlength{\voffset}{-0.2in}
\setlength{\topmargin}{0pt}
\setlength{\headheight}{0pt}
\setlength{\headsep}{0pt}
%
\oddsidemargin -5.4mm
\evensidemargin -5.4mm 
\textwidth 160mm
\topmargin -5.4mm 
\headheight 0pt 
\headsep 0pt 
\textheight 237mm
%
\columnsep 5mm
%


\begin{comment}

\usepackage{ascmac}
\usepackage{textcomp}
%\usepackage[dviout]{graphicx}
\usepackage[dvipdfmx]{graphicx,color}
\usepackage{wrapfig}
\usepackage{ccaption}
\usepackage{color}
%\usepackage{jumoline}
\usepackage{umoline}
\usepackage{fancybox}
\usepackage{pifont}

\usepackage{float}

\usepackage{amsmath,amssymb}
\usepackage{bm}
\usepackage{graphicx}
\usepackage{subfigure}
\usepackage{verbatim}
\usepackage{makeidx}
\usepackage{accents}
%\usepackage{slashbox}


\pagestyle{empty}
%\setlength{\textwidth}{\fullwidth}
\setlength{\textwidth}{156mm}
\setlength{\textheight}{42\baselineskip}
%\setlength{\textheight}{45\baselineskip}

\addtolength{\textheight}{\topskip}
\setlength{\voffset}{-0.75in}

\setlength{\oddsidemargin}{27mm}
\addtolength{\oddsidemargin}{-1.2in}
\setlength{\evensidemargin}{20mm}
\addtolength{\evensidemargin}{-1in}

\renewcommand{\baselinestretch}{0.8} %% 行間
%\renewcommand{\bibname}{参考文献}

\end{comment}

\graphicspath{{./pictures/youshi/}}
\begin{document}

%\setlength{\abovedisplayskip}{3pt} % 上部のマージン
%\setlength{\belowdisplayskip}{2pt} % 下部のマージン


\abovecaptionskip=-5pt
\belowcaptionskip=-5pt

\makeatletter

\newcommand{\msolar}{M_\odot}

\newcommand{\anapow}{A_{y}(\theta)}
\newcommand{\depo}{D^{y}_{y}(\theta)}

\newcommand{\figcaption}[1]{\def\@captype{figure}\caption{#1}}
\newcommand{\tblcaption}[1]{\def\@captype{table}\caption{#1}}
\newcommand{\klpionn}{K_L \to \pi^0 \nu \overline{\nu}}
\newcommand{\kppipnn}{K^+ \to \pi^+ \nu \overline{\nu}}
\newcommand{\hfl}{{}_\Lambda^4\rm{H}}
\newcommand{\htl}{{}_\Lambda^3\rm{H}}
\newcommand{\hefl}{{}_\Lambda^4\rm{He}}
\newcommand{\hefil}{{}_\Lambda^5\rm{He}}
\newcommand{\lisl}{{}_\Lambda^7\rm{Li}}
\newcommand{\benl}{{}_\Lambda^9\rm{Be}}
\newcommand{\btl}{{}_\Lambda^{10}\rm{B}}
\newcommand{\bel}{{}_\Lambda^{11}\rm{B}}

\newcommand{\nfl}{{}_\Lambda^{15}\rm{N}}
\newcommand{\osl}{{}_\Lambda^{16}\rm{O}}
\newcommand{\ctl}{{}_\Lambda^{13}\rm{C}}
\newcommand{\pbtl}{{}_\Lambda^{208}\rm{Pb}}

\def\vector#1{\mbox{\boldmath$#1$}}
\newcommand{\Kpi}{(K^-,\pi^-)}
\newcommand{\piKz}{(\pi^-,K^0)}
\newcommand{\pPK}{(\pi^+,K^+)}
\newcommand{\pMK}{(\pi^-,K^+)}
\newcommand{\pPMK}{(\pi^{\pm},K^+)}

\newcommand{\eeK}{(e,e' K^+)}
\newcommand{\gK}{(\gamma + p \to \Lambda + K^+)}
\newcommand{\PiKL}{\pi^-  p \to K^0 \Lambda}
\newcommand{\multipi}{\pi^-  p \to \pi^-\pi^-\pi^+p}
\newcommand{\PiKX}{\pi^-  p \to K^0 X}
\newcommand{\PiKSM}{\pi^-  p \to K^+ \Sigma^-}
\newcommand{\pipKS}{\pi^{\pm}p \to K^+ \Sigma^{\pm}}
\newcommand{\pipKX}{\pi^{\pm}p \to K^+ X}
\newcommand{\pipLn}{\pi^- p \to \Lambda n}
\newcommand{\PiKS}{\pi^{-}p \to K^{0}\Sigma^{0}}

\newcommand{\kzdecay}{K^0 \to \pi^+ \pi^-}
\newcommand{\kzsd}{K^0_s \to \pi^+ \pi^-\ \rm{or}\ \pi^0 \pi^0}
\newcommand{\Ldecay}{\Lambda\to p\pM}
\newcommand{\scatldecay}{\Lambda'\to p\pM}




\newcommand{\triton}{{}^3\rm{H}}

\newcommand{\BB}{B_{8}B_{8}}
\newcommand{\SM}{\Sigma^{-}}
\newcommand{\SP}{\Sigma^{+}}
\newcommand{\Sz}{\Sigma^{0}}
\newcommand{\SMp}{\Sigma^{-}p}
\newcommand{\SMn}{\Sigma^{-}n}
\newcommand{\SPp}{\Sigma^{+}p}
\newcommand{\SPn}{\Sigma^{+}n}
\newcommand{\Sp}{\Sigma p}
\newcommand{\SPMp}{\Sigma^{\pm}p}
\newcommand{\SPM}{\Sigma^{\pm}}
\newcommand{\SPdecay}{\Sigma^+ \to \pi^0 p}
\newcommand{\SMdecay}{\Sigma^- \to \pi^- n}
\newcommand{\SMpLn}{\Sigma^- p \to \Lambda n}

\newcommand{\XM}{\Xi^{-}}
\newcommand{\Xz}{\Xi^{0}}

\newcommand{\pM}{\pi^{-}}
\newcommand{\pP}{\pi^{+}}
\newcommand{\pZ}{\pi^{0}}
\newcommand{\pPM}{\pi^{\pm}}
\newcommand{\KP}{K^{+}}
\newcommand{\KM}{K^{-}}
\newcommand{\Kz}{K^{0}}
\newcommand{\Lp}{\Lambda p}
\newcommand{\LpLX}{\Lambda p \to \Lambda X}

\newcommand{\LN}{\Lambda N}
\newcommand{\SN}{\Sigma N}
\newcommand{\LNtoSN}{\Lambda N\to\Sigma N}
\newcommand{\LS}{\Lambda - \Sigma}

%\newcommand{\dp}{\Delta p}
%\newcommand{\dE}{\Delta E}

\newcommand{\dcs}{d\sigma/d\Omega}
\newcommand{\fdcs}{\frac{d\sigma}{d\Omega}}
\newcommand{\dz}{\Delta z}
\newcommand{\dzkz}{\Delta z_{K^{0}}}


\newcommand{\bgct}{\beta\gamma c\tau}

\newcommand{\costp}{\cos{\theta_p}}
\newcommand{\costkz}{\cos{\theta_{K0,CM}}}
\newcommand{\costcm}{\cos{\theta}_{CM}}
\newcommand{\PL}{P_{\Lambda}}
\newcommand{\PLall}{P_{\Lambda,\ all}}
\newcommand{\PLsele}{P_{\Lambda,\ selected}}
\newcommand{\errPL}{\sigma(P_{\Lambda})}

\newcommand{\rud}{r_{ud}}
\newcommand{\errrud}{\sigma(\rud)}

\newcommand{\accPL}{\epsilon_{\PL}}
\newcommand{\erraccPL}{\sigma(\epsilon_{\PL})}

\newcommand{\PLscat}{P_{\Lambda'}}
\newcommand{\effPLw}{\epsilon_{\PL,\ w/}}
\newcommand{\erreffPLw}{\sigma(\epsilon_{\PL,\ w/})}
\newcommand{\effPLwo}{\epsilon_{\PL,\ w/o}}
\newcommand{\erreffPLwo}{\sigma(\epsilon_{\PL,\ w/o})}

\newcommand{\chisq}{\chi^{2}}
\newcommand*{\rom}[1]{\expandafter\@slowromancap\romannumeral #1@}
\makeatother

%\renewcommand{\refname}{}
\vspace{-20pt}
\title{%{\small 令和5年度 博士論文要旨} \\ %J-PARCにおける次世代$\Lp$散乱実験に向けたビーム$\Lambda$偏極度の測定\\ 
Measurement of the beam $\Lambda$ polarization \\for the next $\Lp$ scattering experiment at J-PARC}
\author{An abstract of dissertation presented to\\
Department of Physics, Graduate School of Science and Faculty of Science, Tohoku University\\
}
%\author{東北大学大学院理学研究科 物理学専攻 原子核物理研究室\\ 坂尾 珠和}
\date{Tamao Sakao\\FY2023}
\maketitle

%%%%%
\section{Introduction}
\label{sec-intro}

This paper focuses on baryon-baryon ($\BB$) interactions, aiming to understand the nuclear force from quark degree of freedom by extending to flavor SU(3) symmetry. The J-PARC E40 experiment measured $\SPp$ ($\SN (I=3/2)$) and $\SMp$ ($\SN (I=1/2)$) cross-sections to investigate the repulsiveness from the quark Pauli effect in ${\bf(8)_s}$ and ${\bf(10)}$ terms. Next, the new $\Lp$ scattering experiment (J-PARC E86) will measure the differential cross-section and spin observables. Building realistic $\LN$ interactions by accumulating two-body scattering data is essential for studying many-body systems such as hypernuclei and neutron stars.

As the basic research for J-PARC E86, we performed the beam $\Lambda$ polarization ($\PL$) measurement and the $\Lp$ scattering identification using the $\PiKL$ reaction data taken in J-PARC E40. In the $\PL$ measurement, we address the necessity of a highly polarized $\Lambda$ beam. Ref. \cite{Baker} reported that the beam $\Lambda$ is almost 100\% polarized in the forward $\Kz$ scattering angular range at $E_{CM}=1847$ MeV (same as J-PARC E40), as shown in Figure \ref{fig-Baker1847}, but uncertainty existed due to small statistics. Therefore, this paper has verified the result of Ref. \cite{Baker} by measuring $\PL$. In the $\Lp$ scattering identification, we developed kinematical consistency analysis methods. The main backgrounds, such as multiple $\pi$ production, $pp$ elastic scattering, and $\Ldecay$ decay, were removed by optimized cuts and S/N was estimated by Geant4 simulation. 

%Baker  ECM=1847 MeV
\begin{figure}[h]
  \centering
  \includegraphics[width=6cm]{Baker1847.eps}
  \caption{$\Lambda$ polarization in the $\PiKL$ reaction at the energy range of $E_{CM}=1847$ MeV (same as J-PARC E40) \cite{Baker}.}
  \label{fig-Baker1847}
\end{figure}

%%%%%
\section{Analsis \rom{1}: Beam $\Lambda$ identification}
\label{sec-LbeamID}

The $\Kz$ from the $\PiKL$ reaction was reconstructed using $\pM$ and $\pP$ vectors detected by a cylindrical detector cluster (CATCH) and a forward magnetic spectrometer (KURAMA), respectively. Then, $\Lambda$ was tagged from the missing mass of the $\PiKX$ reaction. The $\PL$ measurement required CATCH to detect decay products of $\Ldecay$ decay (case \rom{1}). The $\Lp$ scattering identification required CATCH to detect recoil proton and decay products of $\scatldecay$ decay (case \rom{2}).

Each missing mass in case \rom{1} and case \rom{2} was fitted with two Gaussians and a \nth{3}-order polynomial. The $\Lambda$ region was defined as the $1.0707 - 1.1626$ GeV/$c^{2}$ range. The yields of $\Lambda$ peak in case \rom{1} and case \rom{2} were $6.976\times10^{4}$ (S/N = 1.523), and $2.721\times10^{3}$ (S/N = 1.471).

%%%%%
\section{Analysis \rom{2}: Beam $\Lambda$ polarization measurement}
\label{sec-Pl}

In case \rom{1}, $\PL$ can be obtained by fitting the emission angle distribution of decay protons in the rest frame of $\Lambda$ ($\costp$) with
\begin{equation}
  \frac{dN}{d\costp} = \frac{N_0}{2}(1+\alpha\PL\costp), 
  \label{eq-costp}
\end{equation}
where $N$ is the yield of each $\costp$ bin, $N_0$ is the total yield of $\costp$, and $\alpha$ is an asymmetry parameter ($\alpha=0.750\pm0.009\pm0.004$ \cite{Alpha}). $\PL$ was obtained in the $\Kz$ scattering angular range ($\costkz$) of 0.6 - 1.0 with an angular step of $d\costkz = 0.05$. Due to the uncertainty in the correction of CATCH acceptance, an angular region deviates from the expected distribution in $\costp$. Therefore, the angular region used for fitting was optimized using a $\chisq$ test. Figure \ref{fig-Pl} shows the results of the E40 data with a selected degree of freedom (red points), the results of the E40 data with a full degree of freedom (blue points), and results in Ref. \cite{Baker} (green points). 

The average $\PL$ value obtained with a selected degree of freedom was $0.917\pm0.059\pm0.003$ in the $\costkz$ range of 0.6 - 0.85. Therefore, we concluded that by selecting this specific $\costkz$ region, we could use a beam $\Lambda$ with high polarization, resulting in that $\Lp$ spin observable measurement is fully possible in J-PARC E86.

\begin{figure}[h]
  \centering
  \includegraphics[width=8cm]{Pl.eps}
  \caption{$\Lambda$ polarizations in the $\PiKL$ reaction.}
  \label{fig-Pl}
\end{figure}


%%%%%
\section{Analysis \rom{3}: $\Lp$ scattering identification}
\label{sec-Lp_2p}

In case \rom{2}, $\Lp$ scattering can be identified by kinematical consistency analysis called \say{$\Delta E\Delta p$ method}. Here, the kinetic energy of the recoil proton and momentum of scattered $\Lambda$ ($\Lambda'$) were kinematically calculated assuming the $\Lp$ scattering kinematics. %The contamination of backgrounds ($pp$ elastic scattering and $\Ldecay$ decay) was estimated by Geant4 simulation. 
As a result, the yield of $\Lp$ scattering events was expected at most 14 with the estimated S/N of 18. Therefore, we concluded that it is possible to identify $\Lp$ scattering events with a high S/N using the analysis method developed in this paper, although it is expected that there are still unknown background events that the simulation cannot reproduce. When we obtain high statistics in J-PARC E86, we can rigorously estimate the background structure and yield of $\Lp$ scatterings. 

%%%%%
\section{Summary}
\label{sec-summary}

As the basic research for J-PARC E86, the $\PL$ measurement and the $\Lp$ scattering identification were performed using the $\PiKL$ reaction data taken in J-PARC E40. %$\Kz$ was reconstructed by $\pM$ and $\pP$ detected by CATCH and KURAMA, respectively. Then, $\Lambda$ was tagged by the missing mass of the $\PiKX$ reaction. 

%The beam $\Lambda$ polarization ($\PL$) was obtained by fitting the emission angle distribution ($\costp$) of decay protons of $\Ldecay$ decay using Equation (\ref{eq-costp}) in case \rom{1}. As a result, the average $\PL$ value measured in E40 data was $\sim0.917\pm0.059\pm0.003$ in the $\costkz$ range of $0.6 - 0.85$. Therefore, we concluded that by selecting this specific $\costkz$ region, we could use a beam $\Lambda$ with high polarization, resulting in that $\Lp$ spin observable measurement is fully possible in J-PARC E86.
The $\PL$ measurement in case \rom{1} used the tagged $\Lambda$ of $6.976\times10^{4}$ (S/N $=1.523$) and found that the averaged $\PL$ value was $0.917\pm0.059\pm0.003$ in the $\costkz$ range of 0.6 - 0.85. Therefore, we concluded that by selecting this specific $\costkz$ region, $\Lp$ spin observable measurement is possible in J-PARC E86.

%The $\Lp$ scattering can be identified by kinematical consistency analysis (\say{$\Delta E-\Delta p$} method) in case \rom{2}. In the range of $|\Delta E|<20$ MeV and $|\Delta p|<0.05$ GeV/$c$, the S/N estimated by Geant4 simulation was 17.979, and the yield of $\Lp$ scattering events was expected at most 14. Therefore, we can say it is possible to identify $\Lp$ scattering events with a high S/N using the analysis method developed in this paper, although it is expected that there are still unknown background events that the simulation cannot predict. When we obtain high statistics in J-PARC E86, we can rigorously fit $\Delta p$ and $\Delta E$ spectra and estimate the yield of $\Lp$ scattering events. 
The $\Lp$ scattering identification in case \rom{2} used the tagged $\Lambda$ of $2.721\times10^{3}$ (S/N $=1.471$). The final yield of the scattering was 14 with the estimated S/N of 18. Therefore, we concluded that identifying $\Lp$ scatterings with a high S/N is possible by the analysis method developed in this paper. When we obtain high statistics in J-PARC E86, we can rigorously estimate the background structure and $\Lp$ scattering yield. 


%%%%%%%%%%%%%%%%%%%%%%%%%%%%%%%%%%%%%%%%%%%%%%%%%%%%%%%%%%%%%%%%%%%%%%%%%%%%%%%%%%%%%%%%%%%%%%%%%%%%%%%%%%%%%%%%%%%%
%%%%%%%%%%%%%%%%%%%%%%%%%%%%%%%%%%%%%%%%%%%%%%%%%%%%%%%%%%%%%%%%%%%%%%%%%%%%%%%%%%%%%%%%%%%%%%%%%%%%%%%%%%%%%%%%%%%%

{\scriptsize{
\begin{thebibliography}{9}
\bibitem{Baker}R. D. Baker $et\ al.$, Nucl. Phys. B 141, 29 (1978).
\bibitem{Alpha}M. Ablikim $et\ al.$, Nature Physics 15, 631 (2019).
%\bibitem{NS}P.B. Demorest et al., A two-solar-mass neutron star measured using Shapiro delay, Nature. 467., 2010 
%\bibitem{Proposal}K.Miwa et al, ”Measurement of the cross sections of $\Sigma p$ scattering”, Proposal an experiment at 50-GeV PS
\end{thebibliography}
}}
%\layout
\end{document}
